% !TEX encoding = UTF-8 Unicode.

% Based on https://github.com/Miracle0565/BUCT-Beamer-Theme

\documentclass[
10pt,
aspectratio=43,
]{beamer}
\setbeamercovered{transparent=10}
\usetheme[
%  showheader,
%  red,
  purple,
%  gray,
%  graytitle,
  colorblocks,
%  noframetitlerule,
]{Verona}

\usepackage[T1]{fontenc}
\usepackage{tikz}
\usepackage[utf8]{inputenc}
\usepackage{lipsum}
%%%%%%%%%%%%%%%%%%%%%%%%%%%%%%%
% Mac上使用如下命令声明隶书字体,windows也有相关方式,大家可自行修改
\providecommand{\lishu}{\CJKfamily{zhli}}
%%%%%%%%%%%%%%%%%%%%%%%%%%%%%%%
\usepackage{tikz}
\usetikzlibrary{fadings}
%
%\setbeamertemplate{sections/subsections in toc}[ball]
\usepackage{xeCJK}
\usepackage{listings}
\usepackage{caption}
\usepackage{subfigure}
\usefonttheme{professionalfonts}
\def\mathfamilydefault{\rmdefault}
\usepackage{amsmath}
\usepackage{multirow}
\usepackage{booktabs}
\usepackage{bm}
\setbeamertemplate{section in toc}{\hspace*{1em}\inserttocsectionnumber.~\inserttocsection\par}
\setbeamertemplate{subsection in toc}{\hspace*{2em}\inserttocsectionnumber.\inserttocsubsectionnumber.~\inserttocsubsection\par}
\setbeamerfont{subsection in toc}{size=\small}
\AtBeginSection[]{%
	\begin{frame}%
		\frametitle{Outline}%
		\textbf{\tableofcontents[currentsection]} %
	\end{frame}%
}

\AtBeginSubsection[]{%
	\begin{frame}%
		\frametitle{Outline}%
		\textbf{\tableofcontents[currentsection, currentsubsection]} %
	\end{frame}%
}

\title{高等数学C}
%\subtitle{A Simple while elegant template}
\author[P.Yu]{余沛}
\mail{peiy\_gzgs@qq.com}
\institute[Guangzhou College of Technology and Business]{Guangzhou College of Technology and Business \\
  广州工商学院}
\date{\today}
\titlegraphic[width=4cm]{logo.png}{}




%%%%%%%%%%%%%%%%%%%%%%%%%%%%%%%%
% ----------- 标题页 ------------
%%%%%%%%%%%%%%%%%%%%%%%%%%%%%%%%



\begin{document}

\maketitle

%%% define code
\defverbatim[colored]\lstI{
	\begin{lstlisting}[language=C++,basicstyle=\ttfamily,keywordstyle=\color{red}]
	int main() {
	// Define variables at the beginning
	// of the block, as in C:
	CStash intStash, stringStash;
	int i;
	char* cp;
	ifstream in;
	string line;
	[...]
	\end{lstlisting}
}
%%%%%%%%%%%%%%%%%%%%%%%%%%%%%%%%
% ----------- FRAME ------------
%%%%%%%%%%%%%%%%%%%%%%%%%%%%%%%%

\section{最值}

\begin{frame}
	\frametitle{极值的必要条件}
	首先, 我们首先给出不需要 $x_0$ 处可导性条件下可导函数的取极值的定理:
	\begin{theorem}{极大值与极小值}
		设函数$f(x)$在 $(x_0-\delta,x_0+\delta)$ 连续, 在 $(x_0-\delta,x_0)\cup(x_0,x_0+\delta)$上可导, 如果
		$$
			f'(x)\ge(\le)0, x\in(x_0-\delta,x_0),\,\,\text{且} \,\,f'(x)\le(\ge)0, x\in(x_0,x_0+\delta).
		$$
		那么函数在点$x_0$处取得极大(小)值.
	\end{theorem}
	试一试: 判断函数 $y=x^2,\,\,y=x^{\frac{1}{3}}$ 和 $y=\arcsin x$ 在 $x=0$ 处的极值情况.
\end{frame}

\subsection{导数为零: 驻点与驻点判断}
\begin{frame}
	\frametitle{导数为零: 驻点与驻点判断}
	由Fermat引理, 我们得知:
	\begin{block}{Fermat引理}
		设 $x_0$ 是 $f(x)$ 的一个极值点, 且 $f(x)$ 在 $x_0$ 处导数存在, 则
		$$
			f^{\prime}\left(x_0\right)=0 .
		$$
	\end{block}
	为了更加确定地讨论导数为零的情形, 引入驻点概念.
	\begin{block}{驻点}
		称 $x_0$ 是 $f(x)$ 的一个驻点, 如果
		$$
			f^{\prime}\left(x_0\right)=0.
		$$
	\end{block}
\end{frame}

\begin{frame}
	\frametitle{导数为零: 驻点与驻点判断}
	\begin{block}{}
		称 $x_0$ 是 $f(x)$ 的一个驻点, 如果
		$$
			f^{\prime}\left(x_0\right)=0.
		$$
	\end{block}
	\begin{block}{}
		设函数$f(x)$在 $(x_0-\delta,x_0+\delta)$ 连续, 在 $(x_0-\delta,x_0)\cup(x_0,x_0+\delta)$上可导, 如果
		$$
			f'(x)\ge(\le)0, x\in(x_0-\delta,x_0),\,\,\text{且} \,\,f'(x)\le(\ge)0, x\in(x_0,x_0+\delta).
		$$
		那么函数在点$x_0$处取得极大(小)值.
	\end{block}
	这两者可以得到高阶导数存在时的极值判定定理:
	\begin{exampleblock}{}
		设 $f'(x_0)=0$, $f''(x_0)$ 存在, 如果 $f''(x_0)>(<)0$, 则 $f(x_0)$ 为 $f(x)$ 的极小(大)值. 
	\end{exampleblock}
\end{frame}

\section{凹凸性与拐点}
\subsection{凹还是凸: 凹凸性}

\subsection{凹凸突变: 拐点}

% Thank you page
\beamertemplateshadingbackground{structure.fg!90}{structure.fg}
\begin{frame}[plain]
	\vfill
	\centering
	{
		\centering \Huge \color{white} Thank you for your attention!\\[10pt]Questions?
	}
	\vfill
\end{frame}


\end{document}


