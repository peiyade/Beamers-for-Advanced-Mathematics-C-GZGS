% !TEX encoding = UTF-8 Unicode.

% Based on https://github.com/Miracle0565/BUCT-Beamer-Theme

\documentclass[
10pt,
aspectratio=43,
]{beamer}
\setbeamercovered{transparent=10}
\usetheme[
%  showheader,
%  red,
  purple,
%  gray,
%  graytitle,
  colorblocks,
%  noframetitlerule,
]{Verona}

\usepackage[T1]{fontenc}
\usepackage{tikz}
\usepackage[utf8]{inputenc}
\usepackage{lipsum}
%%%%%%%%%%%%%%%%%%%%%%%%%%%%%%%
% Mac上使用如下命令声明隶书字体, windows也有相关方式, 大家可自行修改
\providecommand{\lishu}{\CJKfamily{zhli}}
%%%%%%%%%%%%%%%%%%%%%%%%%%%%%%%
\usepackage{tikz}
\usetikzlibrary{fadings}
%
%\setbeamertemplate{sections/subsections in toc}[ball]
\usepackage{xeCJK}
\usepackage{listings}
\usepackage{caption}
\usepackage{subfigure}
\usefonttheme{professionalfonts}
\def\mathfamilydefault{\rmdefault}
\usepackage{amsmath}
\usepackage{multirow}
\usepackage{booktabs}
\usepackage{bm}
\setbeamertemplate{section in toc}{\hspace*{1em}\inserttocsectionnumber.~\inserttocsection\par}
\setbeamertemplate{subsection in toc}{\hspace*{2em}\inserttocsectionnumber.\inserttocsubsectionnumber.~\inserttocsubsection\par}
\setbeamerfont{subsection in toc}{size=\small}
\AtBeginSection[]{%
	\begin{frame}%
		\frametitle{Outline}%
		\textbf{\tableofcontents[currentsection]} %
	\end{frame}%
}

\AtBeginSubsection[]{%
	\begin{frame}%
		\frametitle{Outline}%
		\textbf{\tableofcontents[currentsection, currentsubsection]} %
	\end{frame}%
}

\title{高等数学C}
%\subtitle{A Simple while elegant template}
\author[P.Yu]{余沛}
\mail{peiy\_gzgs@qq.com}
\institute[Guangzhou College of Technology and Business]{Guangzhou College of Technology and Business \\
  广州工商学院}
\date{\today}
\titlegraphic[width=4cm]{logo.png}{}




%%%%%%%%%%%%%%%%%%%%%%%%%%%%%%%%
% ----------- 标题页 ------------
%%%%%%%%%%%%%%%%%%%%%%%%%%%%%%%%



\begin{document}

\maketitle

%%% define code
\defverbatim[colored]\lstI{
	\begin{lstlisting}[language=C++,basicstyle=\ttfamily,keywordstyle=\color{red}]
	int main() {
	// Define variables at the beginning
	// of the block, as in C:
	CStash intStash, stringStash;
	int i;
	char* cp;
	ifstream in;
	string line;
	[...]
	\end{lstlisting}
}
%%%%%%%%%%%%%%%%%%%%%%%%%%%%%%%%
% ----------- FRAME ------------
%%%%%%%%%%%%%%%%%%%%%%%%%%%%%%%%

\section{导数与极值与最值}

\subsection{导数与极值}
\begin{frame}
	\frametitle{极值的必要条件}
	由极值定义和导数及单调性的联系, 我们首先给出不需要 $x_0$ 处可导性条件下可导函数的取极值的一个定理:
	\begin{theorem}{极大值与极小值}
		设函数$f(x)$在 $(x_0-\delta,x_0+\delta)$ 连续, 在 $(x_0-\delta,x_0)\cup(x_0,x_0+\delta)$上可导, 如果
		$$
			f'(x)\ge(\le)0, x\in(x_0-\delta,x_0),\,\,\text{且} \,\,f'(x)\le(\ge)0, x\in(x_0,x_0+\delta).
		$$
		那么函数在点$x_0$处取得极大(小)值.
	\end{theorem}
	试一试: 判断函数 $y=x^2,\,\,y=x^{\frac{1}{3}}$ 和 $y=\arcsin x$ 在 $x=0$ 处的极值情况.
\end{frame}

\subsection{导数为零: 驻点与驻点判断}
\begin{frame}
	\frametitle{导数为零: 驻点与驻点判断}
	由Fermat引理, 我们得知:
	\begin{block}{Fermat引理}
		设 $x_0$ 是 $f(x)$ 的一个极值点, 且 $f(x)$ 在 $x_0$ 处导数存在, 则
		$$
			f^{\prime}\left(x_0\right)=0 .
		$$
	\end{block}
	为了更加确定地讨论导数为零的情形, 引入驻点概念.
	\begin{block}{驻点}
		称 $x_0$ 是 $f(x)$ 的一个驻点, 如果
		$$
			f^{\prime}\left(x_0\right)=0.
		$$
	\end{block}
\end{frame}

\begin{frame}
	\frametitle{导数为零: 驻点与驻点判断}
	\begin{block}{}
		称 $x_0$ 是 $f(x)$ 的一个驻点, 如果
		$$
			f^{\prime}\left(x_0\right)=0.
		$$
	\end{block}
	\begin{block}{}
		设函数$f(x)$在 $(x_0-\delta,x_0+\delta)$ 连续, 在 $(x_0-\delta,x_0)\cup(x_0,x_0+\delta)$上可导, 如果
		$$
			f'(x)\ge(\le)0, x\in(x_0-\delta,x_0),\,\,\text{且} \,\,f'(x)\le(\ge)0, x\in(x_0,x_0+\delta).
		$$
		那么函数在点$x_0$处取得极大(小)值.
	\end{block}
\end{frame}


\begin{frame}
	\frametitle{求函数 $f(x)=x^2+1-\ln x$ 的极值}
	\everymath{\displaystyle}
	\begin{block}{}
		首先求函数的导数 $f'(x)$:
		\[ f'(x) = 2x - \frac{1}{x} = \frac{2x^2 - 1}{x} \]
	\end{block}
	\pause
	\begin{block}{求驻点}
		令 $f'(x) = 0$, 解得驻点 $x = \frac{\sqrt{2}}{2}$. 注意定义域!
	\end{block}
	\pause
	\begin{block}{判断极值}
		导数在驻点左侧为负, 在右侧为正, 所以 $f\left(\frac{\sqrt{2}}{2}\right) = \frac{3}{2} + \frac{\ln 2}{2}$ 是极小值.
	\end{block}
\end{frame}

\begin{frame}
	\frametitle{求函数 $f(x)=x^3-3 x^2-9 x+5$ 的极值}
	\everymath{\displaystyle}
	\begin{block}{}
		首先求函数的导数 $f'(x)$:
		\[ f'(x) = 3x^2 - 6x - 9 = 3(x+1)(x-3) \]
	\end{block}
	\pause
	\begin{block}{求驻点}
		令 $f'(x) = 0$, 解得驻点 $x_1 = -1$, $x_2 = 3$.
	\end{block}
	\pause
	\begin{block}{列表讨论}
		\begin{tabular}{|c|c|c|c|c|c|}
			\hline
			$x$     & $(-\infty,-1)$ & -1           & $(-1,3)$   & $\mathbf{3}$ & $(3,+\infty)$ \\
			\hline
			$f'(x)$ & +              & $\mathbf{0}$ & -          & $\mathbf{0}$ & +             \\
			\hline
			$f(x)$  & $\nearrow$     & 极大值          & $\searrow$ & 极小值          & $\nearrow$    \\
			\hline
		\end{tabular}
	\end{block}
	\pause
	\begin{block}{结果}
		极大值 $f(-1) = 10$, 极小值 $f(3) = -22$.
	\end{block}
\end{frame}

\begin{frame}
	\frametitle{求函数 $f(x)=\sin x+\cos x$ 在 $[0,2 \pi]$ 上的极值}
	\everymath{\displaystyle}
	\begin{block}{}
		首先求函数的导数 $f'(x)$:
		\[ f'(x) = \cos x - \sin x \]
	\end{block}
	\pause
	\begin{block}{求驻点}
		令 $f'(x) = 0$, 解得驻点 $x_1 = \frac{\pi}{4}$, $x_2 = \frac{5\pi}{4}$.
	\end{block}
	\pause
	\begin{block}{求二阶导数}
		\[ f''(x) = -\sin x - \cos x \]
	\end{block}
	\pause
	\begin{block}{判断极值}
		由于 $f''\left(\frac{\pi}{4}\right) = -\sqrt{2} < 0$, 所以 $f\left(\frac{\pi}{4}\right) = \sqrt{2}$ 是极大值;
		$f''\left(\frac{5\pi}{4}\right) = \sqrt{2} > 0$, 所以 $f\left(\frac{5\pi}{4}\right) = -\sqrt{2}$ 是极小值.
	\end{block}
\end{frame}

\subsection{最大值与最小值:极值的应用问题}
\begin{frame}
	\frametitle{求函数最大值和最小值的方法}
	\begin{block}{问题:使用导数的方法来求函数的最大值和最小值}
		计算函数 $f(x)=x-\frac32x^{\frac23}$ 在区间 $[-1,\frac{27}{8}]$ 上的最大值和最小值.
	\end{block}
	\pause
	\begin{block}{}
		\begin{enumerate}
			\item 求函数的导数 $f'(x)$.
			      \pause
			\item 解方程 $f'(x)=0$, 找出导数为零的点.
			      \pause
			\item 将这些点,导数不存在的点和端点, 代入函数 $f(x)$, 求出函数的值.
			      \pause
			\item 比较, 找出最大值和最小值.
		\end{enumerate}
	\end{block}
	\pause
	\begin{block}{}
		\begin{itemize}
			\item 函数的导数为 $f'(x)=1-x^{-\frac13}$.
			      \pause
			\item 解方程 $f'(x)=0$, 得到 $x=1$. 导数不存在点为 $x=0$
			      \pause
			\item 比较驻点 $x=1$, $f(1)=-\frac12$; 导数不存在点 $x=0$, $f(0)=0$; 左端点 $x=-1$, $f(-1)=-\frac{5}{2}$; 右端点 $x=\frac{27}{8}$, $f(\frac{27}{8})=0$.
			      \pause
			\item 在区间 $[-1,\frac{27}{8}]$, 函数$f(x)=x-\frac32x^{\frac23}$ 在$x=-1$, 取最小值 $f(-1)=-\frac{5}{2}$, 在 $x=\frac{27}{8}$ 和 $x=0$ 取最大值 $f(0)=f(\frac{27}{8})=0$.
		\end{itemize}
	\end{block}
\end{frame}



\begin{frame}
	\frametitle{求函数 $f(x)=\sqrt[3]{\left(2 x-x^2\right)^2}$ 在 $[-1,4]$ 上的最值}
	函数 $f(x)$ 的定义域为 $(-\infty,+\infty)$ . 由
	$$
		f^{\prime}(x)=\frac{4}{3}\left(2 x-x^2\right)^{-\frac{1}{3}}(1-x),
	$$

	知 $f(x)$ 的驻点为 $x=1$, 使得 $f^{\prime}(x)$ 不存在的点为 $x=0$ 和 $x=2$ .
	\begin{itemize}
		\item 当 $-\infty<x<0$ 时, $f^{\prime}(x)<0$;
		\item 当 $0<x<1$ 时, $f^{\prime}(x)>0$;
		\item 当 $1<x<2$ 时, $f^{\prime}(x)<0$;
		\item 当 $2<x<+\infty$ 时, $f^{\prime}(x)>0$,
	\end{itemize}
	由极值判定定理知 $f(0)=0$ 是极小值, $f(1)=1$ 是极大值, $f(2)=0$是极小值.\\ \vspace{0.2cm}
	极大值点: $x=1$, $f(1)=1$;极小值点: $x=0$和$x=2$, $f(0)=f(2)=0$; 端点: $f(-1)=\sqrt[3]{9}$, $f(4)=4$;\\
	\vspace{0.4cm}
	经过比较, 最大值点为 $x=4$, $f(4)=4$; 最小值点为 $x=0$和$x=2$, $f(0)=f(2)=0$.
\end{frame}

\section{凹凸性与拐点}
\begin{frame}
	\frametitle{函数的凹凸性和二阶导数}

	\begin{block}{凹凸性定义}
		设函数 $f(x)$ 在区间 $I$ 上有定义, 如果对于任意 $x_1, x_2 \in I$ 以及 $0 \leq t \leq 1$, 都有
		\[ f(tx_1 + (1-t)x_2) \leq tf(x_1) + (1-t)f(x_2) \]
		则称函数 $f(x)$ 在区间 $I$ 上是上凹函数; 如果不等号反向, 则称函数 $f(x)$ 在区间 $I$ 上是下凹函数.\\
		若区间在点 $x_0$ 左右两边凹凸性不同, 则称 $x_0$ 为拐点.
	\end{block}
\end{frame}

\begin{frame}
	\frametitle{求凹凸区间及拐点}
	\begin{block}{问题}
		求曲线 $y=3 x^4-4 x^3+1$ 的凹凸区间及拐点.
	\end{block}
	\pause
	\begin{block}{}
		首先求一阶导数和二阶导数:
		\[ y' = 12x^3 - 12x^2, \quad y'' = 36x^2 - 24x = 36x\left(x-\frac{2}{3}\right). \]
	\end{block}
	\pause
	\begin{block}{}
		令 $y'' = 0$, 解得拐点 $x_1 = 0$, $x_2 = \frac{2}{3}$.
	\end{block}
	\pause
	\begin{block}{}
		\begin{tabular}{|c|c|c|c|c|c|}
			\hline
			$x$      & $(-\infty, 0)$ & 0                  & $(0,\frac{2}{3})$ & $\frac{2}{3}$                                      & $(\frac{2}{3},+\infty)$ \\
			\hline
			$y''(x)$ & +              & 0                  & -                 & 0                                                  & +                       \\
			\hline
			$y(x)$   & 上凹             & \begin{tabular}{c}
				                            拐点 \\
				                            $(0,1)$
			                            \end{tabular} & 下凹                & \begin{tabular}{c}
				                                                                拐点 \\
				                                                                $(\frac{2}{3}, \frac{11}{27})$
			                                                                \end{tabular} & 上凹                                 \\
			\hline
		\end{tabular}
	\end{block}
\end{frame}

\section{渐近线和函数作图}
\begin{frame}
	\frametitle{渐近线分类}
	渐近线可以用于描述函数的趋势.
	\vspace{0.3cm}
	\begin{block}{渐近线分类}
		\begin{itemize}
			\item 铅垂渐近线: $\lim_{x\to a^+} f(x) = \infty$;
			\item 斜渐近线: $\lim_{x\to \pm\infty}\left[f(x)-(kx+c)\right]=0$;
			\item 特例: 水平渐近线: $\lim_{x\to \pm\infty}f(x)-c=0$.
		\end{itemize}
	\end{block}
	\vspace{1.2cm}
	\begin{columns}[T]
		\begin{column}{0.33\textwidth}
			\begin{equation*}
				y = \frac{1}{(x-1)^2},
			\end{equation*}
		\end{column}
		\begin{column}{0.33\textwidth}
			\begin{equation*}
				y = x+\frac{1}{x},
			\end{equation*}
		\end{column}
		\begin{column}{0.33\textwidth}
			\begin{equation*}
				y = e^{x}.
			\end{equation*}
		\end{column}
	\end{columns}
\end{frame}

\begin{frame}
	\frametitle{函数图形的作法}
	\begin{block}{函数图形的作法流程}
		\begin{enumerate}
			\item 确定定义域(必须);
			      \pause
			\item 确定对称性(降低工作量);
			      \pause
			\item 求导数确定函数的增减性, 极值(确定连线趋势);
			      \pause
			\item 求二阶导数确定函数的凹凸性, 拐点(确定凹凸性);
			      \pause
			\item 列表确定性质 (基本确定函数行为性质)
			      \pause
			\item 确定函数渐近线(确定渐近行为);
			      \pause
			\item 重要点坐标确定(标点).
		\end{enumerate}
	\end{block}
\end{frame}

\begin{frame}
	\frametitle{作出函数 $y=\frac{1}{\sqrt{2 \pi}} \mathrm{e}^{-\frac{x^2}{2}}$ 的图像}
	{\small
	\begin{block}{确定定义域}
		定义于整个实数域上.
	\end{block}
	\pause
	\begin{block}{确定对称性}
		偶函数.
	\end{block}
	\pause
	\begin{block}{求导数确定函数的增减性, 极值}
		$$
			f^{\prime}(x)=-\frac{1}{\sqrt{2 \pi}} x \mathrm{e}^{-\frac{x^2}{2}}
		$$
		可知, $x<0$ 时 $f'(x)>0$, $x>0$ 时 $f'(x)<0$, $x=0$ 时 $f'(x)=0$.
	\end{block}
	\pause
	\begin{block}{求二阶导数确定函数的凹凸性, 拐点}
		$$
			f^{\prime \prime}(x)=\frac{1}{\sqrt{2 \pi}} \mathrm{e}^{-\frac{x^2}{2}}\left(x^2-1\right)
		$$
		可知, $x\in(-\infty,1)\cup(1,\infty)$时, $f'(x)''>0$, $x\in(-1,1)$ 时, $f'(x)''<0$, $x=\pm1$ 时 $f''(x)=0$.
	\end{block}
	}
\end{frame}

\begin{frame}
	\frametitle{作出函数 $y=\frac{1}{\sqrt{2 \pi}} \mathrm{e}^{-\frac{x^2}{2}}$ 的图像}
	\begin{block}{列表确定性质}
		$$
			\begin{array}{|c|c|c|c|c|}
				\hline x                    & 0                      & (0,1)     & 1                                                 & (1,+\infty) \\
				\hline f^{\prime}(x)        & 0                      & -         & -                                                 & -           \\
				\hline f^{\prime \prime}(x) & -                      & -         & 0                                                 & +           \\
				\hline                      & \text { 极大值 }          &           & \text { 拐点 }                                      &             \\
				f(x)                        & \frac{1}{\sqrt{2 \pi}} & \text{下凹} & \left(1, \frac{1}{\sqrt{2 \pi \mathrm{e}}}\right) & \text{上凹}   \\
				\hline
			\end{array}
		$$
	\end{block}
	\pause
	\begin{center}
		\begin{tikzpicture}
			\draw[->] (-3,0) -- (3,0) node[right] {$x$};
			\draw[->] (0,-0.5) -- (0,1.5) node[above] {$y$};

			\draw[domain=-3:3,smooth,variable=\x,blue] plot ({\x},{1/sqrt(2*pi) * exp(-\x*\x/2)});

			\draw[dashed] (-1,0) node[below] {$-1$} -- (-1,{1/sqrt(2*pi) * exp(-1*1/2)});
			\draw[dashed] (1,0) node[below] {$1$} -- (1,{1/sqrt(2*pi) * exp(-1*1/2)});

			\draw (-1,0.1) -- (-1,-0.1) node[below] {};
			\draw (1,0.1) -- (1,-0.1) node[below] {};

			\draw (0.1,{1/sqrt(2*pi)}) -- (-0.1,{1/sqrt(2*pi)}) node[left] {$\frac{1}{\sqrt{2 \pi}}$};
		\end{tikzpicture}
	\end{center}
\end{frame}


\begin{frame}
	\frametitle{作出函数 $y=\frac{(x-1)^2}{3(x+1)}$ 的图像}
	\begin{block}{确定定义域}
		定义域为 $(-\infty,-1) \cup(-1,+\infty)$.
	\end{block}
	\pause
	\begin{block}{确定对称性}
		无对称.
	\end{block}
	\pause
	\begin{block}{求导数确定函数的增减性, 极值}
		$$
			f^{\prime}(x)=\frac{(2 x-2)(x+1)-(x-1)^2}{3(x+1)^2}=\frac{(x+3)(x-1)}{3(x+1)^2}
		$$
		可知, $f^{\prime}(x)$ 有零点 $x=1$ 和 $x=-3$ . 由于 $f^{\prime}(x)$ 在 $x=-3$ 的右侧和左侧的符号分别为负和正, 而在 $x=1$ 的右侧和左侧的符号分别为正和负, 所以 $x=-3$ 是 $f(x)$ 的极大值点, $x=1$ 是 $f(x)$ 的极小值点. 
	\end{block}

\end{frame}

\begin{frame}
	\frametitle{作出函数 $y=\frac{1}{\sqrt{2 \pi}} \mathrm{e}^{-\frac{x^2}{2}}$ 的图像}
	\begin{block}{求二阶导数确定函数的凹凸性, 拐点}
		$$
			f^{\prime \prime}(x)=\frac{8}{3(x+1)^3}.
		$$
		因为 $f^{\prime \prime}(x)$ 在定义域中没有零点, 所以曲线上没有拐点. 
	\end{block}
	\pause
	{\small
	\begin{block}{列表确定性质}
		$$
			\begin{array}{|c|c|c|c|}
				\hline x                    & (-\infty,-3) & -3               & (-3,-1) \\
				\hline f^{\prime}(x)        & +            & 0                & -       \\
				\hline f^{\prime \prime}(x) & -            & -                & -       \\
				\hline f(x)                 & \text{下凹}    & \begin{array}{c}
					                                             \text { 极大值 } \\
					                                             -\frac{8}{3}
				                                             \end{array} & \text{下凹}    \\
				\hline
			\end{array}
		$$
		$$
			\begin{array}{|c|c|c|c|c|}
				\hline x                    & -1            & (-1,1)    & 1                & (1,+\infty) \\
				\hline f^{\prime}(x)        & \text { 无定义 } & -         & 0                & +           \\
				\hline f^{\prime \prime}(x) & \text { 无定义 } & +         & +                & +           \\
				\hline f(x)                 & \text { 无定义 } & \text{上凹} & \begin{array}{c}
					                                                          \text { 极小值 } \\
					                                                          0
				                                                          \end{array} & \text{上凹}        \\
				\hline
			\end{array}
		$$
	\end{block}
	}
\end{frame}

\begin{frame}
	\frametitle{作出函数 $y=\frac{(x-1)^2}{3(x+1)}$ 的图像}
	\begin{block}{渐近线}
		$$
			\begin{aligned}
				a & =\lim _{x \rightarrow \infty} \frac{y}{x}=\lim _{x \rightarrow \infty} \frac{(x-1)^2}{3 x(x+1)}=\frac{1}{3},                                       \\
				b & =\lim _{x \rightarrow \infty}\left[\frac{(x-1)^2}{3(x+1)}-a x\right]=\lim _{x \rightarrow \infty}\left[\frac{(x-1)^2}{3(x+1)}-\frac{1}{3} x\right] \\
				  & =\frac{1}{3} \lim _{x \rightarrow \infty} \frac{-3 x+1}{x+1}=-1,
			\end{aligned}
		$$
		因此 $y=\frac{x}{3}-1$ 为曲线 $y=\frac{(x-1)^2}{3(x+1)}$ 的斜浙近线. 又因为
		$$
			\lim _{2-1} \frac{(x-1)^2}{3(x+1)}=+\infty, \quad \lim _{x \rightarrow-1} \frac{(x-1)^2}{3(x+1)}=-\infty
		$$

		可知 $x=-1$ 是曲线 $y=\frac{(x-1)^2}{3(x+1)}$ 的垂直渐近线. 
	\end{block}
\end{frame}

\begin{frame}
	\frametitle{作出函数 $y=\frac{(x-1)^2}{3(x+1)}$ 的图像}
	\begin{figure}
		\centering
		\includegraphics[width=0.8\textwidth]{frac{(x-1)^2}{3(x+1)}.png}
		\caption{函数 $y=\frac{(x-1)^2}{3(x+1)}$ 的图像}
	  \end{figure}
\end{frame}

%\begin{frame}
%	\frametitle{扩展阅读:分形几何与处处连续处处不可导的曲线}
%\end{frame}

% Thank you page
\beamertemplateshadingbackground{structure.fg!90}{structure.fg}
\begin{frame}[plain]
	\vfill
	\centering
	{
	\centering \Huge \color{white} Thank you for your attention!\\[10pt]Questions?\\ Homework: \\\vspace{0.2cm}Pages 169: 19、20\\\vspace{0.2cm}绘制图像:
	{
	$$
		e^{\frac{1}{|x-1|^2}},\,\,\ln{1+sin{e^{\frac{1}{|x|}}}},\,\,\sqrt{x^2+1}.
	$$
	}
	}
	\vfill
\end{frame}


\end{document}


