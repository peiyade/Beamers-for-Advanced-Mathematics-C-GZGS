% !TEX encoding = UTF-8 Unicode.

% Based on https://github.com/Miracle0565/BUCT-Beamer-Theme

\documentclass[
10pt,
aspectratio=43,
]{beamer}
\setbeamercovered{transparent=10}
\usetheme[
%  showheader,
%  red,
  purple,
%  gray,
%  graytitle,
  colorblocks,
%  noframetitlerule,
]{Verona}

\usepackage[T1]{fontenc}
\usepackage{tikz}
\usepackage[utf8]{inputenc}
\usepackage{lipsum}
\usepackage{pgfplots}
%%%%%%%%%%%%%%%%%%%%%%%%%%%%%%%
% Mac上使用如下命令声明隶书字体, windows也有相关方式, 大家可自行修改
\providecommand{\lishu}{\CJKfamily{zhli}}
%%%%%%%%%%%%%%%%%%%%%%%%%%%%%%%
\usepackage{tikz}
\usetikzlibrary{fadings}
%
%\setbeamertemplate{sections/subsections in toc}[ball]
\usepackage{xeCJK}
\usepackage{listings}
\usepackage{caption}
\usepackage{subfigure}
\usefonttheme{professionalfonts}
\def\mathfamilydefault{\rmdefault}
\usepackage{amsmath}
\usepackage{multirow}
\usepackage{booktabs}
\usepackage{bm}
\usepackage{mathtools}
\usepackage[T1]{fontenc}
\setbeamertemplate{section in toc}{\hspace*{1em}\inserttocsectionnumber.~\inserttocsection\par}
\setbeamertemplate{subsection in toc}{\hspace*{2em}\inserttocsectionnumber.\inserttocsubsectionnumber.~\inserttocsubsection\par}
\setbeamerfont{subsection in toc}{size=\small}
\AtBeginSection[]{%
	\begin{frame}%
		\frametitle{Outline}%
		\textbf{\tableofcontents[currentsection]} %
	\end{frame}%
}

\AtBeginSubsection[]{%
	\begin{frame}%
		\frametitle{Outline}%
		\textbf{\tableofcontents[currentsection, currentsubsection]} %
	\end{frame}%
}

\pgfplotsset{
    integral segments/.code={\pgfmathsetmacro\integralsegments{#1}},
    integral segments=3,
    integral/.style args={#1:#2}{
        ybar interval,
        domain=#1+((#2-#1)/\integralsegments)/2:#2+((#2-#1)/\integralsegments)/2,
        samples=\integralsegments+1,
        x filter/.code=\pgfmathparse{\pgfmathresult-((#2-#1)/\integralsegments)/2}
    }
}


\title{高等数学C}
%\subtitle{A Simple while elegant template}
\author[P.Yu]{余沛}
\mail{peiy\_gzgs@qq.com}
\institute[Guangzhou College of Technology and Business]{Guangzhou College of Technology and Business \\
  广州工商学院}
\date{\today}
\titlegraphic[width=4cm]{logo.png}{}




%%%%%%%%%%%%%%%%%%%%%%%%%%%%%%%%
% ----------- 标题页 ------------
%%%%%%%%%%%%%%%%%%%%%%%%%%%%%%%%



\begin{document}

\maketitle

%%% define code
\defverbatim[colored]\lstI{
	\begin{lstlisting}[language=C++,basicstyle=\ttfamily,keywordstyle=\color{red}]
	int main() {
	// Define variables at the beginning
	// of the block, as in C:
	CStash intStash, stringStash;
	int i;
	char* cp;
	ifstream in;
	string line;
	[...]
	\end{lstlisting}
}
%%%%%%%%%%%%%%%%%%%%%%%%%%%%%%%%
% ----------- FRAME ------------
%%%%%%%%%%%%%%%%%%%%%%%%%%%%%%%%
\section{微积分基本定理}
\subsection{变上限积分与原函数存在定理}

\begin{frame}
	\frametitle{变上限积分}
	\begin{block}{定义: 变上限积分}
		设函数 $f(x)$ 在 $[a, b]$ 上连续, 变上限积分
	$$
	p(x)=\int_a^x f(t) \mathrm{~d} t, \quad x \in[a, b]
	$$
	\end{block}
	\begin{block}{定理}
		设函数 $f(x)$ 在 $[a, b]$ 上连续, 则有
$$
p^{\prime}(x)=\left(\int_a^x f(t) \mathrm{~d} t\right)^{\prime}=f(x)
$$
	\end{block}


\end{frame}

\begin{frame}
	\frametitle{变上限积分}

	\begin{proof}
		对于任意 $x$ 和改变量 $\Delta x$, 有
		$$ 
			p(x+\Delta x) - p(x) = \int_a^{x+\Delta x}f(t)\mathrm{~d}t -\int_a^{x}f(t)\mathrm{~d}t =\int_{x}^{x+\Delta x}f(t)\mathrm{~d}t
		$$
		由积分中值定理, 存在 $\xi\in(x,x+\Delta x)$ 使得
		$$
		p(x+\Delta x) - p(x)=f(\xi)(\Delta x)
		$$
		于是 $\Delta x\to0 \Rightarrow \xi\to x$, 由 $f(x)$ 在$[a,b]$上的连续性, 有
		$$
			p'(x)=\lim_{\Delta x\to0}\frac{p(x+\Delta x) - p(x)}{\Delta x} = \lim_{\Delta x\to0}f(\xi) = f(x).
		$$
	\end{proof}
	这个证明也给出了不定积分里的原函数存在性.
\end{frame}

\begin{frame}
	\frametitle{训练题}
	\everymath{\displaystyle}
	\begin{block}{求导数}
		\begin{columns}
			\begin{column}{0.5\textwidth}
				\begin{enumerate}
					\item $\int_0^x\sqrt{1+t}\mathrm{~d}t$;
					\item $\int_{x}^3se^{-3s}\mathrm{~d}s$;
				\end{enumerate}
			\end{column}
			\begin{column}{0.5\textwidth}
				
			\end{column}
		\end{columns}
	\end{block}
\end{frame}
\subsection{Newton-Leibniz公式}
\section{积分法}
\subsection{换元积分}
\subsection{分部积分}
\section{定积分的应用}

% Thank you page
\beamertemplateshadingbackground{structure.fg!90}{structure.fg}
\begin{frame}[plain]
	\vfill
	\centering
	{
		\centering \Huge \color{white} Thank you for your attention!\\[10pt]Questions?
	}
	\vfill
\end{frame}


\end{document}


