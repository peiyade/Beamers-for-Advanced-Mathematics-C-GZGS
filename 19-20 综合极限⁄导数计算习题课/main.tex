% !TEX encoding = UTF-8 Unicode.

% Based on https://github.com/Miracle0565/BUCT-Beamer-Theme

\documentclass[
10pt,
aspectratio=43,
]{beamer}
\setbeamercovered{transparent=10}
\usetheme[
%  showheader,
%  red,
  purple,
%  gray,
%  graytitle,
  colorblocks,
%  noframetitlerule,
]{Verona}

\usepackage[T1]{fontenc}
\usepackage{tikz}
\usepackage[utf8]{inputenc}
\usepackage{lipsum}
%%%%%%%%%%%%%%%%%%%%%%%%%%%%%%%
% Mac上使用如下命令声明隶书字体,windows也有相关方式,大家可自行修改
\providecommand{\lishu}{\CJKfamily{zhli}}
%%%%%%%%%%%%%%%%%%%%%%%%%%%%%%%
\usepackage{tikz}
\usetikzlibrary{fadings}
%
%\setbeamertemplate{sections/subsections in toc}[ball]
\usepackage{xeCJK}
\usepackage{listings}
\usepackage{caption}
\usepackage{subfigure}
\usefonttheme{professionalfonts}
\def\mathfamilydefault{\rmdefault}
\usepackage{amsmath}
\usepackage{multirow}
\usepackage{booktabs}
\usepackage{bm}
\setbeamertemplate{section in toc}{\hspace*{1em}\inserttocsectionnumber.~\inserttocsection\par}
\setbeamertemplate{subsection in toc}{\hspace*{2em}\inserttocsectionnumber.\inserttocsubsectionnumber.~\inserttocsubsection\par}
\setbeamerfont{subsection in toc}{size=\small}
\AtBeginSection[]{%
	\begin{frame}%
		\frametitle{Outline}%
		\textbf{\tableofcontents[currentsection]} %
	\end{frame}%
}

\AtBeginSubsection[]{%
	\begin{frame}%
		\frametitle{Outline}%
		\textbf{\tableofcontents[currentsection, currentsubsection]} %
	\end{frame}%
}

\title{高等数学C}
%\subtitle{A Simple while elegant template}
\author[P.Yu]{余沛}
\mail{peiy\_gzgs@qq.com}
\institute[Guangzhou College of Technology and Business]{Guangzhou College of Technology and Business \\
  广州工商学院}
\date{\today}
\titlegraphic[width=4cm]{logo.png}{}




%%%%%%%%%%%%%%%%%%%%%%%%%%%%%%%%
% ----------- 标题页 ------------
%%%%%%%%%%%%%%%%%%%%%%%%%%%%%%%%



\begin{document}

\maketitle

%%% define code
\defverbatim[colored]\lstI{
	\begin{lstlisting}[language=C++,basicstyle=\ttfamily,keywordstyle=\color{red}]
	int main() {
	// Define variables at the beginning
	// of the block, as in C:
	CStash intStash, stringStash;
	int i;
	char* cp;
	ifstream in;
	string line;
	[...]
	\end{lstlisting}
}
%%%%%%%%%%%%%%%%%%%%%%%%%%%%%%%%
% ----------- FRAME ------------
%%%%%%%%%%%%%%%%%%%%%%%%%%%%%%%%
\section{求导训练}

\subsection{基本求导训练}\begin{frame}
	\frametitle{求导示例}
	\everymath{\displaystyle}
	\begin{block}{}
		\begin{columns}[onlytextwidth]
			\column{0.5\textwidth}
			\begin{enumerate}
				\item $y=\frac{a x+b}{c x+d}$;
				\item $y=\frac{\sin ^2 x}{\sin x^2}$;
				\item $y=(1+x) \sqrt{2+x^2} \sqrt[3]{3+x^3}$;
			\end{enumerate}
			\column{0.7\textwidth}
			\begin{enumerate}
				\setcounter{enumi}{3}
				\item $y=\frac{1}{\sqrt{1+x^2}\left(x+\sqrt{1+x^2}\right)}$;
				\item $y=\frac{1+x-x^2}{1-x+x^2}$;
				\item $y=\mathrm{e}^{-x^2}$.
			\end{enumerate}
		\end{columns}
	\end{block}

	\begin{exampleblock}{}
		\begin{columns}[onlytextwidth]
			\column{0.6\textwidth}
			\begin{enumerate}
				\pause
				\item $\displaystyle \frac{a d-b c}{(c x+d)^2}$;
				\item $\frac{2 \sin x\left(\cos x \sin x^2-x \sin x \cos x^2\right)}{\sin ^2 x^2}$,\\$(x \neq \sqrt{k\pi}, k\in\mathbb{Z}^+)$;
				\item $\frac{6+3 x+8 x^2+4 x^3+2 x^4+3 x^5}{\sqrt{2+x^2} \sqrt[3]{\left(3+x^3\right)^2}}$,\\$(x \neq \sqrt[3]{-3})$;
			\end{enumerate}
			\column{0.6\textwidth}
			\begin{enumerate}
				\setcounter{enumi}{3}
				\pause
				\item $\displaystyle \frac{-1}{(1+x^2)^{3/2}}$;
				\item $\displaystyle \frac{2(1-2x)}{(x^2-x+1)^2}$;
				\item $\displaystyle -2 x e^{-x^2}$;
			\end{enumerate}
		\end{columns}
	\end{exampleblock}
\end{frame}

\subsection{抽象求导问题}
\begin{frame}
	\frametitle{求导示例}
	\everymath{\displaystyle}
	\begin{block}{}
		\begin{columns}[onlytextwidth]
			\column{0.5\textwidth}
			\begin{enumerate}
				\item $y=\sqrt{\varphi^2(x)+\psi^2(x)}$;
				\item $y=\arctan \frac{\varphi(x)}{\psi(x)}$;
			\end{enumerate}
			\column{0.5\textwidth}
			\begin{enumerate}
				\setcounter{enumi}{2}
				\item $y=\sqrt[\varphi(x)]{\psi(x)}$;
				\item $y=\log _{\varphi(x)} \psi(x)$.
			\end{enumerate}
		\end{columns}
	\end{block}

	\begin{exampleblock}{}
		\begin{columns}[onlytextwidth]
			\column{0.2\textwidth}
			\begin{enumerate}
				\pause
				\item $\displaystyle \frac{\varphi(x) \varphi'(x)+\psi(x) \psi'(x)}{\sqrt{\varphi^2(x)+\psi^2(x)}}$,\\\vspace{0.1cm}$\left(\varphi^2(x)+\psi^2(x) \neq 0\right)$;
				      \vspace{0.4cm}
					  \pause
				\item $\displaystyle \frac{\psi(x) \varphi'(x)-\varphi(x) \psi'(x)}{\varphi^2(x)+\psi^2(x)}$,\\\vspace{0.1cm}$\left(\varphi^2(x)+\psi^2(x) \neq 0\right)$;
			\end{enumerate}
			\column{0.6\textwidth}

			\begin{enumerate}
				\setcounter{enumi}{2}
				{
				\small
				\pause
				\item $\sqrt[\varphi(x)]{\psi(x)}\left\{\frac{1}{\varphi(x)} \frac{\psi^{\prime}(x)}{\psi(x)}-\frac{\varphi^{\prime}(x)}{\varphi^2(x)} \log \psi(x)\right\}$,\\\vspace{0.1cm}$(\varphi(x) \neq 0, \psi(x)>0)$;
				      }
				      \vspace{0.4cm}
					  \pause
				\item $\displaystyle \frac{\psi'(x)}{\psi(x) \log \varphi(x)}-\frac{\varphi'(x) \log \psi(x)}{\varphi(x) \log^2 \varphi(x)}$,\\\vspace{0.1cm}$(\varphi(x)>0, \psi(x)>0)$.
			\end{enumerate}

		\end{columns}
	\end{exampleblock}

\end{frame}

\begin{frame}
	\frametitle{求导示例}
	\everymath{\displaystyle}
	\begin{block}{设$f(u)$为可微函数}
		\begin{columns}[onlytextwidth]
			\column{0.5\textwidth}
			\begin{enumerate}
				\item $y=f\left(x^2\right)$;
				\item $y=f\left(\sin ^2 x\right)+f\left(\cos ^2 x\right)$;
			\end{enumerate}
			\column{0.5\textwidth}
			\begin{enumerate}
				\setcounter{enumi}{2}
				\item $y=f\left(\mathrm{e}^x\right) \cdot \mathrm{e}^{f(x)}$;
				\item $y=f\{f[f(x)]\}$.
			\end{enumerate}
		\end{columns}
	\end{block}

	\begin{exampleblock}{}
		\begin{columns}[onlytextwidth]
			\column{0.5\textwidth}
			\begin{enumerate}
				\pause
				\item $\displaystyle 2x f'\left(x^2\right)$;
				\pause
				\item $\displaystyle \sin 2x \left(f'\left(\sin ^2 x\right)-f'\left(\cos ^2 x\right)\right)$;
			\end{enumerate}
			\column{0.5\textwidth}
			\begin{enumerate}
				\setcounter{enumi}{2}
				\pause
				\item $\mathrm{e}^{f(x)}\left[\mathrm{e}^x f^{\prime}\left(\mathrm{e}^x\right)+f^{\prime}(x) f\left(\mathrm{e}^x\right)\right]$;
				\pause
				\item $\displaystyle f'\{f[f(x)]\} \cdot f'[f(x)] \cdot f'(x)$.
			\end{enumerate}
		\end{columns}
	\end{exampleblock}

\end{frame}

\subsection{高阶导数计算}

\subsubsection{二阶导数}
\begin{frame}
	\frametitle{求函数的二阶导数}
	\everymath{\displaystyle}
	\begin{block}{设$(f(x)>0)$ 二次可微}
		\begin{columns}[onlytextwidth]
			\column{0.5\textwidth}
			\begin{enumerate}
				\item $y=x \sqrt{1+x^2}$;
				\item $y=\mathrm{e}^{-x^2}$;
				\item $y=x \log x$;
			\end{enumerate}
			\column{0.5\textwidth}
			\begin{enumerate}
				\setcounter{enumi}{3}
				\item $y=\tan x$;
				\item $y=\log f(x)$;
				\item $y=x[\sin (\log x)+\cos (\log x)]$.
			\end{enumerate}
		\end{columns}
	\end{block}

	\begin{exampleblock}{答案}
		\begin{columns}[onlytextwidth]
			\column{0.4\textwidth}
			\begin{enumerate}
				\pause
				\item $\frac{d^2y}{dx^2} = \frac{x(3+2x^2)}{(1+x^2)^{3/2}}$;
				\item $\frac{d^2y}{dx^2} = 2(2x^2-1)e^{-x^2}$;
				\item $\frac{d^2y}{dx^2} = \frac{1}{x},\,\, x>0$;
			\end{enumerate}
			\column{0.6\textwidth}
			\begin{enumerate}
				\pause
				\setcounter{enumi}{3}
				\item $\frac{d^2y}{dx^2} = 2\sin x\sec^3 x$,\\\vspace{.1cm}$(x \neq \frac{2 k+1}{2} \pi,\,k\in\mathbb{Z})$;
				\item $\frac{d^2y}{dx^2} = \frac{f''(x)}{f(x)} - \left(\frac{f'(x)}{f(x)}\right)^2, (f(x)>0)$;
				\item $-\frac{2}{x} \sin (\log x)(x>0)$.
			\end{enumerate}
		\end{columns}
	\end{exampleblock}

\end{frame}
\subsubsection{高阶导数}
\begin{frame}
	\frametitle{求函数的二阶导数和三阶导数}
	\everymath{\displaystyle}
	\begin{block}{设$f(x)$ 三次可微}
		\begin{enumerate}
			\item $y=f\left(x^2\right)$;
			\item $y=f\left(\frac{1}{x}\right)$.
		\end{enumerate}
	\end{block}

	\begin{exampleblock}{答案}
		\begin{enumerate}
			\pause
			\item $\frac{d^2y}{dx^2} = 2x^2f''\left(x^2\right) + 4f'\left(x^2\right)$,
			\pause
			      $\frac{d^3y}{dx^3}=8 x^3 f^{\prime \prime \prime}\left(x^2\right)+12 x f^{\prime \prime}\left(x^2\right)$;
				  \pause
			\item $\frac{d^2y}{dx^2} = \frac{2}{x^3}f''\left(\frac{1}{x}\right) - \frac{1}{x^4}f'\left(\frac{1}{x}\right)$,\\\pause
			      $\frac{d^3y}{dx^3}=-\frac{1}{x^6} f^{\prime \prime \prime}\left(\frac{1}{x}\right)-\frac{6}{x^5} f^{\prime \prime}\left(\frac{1}{x}\right)-\frac{6}{x^4} f^{\prime}\left(\frac{1}{x}\right)$.
		\end{enumerate}
	\end{exampleblock}
\end{frame}

\section{L'Hospital 法则应用求极限}
\subsection{求极限:$\displaystyle\frac{0}{0}$型}
\begin{frame}
	\frametitle{求极限:$\displaystyle\frac{0}{0}$型}
	\everymath{\displaystyle}
	{
		\small
		\begin{block}{}
			\begin{columns}[onlytextwidth]
				\column{0.5\textwidth}
				\begin{enumerate}
					\item $\lim _{x \rightarrow 0} \frac{\sin a x}{\sin b x}$;
					\item $\lim _{x \rightarrow 0} \frac{\cosh x-\cos x}{x^2}$;
					\item $\lim _{x \rightarrow 0} \frac{x\left(\mathrm{e}^x+1\right)-2\left(\mathrm{e}^x-1\right)}{x^3}$;
					\item $\lim _{x \rightarrow 0} \frac{\arcsin 2 x-2 \arcsin x}{x^3}$;
				\end{enumerate}
				\column{0.5\textwidth}
				\begin{enumerate}
					\setcounter{enumi}{4}
					\item $\lim _{x \rightarrow 0} \frac{1-\cos x^2}{x^2 \sin x^2}$;
					\item $\lim _{x \rightarrow 0} \frac{a^x-a^{\sin x}}{x^3}(a>0)$;
					\item $\lim _{x \rightarrow 0} \frac{3 \tan 4 x-12 \tan x}{3 \sin 4 x-12 \sin x}$;
					\item $\lim _{x \rightarrow 0} \frac{\log (\cos a x)}{\log (\cos b x)}$.
				\end{enumerate}
			\end{columns}
		\end{block}

		\begin{exampleblock}{}
			\begin{columns}[onlytextwidth]
				\column{0.5\textwidth}
				\begin{enumerate}
					\pause
					\item $\frac{a}{b}$;
					\item $ 1$;
					\pause
					\item $\frac{1}{6}$;
					\item $1$;
				\end{enumerate}
				\column{0.5\textwidth}
				\begin{enumerate}
					\setcounter{enumi}{4}
					\pause
					\item $ \frac{1}{2}$;
					\item $\frac16\log a$;
					\pause
					\item $-2$;
					\item $\left(\frac{ a}{b}\right)^2$.
				\end{enumerate}
			\end{columns}
		\end{exampleblock}
	}
\end{frame}

\subsection{求极限:$\displaystyle\frac{\infty}{\infty}$型}
\begin{frame}
	\frametitle{求极限:$\displaystyle\frac{\infty}{\infty}$型}
	\everymath{\displaystyle}
	\begin{block}{题目}
		\begin{columns}[onlytextwidth]
			\column{0.4\textwidth}
			\begin{enumerate}
				\item $\lim _{x \rightarrow 0} \frac{\ln (\sin a x)}{\ln (\sin b x)}$;
				\item $\lim _{x \rightarrow+\infty} \frac{\ln x}{x^{\varepsilon}}(\varepsilon>0)$;
			\end{enumerate}
			\column{0.5\textwidth}
			\begin{enumerate}
				\setcounter{enumi}{2}
				\item $\lim _{x \rightarrow+\infty} \frac{x^n}{\mathrm{e}^{a x}}(a>0, n>0)$;
				\item $\lim _{x \rightarrow 0+} \frac{\ln (\tan 7 x)}{\ln (\tan 2 x)}$.
			\end{enumerate}
		\end{columns}
	\end{block}

	\begin{exampleblock}{答案}
		\begin{columns}[onlytextwidth]
			\column{0.5\textwidth}
			\begin{enumerate}
				\pause
				\item $1$;
				\item $0$;
			\end{enumerate}
			\column{0.5\textwidth}
			\begin{enumerate}
				\setcounter{enumi}{2}
				\pause
				\item $0$;
				\item $1$.
			\end{enumerate}
		\end{columns}
	\end{exampleblock}
\end{frame}

\subsection{求极限:$\displaystyle\infty-\infty$型}
\begin{frame} \frametitle{欲求的极限和变换方法}
	\everymath{\displaystyle}
	\begin{block}{$0\cdot\infty$型} 
		若 $\lim _{x \rightarrow c} f(x)=0, \lim _{x \rightarrow c} g(x)=\infty$, \\
		可计算
		$$\lim _{x \rightarrow c} f(x) g(x)=\lim _{x \rightarrow c} \frac{f(x)}{1 / g(x)}$$ 或 $$\lim _{x \rightarrow c} \frac{g(x)}{1 / f(x)}$$
	\end{block}
\pause
	\begin{block}{$\infty-\infty$ 型}
		若$\lim _{x \rightarrow c} f(x)=\infty, \lim _{x \rightarrow c} g(x)=\infty$, \\
		可计算$$\lim _{x \rightarrow c}(f(x)-g(x))=\lim _{x \rightarrow c} \frac{1 / g(x)-1 / f(x)}{1 /(f(x) g(x))}.$$
	\end{block}

\end{frame}

\begin{frame} \frametitle{欲求的极限和变换方法}
	\everymath{\displaystyle}
	\begin{block}{$(0^+)^0$ 型与$\infty^0$ 型} 
		若$\lim _{x \rightarrow c} f(x)=0^{+}, \lim _{x \rightarrow c} g(x)=0$ 或 $\lim _{x \rightarrow c} f(x)=\infty, \lim _{x \rightarrow c} g(x)=0$, \\
		可计算$$\lim _{x \rightarrow c} f(x)^{g(x)}=\exp \lim _{x \rightarrow c} \frac{g(x)}{1 / \ln f(x)}.$$
	\end{block}
\pause
	\begin{block}{$1^\infty$ 型} 
		若$\lim _{x \rightarrow c} f(x)=1, \lim _{x \rightarrow c} g(x)=\infty$, 可计算 \\
		$$\lim _{x \rightarrow c} f(x)^{g(x)}=\exp \lim _{x \rightarrow c} \frac{\ln f(x)}{1 / g(x)}.$$
	\end{block}
\end{frame}

\begin{frame}
	\frametitle{求极限:$\displaystyle\infty-\infty$型}
	\everymath{\displaystyle}
	\begin{block}{题目}
		\begin{columns}[onlytextwidth]
			\column{0.4\textwidth}
			\begin{enumerate}
				\item $\lim _{x \rightarrow 0}\left(\frac{1}{x}-\frac{1}{\mathrm{e}^x-1}\right)$;
				\item $\lim _{x \rightarrow 1}\left(\frac{1}{\ln x}-\frac{1}{x-1}\right)$;
			\end{enumerate}
			\column{0.6\textwidth}
			\begin{enumerate}
				\setcounter{enumi}{2}
				\item $\lim _{x \rightarrow 0}\left(\cot x-\frac{1}{x}\right)$;
				\item $\lim _{x \rightarrow 0}\left[\frac{1}{\ln \left(x+\sqrt{1+x^2}\right)}-\frac{1}{\ln (1+x)}\right]$.
			\end{enumerate}
		\end{columns}
	\end{block}

	\begin{exampleblock}{答案}
		\begin{columns}[onlytextwidth]
			\column{0.4\textwidth}
			\begin{enumerate}
				\pause
				\item $\frac12$;
				\item $\frac12$;
			\end{enumerate}
			\column{0.6\textwidth}
			\begin{enumerate}
				\setcounter{enumi}{2}
				\pause
				\item $0$;\vspace{0.2cm}
				\item $-\frac12$.
			\end{enumerate}
		\end{columns}
	\end{exampleblock}
\end{frame}

\subsection{求极限:$\displaystyle1^\infty$型}
\begin{frame}
	\frametitle{求极限:$\displaystyle1^\infty$型}
	\everymath{\displaystyle}
	\begin{block}{题目}
		\begin{columns}[onlytextwidth]
			\column{0.5\textwidth}
			\begin{enumerate}
				\item $\lim _{x \rightarrow 1} x^{\frac{1}{1-x}}$;
				\item $\lim _{x \rightarrow 1}(2-x)^{\tan \frac{\pi x}{2}}$;
			\end{enumerate}
			\column{0.5\textwidth}
			\begin{enumerate}
				\setcounter{enumi}{2}
				\item $\lim _{x \rightarrow \frac{\pi}{4}}(\tan x)^{\tan 2 x}$;
				\item $\lim _{x \rightarrow+\infty}\left(\frac{2}{\pi} \arctan x\right)^x$.
			\end{enumerate}
		\end{columns}
	\end{block}

	\begin{exampleblock}{答案}
		\begin{columns}[onlytextwidth]
			\column{0.5\textwidth}
			\begin{enumerate}
				\pause
				\item $e^{-1}$;
				\item $e^{\frac{2}{\pi}}$;
			\end{enumerate}
			\column{0.5\textwidth}
			\begin{enumerate}
				\setcounter{enumi}{2}
				\pause
				\item $e^{-1}$;
				\item $e^{\frac{2}{\pi}}$.
			\end{enumerate}
		\end{columns}
	\end{exampleblock}

\end{frame}

\subsection{求极限:$\displaystyle0^0$型}
\begin{frame}
	\frametitle{求极限:$\displaystyle0^0$型}
	\everymath{\displaystyle}
	\begin{block}{求极限}
		\begin{columns}[t]
			\begin{column}{0.5\textwidth}
				\begin{enumerate}
					\item $\lim _{x \rightarrow+0} x^x$;
					\item $\lim _{x \rightarrow 0} x^{x^x-1}$;
				\end{enumerate}
			\end{column}
			\begin{column}{0.5\textwidth}
				\begin{enumerate}
					\setcounter{enumi}{2}
					\item $\lim _{x \rightarrow 0}\left(x^{x^x}-1\right)$;
					\item $\lim _{x \rightarrow+0} x^{\frac{k}{1+\ln x}}$.
				\end{enumerate}
			\end{column}
		\end{columns}
	\end{block}

	\begin{exampleblock}{答案}
		\begin{columns}[t]
			\begin{column}{0.5\textwidth}
				\begin{enumerate}
					\pause
					\item $1$;
					\item $1$;
				\end{enumerate}
			\end{column}
			\begin{column}{0.5\textwidth}
				\begin{enumerate}
					\setcounter{enumi}{2}
					\pause
					\item $-1$;
					\item $e^{k}$.
				\end{enumerate}
			\end{column}
		\end{columns}
	\end{exampleblock}
\end{frame}

\subsection{求极限:$\displaystyle0\cdot\infty$型}
\begin{frame}{求极限:$\displaystyle0\cdot\infty$型}
	\everymath{\displaystyle}
	\begin{block}{求极限}
		\begin{columns}[t]
			\begin{column}{0.5\textwidth}
				\begin{enumerate}
					\item $\lim_{x \rightarrow 0} x^2 e^{\frac{1}{x^2}}$;
					\item $\lim_{x \rightarrow +\infty} x^2 \mathrm{e}^{-0.01 x}$;
				\end{enumerate}
			\end{column}
			\begin{column}{0.5\textwidth}
				\begin{enumerate}
					\setcounter{enumi}{2}
					\item $\lim_{x \rightarrow 1-0} \ln x \cdot \ln (1-x)$;
					\item $\lim_{x \rightarrow +0} x^{\varepsilon} \ln x \,\,(\varepsilon>0)$.
				\end{enumerate}
			\end{column}
		\end{columns}
	\end{block}

	\begin{exampleblock}{答案}
		\begin{columns}[t]
			\begin{column}{0.5\textwidth}
				\begin{enumerate}
					\pause
					\item $\infty$;
					\item $0$;
				\end{enumerate}
			\end{column}
			\begin{column}{0.5\textwidth}
				\begin{enumerate}
					\setcounter{enumi}{2}
					\pause
					\item $0$;
					\item $0$.
				\end{enumerate}
			\end{column}
		\end{columns}
	\end{exampleblock}
\end{frame}


% Thank you page
\beamertemplateshadingbackground{structure.fg!90}{structure.fg}
\begin{frame}[plain]
	\vfill
	\centering
	{
		\centering \Huge \color{white} Thank you for your attention!\\[10pt]Questions?\\Homework: Page: 120: 17、23、26\\\vspace{0.4cm}Page: 123:  46、47、57
	}
	\vfill
\end{frame}


\end{document}


