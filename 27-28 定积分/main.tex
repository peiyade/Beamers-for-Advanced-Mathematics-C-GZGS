% !TEX encoding = UTF-8 Unicode.

% Based on https://github.com/Miracle0565/BUCT-Beamer-Theme

\documentclass[
10pt,
aspectratio=43,
]{beamer}
\setbeamercovered{transparent=10}
\usetheme[
%  showheader,
%  red,
  purple,
%  gray,
%  graytitle,
  colorblocks,
%  noframetitlerule,
]{Verona}

\usepackage[T1]{fontenc}
\usepackage{tikz}
\usepackage[utf8]{inputenc}
\usepackage{lipsum}
\usepackage{pgfplots}
\definecolor{accent1}{RGB}{74,172,234}
\definecolor{accent3}{RGB}{180,0,0}
\definecolor{filler4}{RGB}{180,180,180}
%%%%%%%%%%%%%%%%%%%%%%%%%%%%%%%
% Mac上使用如下命令声明隶书字体, windows也有相关方式, 大家可自行修改
\providecommand{\lishu}{\CJKfamily{zhli}}
%%%%%%%%%%%%%%%%%%%%%%%%%%%%%%%
\usepackage{tikz}
\usetikzlibrary{fadings}
%
%\setbeamertemplate{sections/subsections in toc}[ball]
\usepackage{xeCJK}
\usepackage{listings}
\usepackage{caption}
\usepackage{subfigure}
\usefonttheme{professionalfonts}
\def\mathfamilydefault{\rmdefault}
\usepackage{amsmath}
\usepackage{multirow}
\usepackage{booktabs}
\usepackage{bm}
\usepackage{mathtools}
\usepackage[T1]{fontenc}
\setbeamertemplate{section in toc}{\hspace*{1em}\inserttocsectionnumber.~\inserttocsection\par}
\setbeamertemplate{subsection in toc}{\hspace*{2em}\inserttocsectionnumber.\inserttocsubsectionnumber.~\inserttocsubsection\par}
\setbeamerfont{subsection in toc}{size=\small}
\AtBeginSection[]{%
	\begin{frame}%
		\frametitle{Outline}%
		\textbf{\tableofcontents[currentsection]} %
	\end{frame}%
}

\AtBeginSubsection[]{%
	\begin{frame}%
		\frametitle{Outline}%
		\textbf{\tableofcontents[currentsection, currentsubsection]} %
	\end{frame}%
}

\title{高等数学C}
%\subtitle{A Simple while elegant template}
\author[P.Yu]{余沛}
\mail{peiy\_gzgs@qq.com}
\institute[Guangzhou College of Technology and Business]{Guangzhou College of Technology and Business \\
  广州工商学院}
\date{\today}
\titlegraphic[width=4cm]{logo.png}{}




%%%%%%%%%%%%%%%%%%%%%%%%%%%%%%%%
% ----------- 标题页 ------------
%%%%%%%%%%%%%%%%%%%%%%%%%%%%%%%%



\begin{document}

\maketitle

%%% define code
\defverbatim[colored]\lstI{
	\begin{lstlisting}[language=C++,basicstyle=\ttfamily,keywordstyle=\color{red}]
	int main() {
	// Define variables at the beginning
	// of the block, as in C:
	CStash intStash, stringStash;
	int i;
	char* cp;
	ifstream in;
	string line;
	[...]
	\end{lstlisting}
}
%%%%%%%%%%%%%%%%%%%%%%%%%%%%%%%%
% ----------- FRAME ------------
%%%%%%%%%%%%%%%%%%%%%%%%%%%%%%%%

\section{定积分的概念}
\subsection{例子}
\subsubsection{面积}
\begin{frame}
	\frametitle{面积的性质}
	\everymath{\displaystyle}
	回顾一下面积的概念, 对于任意(可以求面积的)集合$A_1, A_2,\cdots A_n$, 面积函数 $m(\cdot)$ 为从集合到非负实数的映射, 应该至少具有以下几个性质:
	\vspace{0.5cm}
	\begin{enumerate}
		\item 单调性: 若 $A_1\subset A_2$, 那么 $m(A_1)\le m(A_2)$;
		\item 并可测性: 可求面积的集合的并可求面积, $m\left(\bigcup_{i=1}^{\infty} E_i\right) \leq \sum_{i=1}^{\infty} m\left(E_i\right)$.
	\end{enumerate}
	\vspace{1cm}
	那么, 对于一般的集合, 怎么定义"面积映射", 或者说, "面积".
\end{frame}

\begin{frame}
	\frametitle{面积的性质}
	\everymath{\displaystyle}
	回顾一下面积的概念, 我们知道: 对于长方形来说
	\[
		\text{面积}=\text{长}\times\text{宽}.
	\]
	对于直角三角形, 也有:
	\[
		\text{面积}=\frac12\times\text{底}\times\text{高}.
	\]
	Wait, 回顾并可测性:
	\begin{quote}
		可求面积的集合的并可求面积, $m\left(\bigcup_{i=1}^{\infty} E_i\right) \leq \sum_{i=1}^{\infty} m\left(E_i\right)$.
	\end{quote}
	暂时只能得到:
	\[
		\text{面积}\le\frac12\times\text{底}\times\text{高}.
	\]
	而且, 我们总希望用尽可能简单地定义来进行面积的定义.
\end{frame}

\begin{frame}
	\frametitle{三角形的面积计算-Darboux上下和版本}
	\everymath{\displaystyle}
	\begin{center}
		\begin{tikzpicture}[
			declare function={f = -3*(\x-5)+0.4*(\x-5)^3+10;},
			scale=1
		]
		% define some special values
\pgfmathsetmacro\sxa{-3*(4.5-5)+0.4*(4.5-5)^3+10}
\pgfmathsetmacro\sxb{-3*(4.25-5)+0.4*(4.25-5)^3+10}
\pgfmathsetmacro\sxc{-3*(4.125-5)+0.4*(4.125-5)^3+10}
		\begin{axis}[
			x=1.5cm, y=0.4cm,
			axis lines=middle,
			ymin=-1.5, ymax=13.5,
			xmin=1.5, xmax=9,
			yticklabels={,,},
			xticklabels={,,}
		]
		\addplot[fill=gray, fill opacity=0.1, thick, domain=2:8] {f}\closedcycle;
		\node at (axis cs:2,-1) {$a$};
		\node at (axis cs:8,-1) {$b$};
		
		\only<2->{\node at (axis cs:6,12) {$y=f(x)\geq0$};}
		\only<3-8>{
		\node at (axis cs:3,-1) {$x_1$};
		\node at (axis cs:7,-1) {$x_{n-1}$};
		\node at (axis cs:4,-1) {$x_{i-1}$};
		\node at (axis cs:5,-1) {$x_i$};}
		\only<3->{
		\addplot[dashed, thin, domain=2:3] {f}\closedcycle;
		\addplot[dashed, thin, domain=3:4] {f}\closedcycle;
		\addplot[dashed, thin, domain=4:5] {f}\closedcycle;
		\addplot[dashed, thin, domain=6:7] {f}\closedcycle;
		\addplot[dashed, thin, domain=7:8] {f}\closedcycle;
		}
		
		\only<4-8>{
		\addplot[draw=accent3, fill=accent1, fill opacity=0.1, thin, dashed] coordinates {(4,\sxa) (5,\sxa)}\closedcycle;
		}
		\only<5-8>{
		\node at (axis cs:4.5,-1) {$\xi_i$};
		\node at (axis cs:4.5,10)[thick] {$f(\xi_i)$};
		\addplot [mark=none, color=accent1, thick, dashed] coordinates{(4.5,0) (4.5,\sxa)};
		}
		\only<6-8>{
		\node at (axis cs:4.5,0.6) {$\Delta x_i$};
		}
		
		\only<8>{
		\addplot [
			integral segments=6,% number of subintervals 20
			draw=accent3,
			fill=filler4,
			integral=2:8
		] {f};
		\addplot[draw=accent3, fill=accent1, fill opacity=0.1, thin, dashed] coordinates {(4,\sxa) (5,\sxa)}\closedcycle;
		\addplot [color=accent1, thick, dashed] coordinates{(4.5,0) (4.5,\sxa)};
		\node at (axis cs:4.5,10)[thick] {$f(\xi_i)$};
		\node at (axis cs:4.5,0.6) {$\Delta x_i$};
		}
		
		\only<9|handout:0>{
		\addplot [
			integral segments=12,% number of subintervals 20
			draw=accent3,
			fill=filler4,
			integral=2:8
		] {f};
		\addplot[draw=accent3, fill=accent1, fill opacity=0.1, thin, dashed] coordinates {(4,\sxb) (4.5,\sxb)}\closedcycle;
		\addplot [color=accent1, thick, dashed] coordinates{(4.25,0) (4.25,\sxb)}\closedcycle;
		\node at (axis cs:2.5,-1) {$x_1$};
		\node at (axis cs:7.35,-1) {$x_{n-1}$};
		\node at (axis cs:3.7,-1) {$x_{i-1}$};
		\node at (axis cs:4.25,-1) {$\xi_i$};
		\node at (axis cs:4.65,-1) {$x_i$};
		}
		
		\only<10-|handout:0>{
		\addplot [
			integral segments=24,% number of subintervals 20
			draw=accent3,
			fill=filler4,
			integral=2:8
		] {f};
		\addplot[accent3, fill=accent1, fill opacity=0.1, thin, dashed] coordinates {(4,\sxc) (4.25,\sxc)}\closedcycle;
		\addplot [color=accent1, thick, dashed] coordinates{(4.125,0) (4.125,\sxc)}\closedcycle;
		\node at (axis cs:2.4,-1) {$x_1$};
		\node at (axis cs:7.5,-1) {$x_{n-1}$};
		\node at (axis cs:3.7,-1) {$x_{i-1}$};
		\node at (axis cs:4.2,-1) {$\xi_i$};
		\node at (axis cs:4.6,-1) {$x_i$};
		}
		\only<4->{\addplot [thick,domain=2:8] {f};}
		
		\end{axis}
		\end{tikzpicture}
	\end{center}
\end{frame}


\subsubsection{曲边梯形的面积}
\begin{frame}
	\frametitle{曲边梯形的面积计算方式-Darboux上下和版本}
	这种计算方法虽然繁琐, 但可以得到更


\end{frame}

\subsection{定积分的定义}
\begin{frame}
	\frametitle{定积分的定义}
	\everymath{\displaystyle}
	回顾求面积的方式, 我们提出定积分的定义:
	\begin{block}{定积分的定义}
		设$f(x)$在区间$[a,b]$上有定义, 用点$a=x_0<x_1<\cdots<x_{n-1}<x_n=b$
		将区间, {\bf 任意}分为$n$段$[x_{i-1},x_i]$ ($i=1,2,\cdots,n$), 其长度分别为$\Delta x_i=x_i-x_{i-1}$.
		在每段$[x_{i-1},x_i]$上, {\bf 任意}选取点$\xi_i$, 得到近似和$$\sum_{i=1}^n f(\xi_i)\Delta x_i.$$
		记$\Delta x=\max_i\{\Delta x_i\}$, 如果$\Delta x \to 0$时近似和的极限存在,
		我们就称$f(x)$在区间$[a,b]$上是, {\bf 可积的}, 并将这个极限值称为$f(x)$在$[a,b]$上的, {\bf 定积分}.
		记为
		\[
			\lim_{\Delta x\to 0} \sum_{i=1}^n f(\xi_i)\Delta x_i = \int_a^b f(x)\mathrm{~d} x.
		\]
	\end{block}
\end{frame}




\subsubsection{定积分的极限定义}
\subsection{定义法求积分}
\section{定积分的性质}
\subsection{基本性质}

\begin{frame}
	\frametitle{定积分}
	我们已经定义
	\[
		\int_a^b f(x)\mathrm{~d}x  = \lim_{\Delta x\to 0} \sum_{i=1}^n f(\xi_i)\Delta x_i
	\]
	其中:
	\begin{itemize}
		\item $x$称为{\bf 积分变量},$f(x)$称为{\bf 被积函数},$f(x)\mathrm{~d}x$称为{\bf 被积表达式}
		\item $a$称为{\bf 积分下限},$b$称为{\bf 积分上限},$[a,b]$称为{\bf 积分区间}
	\end{itemize}
	\begin{block}{}
		定积分的值只与被积函数$f$和积分区间$[a,b]$有关,而与积分变量用什么字母无关.即有
		\[
			\int_a^b f(x)\mathrm{~d}x = \int_a^b f(t)\mathrm{~d}t = \int_a^b f(u)\mathrm{~d}u.
		\]
		如果$f(x)$在区间$[a,b]$上是连续函数(或者是只有有限个间断点的有界函数),则它在$[a,b]$上是可积的.
	\end{block}
\end{frame}

\begin{frame}
	\frametitle{定积分}
	\begin{block}{}
		如果$a>b$,我们规定
		$$
			\int_a^b f(x)\mathrm{~d} x = -\int_b^a f(x)\mathrm{~d} x,\text{ 特别地,若 }\,\,a=b,\quad \int_a^b f(x)\mathrm{~d} x = 0 .
		$$
	\end{block}
	\begin{block}{}
		设由曲线$y=f(x)$, 直线$x=a$, $x=b$, 和$x$轴所围成的曲边梯形的面积为$S$
		\begin{itemize}
			\item 如果在$[a,b]$上$f(x)\geq 0$,则定积分
			      $$
				      \int_a^b f(x)\mathrm{~d} x=S.
			      $$\
			\item 如果在$[a,b]$上$f(x)\leq 0$,则定积分$$\int_a^b f(x)\mathrm{~d} x=-S.$$
		\end{itemize}
	\end{block}
\end{frame}


\begin{frame}
	\begin{block}{数乘性质}设$k$为常数,则有
		\[
			\int_a^b kf(x)\mathrm{~d} x=k\int_a^b f(x)\mathrm{~d} x
		\]
	\end{block}
	\begin{block}{函数可加性}
		\[
			\int_a^b [f(x) \pm g(x)]\mathrm{~d} x = \int_a^b f(x)\mathrm{~d} x \pm \int_a^b g(x) \mathrm{~d} x
		\]
	\end{block}
	\begin{block}{区间可加性}
		设$a<c<b$,则有
		\[
			\int_a^b f(x)\mathrm{~d} x = \int_a^c f(x)\mathrm{~d} x + \int_c^b f(x)\mathrm{~d} x
		\]
	\end{block}
	即使$c$不在$a$和$b$之间,上述性质依然是成立的.
\end{frame}

\begin{frame}
	\begin{block}{}
		\[
			\int_a^b 1\mathrm{~d} x = \int_a^b \mathrm{~d} x= b-a
		\]
	\end{block}
	\begin{block}{}
		设在区间$[a,b]$上$f(x)\geq g(x)$,则有
		\[
			\int_a^b f(x)\mathrm{~d} x \geq \int_a^b g(x)\mathrm{~d} x.
		\]
		特别地,如果在区间$[a,b]$上$f(x)\geq 0$,则有
		\[
			\int_a^b f(x)\mathrm{~d} x \geq 0.
		\]
	\end{block}
	\begin{block}{}
		$$
			\left|\,\int_a^b f(x)\mathrm{~d} x\,\right| \le \int_a^b\big|f(x)\big|\mathrm{~d} x.
		$$
	\end{block}
\end{frame}

\begin{frame}
	\frametitle{比较积分大小}
	\everymath{\displaystyle}
	\begin{block}{}
		\begin{columns}
			\begin{column}{0.5\textwidth}
				\begin{enumerate}
					\item $\int_0^1 x \mathrm{~d} x$ 和 $\int_0^1 x^2 \mathrm{~d} x$;
					\item	$\int_{-\frac{\pi}{2}}^0 \sin x \mathrm{~d} x$ 和 $\int_0^{\frac{\pi}{2}} \sin x \mathrm{~d} x$;
				\end{enumerate}
			\end{column}
			\begin{column}{0.5\textwidth}
				\begin{enumerate}
					\item $\int_1^2 x \mathrm{~d} x$ 和 $\int_1^2 x^2 \mathrm{~d} x$;
					\item $\int_0^{\frac{\pi}{4}} \sin x \mathrm{~d} x$ 和 $\int_0^{\frac{\pi}{4}} \cos x \mathrm{~d} x$.
				\end{enumerate}
			\end{column}
		\end{columns}
	\end{block}
	\begin{exampleblock}{}
		\begin{columns}
			\begin{column}{0.5\textwidth}
				\begin{enumerate}
					\item $\int_0^1 x \mathrm{~d} x\ge\int_0^1 x^2 \mathrm{~d} x$;
					\item	$\int_{-\frac{\pi}{2}}^0 \sin x \mathrm{~d} x\le\int_0^{\frac{\pi}{2}} \sin x \mathrm{~d} x$;
				\end{enumerate}
			\end{column}
			\begin{column}{0.5\textwidth}
				\begin{enumerate}
					\item $\int_1^2 x \mathrm{~d} x\le\int_1^2 x^2 \mathrm{~d} x$;
					\item $\int_0^{\frac{\pi}{4}} \sin x \mathrm{~d} x=\int_0^{\frac{\pi}{4}} \cos x \mathrm{~d} x$.
				\end{enumerate}
			\end{column}
		\end{columns}
	\end{exampleblock}
\end{frame}

\subsection{中值定理}
\begin{frame}
	\begin{block}{积分中值定理}
	设$f(x)$在$[a,b]$上连续,则在$[a,b]$中至少存在一点$\xi$,使得
	\[ \int_a^b f(x)\mathrm{~d} x = f(\xi)(b-a) \]
	\end{block}
	\begin{block}上述性质也是说,存在$\xi\in[a,b]$,使得
	\[ \frac1{b-a} \int_a^b f(x)\mathrm{~d} x = f(\xi)\]
	说明连续函数在区间$[a,b]$上的平均值是可以取到的.
	\end{block}
	\end{frame}
% Thank you page
\beamertemplateshadingbackground{structure.fg!90}{structure.fg}
\begin{frame}[plain]
	\vfill
	\centering
	{
		\centering \Huge \color{white} Thank you for your attention!\\[10pt]Questions?
	}
	\vfill
\end{frame}


\end{document}


