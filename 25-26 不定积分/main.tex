% !TEX encoding = UTF-8 Unicode.

% Based on https://github.com/Miracle0565/BUCT-Beamer-Theme

\documentclass[
10pt,
aspectratio=43,
]{beamer}
\setbeamercovered{transparent=10}
\usetheme[
%  showheader,
%  red,
  purple,
%  gray,
%  graytitle,
  colorblocks,
%  noframetitlerule,
]{Verona}

\usepackage[T1]{fontenc}
\usepackage{tikz}
\usepackage[utf8]{inputenc}
\usepackage{lipsum}
%%%%%%%%%%%%%%%%%%%%%%%%%%%%%%%
% Mac上使用如下命令声明隶书字体, windows也有相关方式, 大家可自行修改
\providecommand{\lishu}{\CJKfamily{zhli}}
%%%%%%%%%%%%%%%%%%%%%%%%%%%%%%%
\usepackage{tikz}
\usetikzlibrary{fadings}
%
%\setbeamertemplate{sections/subsections in toc}[ball]
\usepackage{xeCJK}
\usepackage{listings}
\usepackage{caption}
\usepackage{subfigure}
\usefonttheme{professionalfonts}
\def\mathfamilydefault{\rmdefault}
\usepackage{amsmath}
\usepackage{multirow}
\usepackage{booktabs}
\usepackage{bm}
\usepackage{mathtools}
\usepackage[T1]{fontenc}
\setbeamertemplate{section in toc}{\hspace*{1em}\inserttocsectionnumber.~\inserttocsection\par}
\setbeamertemplate{subsection in toc}{\hspace*{2em}\inserttocsectionnumber.\inserttocsubsectionnumber.~\inserttocsubsection\par}
\setbeamerfont{subsection in toc}{size=\small}
\AtBeginSection[]{%
	\begin{frame}%
		\frametitle{Outline}%
		\textbf{\tableofcontents[currentsection]} %
	\end{frame}%
}

\AtBeginSubsection[]{%
	\begin{frame}%
		\frametitle{Outline}%
		\textbf{\tableofcontents[currentsection, currentsubsection]} %
	\end{frame}%
}

\title{高等数学C}
%\subtitle{A Simple while elegant template}
\author[P.Yu]{余沛}
\mail{peiy\_gzgs@qq.com}
\institute[Guangzhou College of Technology and Business]{Guangzhou College of Technology and Business \\
  广州工商学院}
\date{\today}
\titlegraphic[width=4cm]{logo.png}{}




%%%%%%%%%%%%%%%%%%%%%%%%%%%%%%%%
% ----------- 标题页 ------------
%%%%%%%%%%%%%%%%%%%%%%%%%%%%%%%%



\begin{document}

\maketitle

%%% define code
\defverbatim[colored]\lstI{
	\begin{lstlisting}[language=C++,basicstyle=\ttfamily,keywordstyle=\color{red}]
	int main() {
	// Define variables at the beginning
	// of the block, as in C:
	CStash intStash, stringStash;
	int i;
	char* cp;
	ifstream in;
	string line;
	[...]
	\end{lstlisting}
}
%%%%%%%%%%%%%%%%%%%%%%%%%%%%%%%%
% ----------- FRAME ------------
%%%%%%%%%%%%%%%%%%%%%%%%%%%%%%%%

\section{不定积分的概念}

\subsection{算子, 逆算子, 微分算子, 微分逆算子, 原函数, 不定积分}
\begin{frame}
	\frametitle{算子和逆算子}
	{\small
		\begin{block}{算子}
			算子是指将集合中的元素指向另一个元素的映射. 在这里我们先用大写字母表示, 比如$A$和$B$.
		\end{block}
		\pause
		\begin{block}{逆算子}
			给定一个算子$A$, 如果存在另一个算子$B$, 使得$A$和$B$的复合运算等于单位算子$I$(恒等映射), 即$AB=BA=I$, 则称$B$是$A$的逆算子, 记作 $B=A^{-1}$.
		\end{block}
		\pause
		\begin{exampleblock}{例子:数乘算子与数乘算子的逆}
			考虑数乘算子$A$, 它将实数$x$映射到它的2倍, 即$A(x) = 2x$. 那么$A$的逆算子$B$将实数$x$映射到一半, 即$B(x) = \frac{1}{2}x$. 在这个例子中, $B$是数除算子. \pause $B=A^{-1}$, 这个求逆运算也是减法计算的来源.
		\end{exampleblock}
		\pause
		\begin{exampleblock}{例子:数加算子与数加算子的逆}
			考虑数加算子$A$, 它将实数$x$映射到它加$1$, 即$A(x) = x+1$. 那么$A$的逆算子$B$将实数$x$映射到它减一, 即$B(x) = x-1$. 在这个例子中, $B$就是数减算子. \pause $B=A^{-1}$, 这个求逆运算也是除法计算的来源.
		\end{exampleblock}
	}
\end{frame}

\begin{frame}
	\frametitle{微分算子}
	回顾到微分运算, 回顾微分运算的本质:
	\begin{block}{微分算子}
		在微积分中, 微分算子是指将一个函数映射到它的导数的操作符.
		$$
			\frac{\mathrm{d}}{\mathrm{d}x}f(x)=f'(x)
		$$
		在这里, 我们用标准的微分符号$\frac{\mathrm{d}}{\mathrm{d}x}$来描述某个自变量为 $x$ 的函数的微分算子.
	\end{block}
	\pause
	\vspace{0.2cm}
	\vfill
	\centering
	{
		\centering
		那么
	}
	\pause
	\vfill
	\centering
	{
		\centering
		微分算子的逆运算是什么? \pause 姑且先记录为 $\displaystyle\left(\frac{\mathrm{d}}{\mathrm{d}x}\right)^{-1}$. \pause 又会有什么作用?
	}
\end{frame}

\begin{frame}
	\frametitle{微分算子的逆}
	\begin{block}{}
		要对算子 $\left(\frac{\mathrm{d}}{\mathrm{d}x}\right)^{-1}$ 进行刻画, \pause 就要问以下两个问题:
		\vspace{0.3cm}
		\begin{enumerate}
			\pause
			\item 对于哪些函数来说, 有
			      $$
				      \displaystyle\left(\frac{\mathrm{d}}{\mathrm{d}x}\right)^{-1}\left(\frac{\mathrm{d}}{\mathrm{d}x}\right)=I?
			      $$
			      \vspace{0.2cm}
			      \pause
			\item $\displaystyle\left(\frac{\mathrm{d}}{\mathrm{d}x}\right)^{-1}$ 应当怎么计算?
		\end{enumerate}
		\vspace{0.3cm}
	\end{block}
\end{frame}

\begin{frame}
	\frametitle{逆算子: 原函数与原函数的唯一性}
	首先我们给出原函数的概念:
	\begin{block}{}
		对于函数$f(x)$, 如果存在函数 $F(x)$ 满足 $\frac{\mathrm{d}}{\mathrm{d}x} F(x) = f(x)$, 那么自然地,
		$$
			F(x)=\displaystyle\left(\frac{\mathrm{d}}{\mathrm{d}x}\right)^{-1}f(x).
		$$
		在这种情况下, 我们称 $F(x)$ 为 $f(x)$ 的原函数.
	\end{block}
	\pause
	我们可以看到, 原函数是不唯一的.
	\pause
	\begin{block}{}
		对于函数$f(x)$, 假如 $F(x)$ 为 $f(x)$ 的原函数, 那么对于任意常数 $C$, 可以发现 $\frac{\mathrm{d}}{\mathrm{d}x} \left(F(x)+C\right) = f(x)$ 也成立.
	\end{block}
	\pause
	\begin{block}{}
		\vfill
		\centering
		{
			\centering
			没想到吧!
		}
	\end{block}
	但是但是, 如果我们不考虑这个常数, 那么连续函数的原函数其实是唯一的.
\end{frame}

\begin{frame}
	\frametitle{逆算子: 原函数与原函数的唯一性}
	\begin{theorem}
		连续函数 $f(x)$ 存在原函数, 而且对于任意两个原函数 $F_1(x)$ 和 $F_2(x)$ 只差一个常数, 即 存在常数 $C$ 使得 $F(x)-G(x)=C$.
	\end{theorem}
	\begin{block}{Proof.}
		我们先证明唯一性, 可以发现
		$$
			(F(x)-G(x))'=f(x)-f(x)=0.
		$$
		由于Lagrange中值定理, 假如 $F(x)-G(x)\neq C$, 对于任意 不满足 $(F(x_1)-G(x_1))-(F(x_2)-G(x_2))=0$ 的 $x_1, x_2$, 存在 $\xi\in(x_1,x_2)$, 满足
		$$
			(F(\xi)-G(\xi))'=\frac{(F(x_1)-G(x_1))-(F(x_2)-G(x_2))}{x_1-x_2}\neq 0.
		$$
		存在性将在学习定积分的时候给出, 有一个具体的计算方法.
		\qed
	\end{block}
\end{frame}

\begin{frame}
	\frametitle{微分算子的逆: 不定积分算子}

	微分算子的逆, \pause 这种函数$f(x)$到原函数$F(x)$的算子一般被称为:
	$$
		\text{不定积分算子:}\pause\,\,\,\, \int\cdot\,\,\mathrm{d}x:\,\,\pause f(x)\rightarrow F(x)\pause +C.
	$$

	\pause 这里的 $+C$ 是为了说明差一个常数, \pause $\int\cdot\,\,\mathrm{d}x$ 是所谓的 Leibniz 记号.
	\vspace{0.2cm}

	这记号的好处是, 对于微分计算$f(x)\mathrm{d}x=\mathrm{d}F(x)$
	\vspace{0.2cm}
	$$
		\int f(x) \mathrm{d}x = \int \mathrm{d} F(x) = F(x) +C,\quad \text{及}\quad \mathrm{d}\left(\int f(x)\mathrm{d}x\right) = f(x).
	$$
	\vspace{0.2cm}
	\pause 该记号能够更好地将微分算子 $\mathrm{d}$ 和 积分算子 $\int $(微分算子的逆算) 具有 $\mathrm{d}\int=\int \mathrm{d}= I $(相差一个常数)的性质.
\end{frame}

\begin{frame}
	\frametitle{来点练习}
	\everymath{\displaystyle}
	最最标准的求不定积分的方法是直接找到对应的原函数.
	{\small
	\begin{exampleblock}{证明: $\int x^5 \mathrm{d} x=\frac{1}{6} x^6+C$}
		\pause
		$
			\left(\frac{1}{6} x^6\right)^{\prime}=x^5 \pause \Rightarrow \int x^5 d x=\frac{1}{6} x^6+C.
		$
	\end{exampleblock}
	\pause
	\begin{exampleblock}{证明: $\int \frac1x \mathrm{d} x=\ln |x|+C$}
		\pause
		对于 $x>0$ 的情况, 有
		$
			(\ln x)^{\prime}=\frac{1}{x} \Rightarrow \int \frac{1}{x} \mathrm{d} x=\ln x+C.
		$
		\pause

		对于 $x<0$ 的情况, 有
		$
			(\ln (-x))'=\frac{1}{-x}(-x)^{\prime}=\frac{1}{x} \Rightarrow \int \frac{1}{x} \mathrm{d} x=\ln -x+C.
		$
	\end{exampleblock}

	\begin{exampleblock}{证明: $\int \frac{\mathrm{d} x}{\cos ^2 x}=\tan x+C$}
		\pause
		$
			\int \frac{\mathrm{d} x}{\cos ^2 x}=\pause\int \sec ^2 x \mathrm{d} x=\pause\int \mathrm{d} \tan x=\pause\tan x+ C.
		$
	\end{exampleblock}
	}
\end{frame}

\subsection{基本不定积分公式和基本微分公式}
\begin{frame}
	\frametitle{从微分公式到积分公式}
	\everymath{\displaystyle}
	\renewcommand{\arraystretch}{2.3}
	\resizebox{\linewidth}{!}
	{
		\begin{tabular}{|p{0.35\textwidth}|p{0.45\textwidth}|}
			\hline
			$\mathrm{d}x^a=ax^{a-1}\mathrm{d}x$, $a\neq0$                   & $\int x^{a-1}\mathrm{d}x=\frac{1}{a}x^{a}+C$, $a\neq0$  \\
			\hline
			$\mathrm{d}\sin x=\cos x\mathrm{d}x$                            & $\int \cos x\mathrm{d}x = \sin x+C$                     \\
			\hline
			$\mathrm{d}\cos x=-\sin x\mathrm{d}x$                           & $\int \sin x\mathrm{d}x=-\cos x +C$                     \\
			\hline
			$\mathrm{d}\log_ax=\frac{1}{\log a}\cdot\frac{1}{x}\mathrm{d}x$ & $\int \frac{1}{x}\mathrm{d}x = \log |x|+C$              \\
			\hline
			$\mathrm{d}a^x=\log a^x\mathrm{d}x$                             & $\int a^x\mathrm{d}x = \frac{1}{\log a}a^x+C$           \\
			\hline
			$\mathrm{d}\arcsin x =\frac{1}{\sqrt{1-x^2}}\mathrm{d}x$        & $\int \frac{1}{\sqrt{1-x^2}}\mathrm{d}x = \arcsin x +C$ \\
			\hline
			$\mathrm{d}\arctan x =\frac{1}{1+x^2}\mathrm{d}x$               & $\int \frac{1}{1+x^2}\mathrm{d}x = \arctan x +C$        \\
			\hline
		\end{tabular}
	}
\end{frame}



\begin{frame}
	\frametitle{从微分计算法则到不定积分计算法则}
	接下来我们回顾微分计算法则, 并导出重要的不定积分计算法则:\pause
	\begin{block}{}
		\begin{enumerate}
			\item 线性叠加法则:
			      \begin{itemize}
				      \item \pause 假设$k, l$为实数, $u, v$ 为函数微分的和差法则和数乘法则:
				            $$
					            (ku'+lv')\mathrm{d}x=\mathrm{d}(ku+ lv) = \mathrm{d}ku + \mathrm{d}lv=ku'\mathrm{d}x+lv'\mathrm{d}x.
				            $$
				      \item \pause 不定积分的和差法则:
				            $$
					            \int(ku'+lv')\mathrm{d}x=\int\mathrm{d}(ku\pm lv) = \int k\mathrm{d}u + l\mathrm{d}v=k\int u'\mathrm{d}x+l\int v'\mathrm{d}x.
				            $$
			      \end{itemize}
			      \pause
			\item 乘积法则:
			      \begin{itemize}
				      \item \pause 微分的乘积法则:
				            $$
					            u\mathrm{d}v+v\mathrm{d}u =\mathrm{d}(uv).
				            $$
				      \item \pause 不定积分的和差法则:
				            $$
					            \int u\mathrm{d}v+\int v\mathrm{d}u=\int\mathrm{d}(uv) = uv + C.
				            $$
				            \pause 最后这一条经常被称为分部积分公式.
			      \end{itemize}
		\end{enumerate}
	\end{block}
\end{frame}

\begin{frame}
	\frametitle{从微分计算法则到不定积分计算法则}
	\begin{block}{}
		\begin{enumerate}\setcounter{enumi}{2}
			\pause \item 商法则:
			      \begin{itemize}
				      \pause \item 微分的商法则
				            $$
					            \mathrm{d}\left(\frac{u}{v}\right)=\frac{u\mathrm{d}v-v\mathrm{d}u}{v^2}.
				            $$
				            \pause \item 不定积分的商法则 \pause (基本不会用到这个法则)
				            $$
					            \frac{u}{v}=\int \frac{u}{v^2}\mathrm{d}v-\int\frac{1}{v}\mathrm{d}u.
				            $$

			      \end{itemize}
			      \pause \item 复合函数微分法则:
			      \begin{itemize}
				      \pause\item 微分的复合函数求导法则, 设$f(x)$ 和 $g(x)$ 是可导函数,记 $h(x) = f(g(x))$
				            $$
					            f'(g(x))g'(x)\mathrm{d}x=f'(g(x))\mathrm{d}g(x)=\mathrm{d}h(x).
				            $$
				      \item \pause 不定积分的和差法则
				            $$
					            \int f'(g(x))g'(x)\mathrm{d}x=\int f'(g(x))\mathrm{d}g(x)=\int \mathrm{d}h(x)= h(x)+C.
				            $$
			      \end{itemize}
		\end{enumerate}
	\end{block}
\end{frame}

\subsection{熟悉直接积分法}
\begin{frame}
	\frametitle{熟悉直接积分法}
	\everymath{\displaystyle}
	{\small
		\begin{exampleblock}{$\int \sqrt{x}\left(x^2-5\right) \mathrm{d} x$}
			$
				\int \sqrt{x}\left(x^2-5\right) \mathrm{d}x= \pause \int x^{\frac{5}{2}}\mathrm{d}x-\int 5x^{\frac{1}{2}} \mathrm{d}x \pause =\frac{7}{2}x^{\frac{2}{7}}-\frac{10}{3}x^{\frac{3}{2}}+C.
			$
		\end{exampleblock}
		\pause
		\begin{exampleblock}{$\int \frac{(x-1)^3}{x^2} \mathrm{d} x$}
			$
				\int \frac{(x-1)^3}{x^2} \mathrm{d} x= \pause \int x-3+3\frac{1}{x}-\frac{1}{x^2} \mathrm{d}x= \pause \frac{1}{2}x^2+3x+3\log|x|+\frac{1}{x}+C.
			$
		\end{exampleblock}
		\pause
		\begin{exampleblock}{$\int\left(e^x-3 \cos x\right) \mathrm{d} x$}
			$
				\int\left(e^x-3 \cos x\right) \mathrm{d} x=\pause\int e^x\mathrm{d} x-3 \int\cos x \mathrm{d} x=\pause e^x-3\sin x+C.
			$
		\end{exampleblock}
	}
\end{frame}

\begin{frame}
	\frametitle{熟悉直接积分法}
	\everymath{\displaystyle}
	{\small
		\begin{exampleblock}{$\int \frac{2 x^4+x^2+3}{x^2+1} \mathrm{d} x$}
			$
				\int \frac{2 x^4+x^2+3}{x^2+1} \mathrm{d} x =\pause \int \frac{2 x^4+2x^2}{x^2+1}+\frac{-x^2+3}{x^2+1} \mathrm{d} x =\pause \int 2x^2-1+4 \frac{1}{x^2+1}\mathrm{d} x,
			$
			\pause
			$
				\int \frac{2 x^4+x^2+3}{x^2+1} \mathrm{d} x = \pause\frac{2}{3}x^3-x+4\arctan x+ C.
			$
		\end{exampleblock}
		\begin{exampleblock}{$\int \tan^2 x \mathrm{d} x$}
			$
				\int \tan^2 x \mathrm{d} x =\pause \int \frac{\sin^2 x}{\cos^2}\mathrm{d} x =\pause \int \frac{1}{\cos^2 x}-1\mathrm{d} x=\tan x-x+C.
			$
		\end{exampleblock}

		\begin{exampleblock}{$\int \frac{1}{\sin ^2 \frac{x}{2} \cos ^2 \frac{x}{2}} \mathrm{d} x$}
			$
				\int \frac{1}{\sin ^2 \frac{x}{2} \cos ^2 \frac{x}{2}} \mathrm{d} x =\pause \int \frac{1}{(\frac{1}{2}\sin x)^2} \mathrm{d} x =\pause \int \frac{4}{\sin^2 x}\mathrm{d} x=\pause-4\cot x+C.
			$
		\end{exampleblock}
	}
\end{frame}

\section{基本积分法}

\subsection{变量代换方法一: 凑微分法}

 
\begin{frame}
	\frametitle{凑微分法}
	\everymath{\displaystyle}
	回顾不定积分的和差法则
	$$
		\int f'(g(x))g'(x)\mathrm{d}x=\int f'(g(x))\mathrm{d}g(x)=\int \mathrm{d}f(g(x))= f(g(x))+C.
	$$
	\pause 
	用这种办法, 如果能用函数凑出因子 $g'(x)$, 而 $g(x)$ 作为变量时, 能简化计算...这种方法就称为凑微分法.
	\vspace{0.3cm}

	来看一个例子:
	\pause 
	\vspace{0.1cm}
	\begin{exampleblock}{$\int(2 x+1)^6 \mathrm{d} x$}
		$$
			\begin{aligned}
				\int(2 x+1)^6 \mathrm{d} x & \pause=  \int(2 x+1)^6 \frac{1}{2}\cdot 2 \mathrm{d} x \\
				                           & \pause=  \frac{1}{2}\int(2 x+1)^6 \mathrm{d} (2x+1)         \\
				                           & \pause= \frac{1}{14}(2x+1)^7+C.
			\end{aligned}
		$$
	\end{exampleblock}

\end{frame}

\begin{frame}
	\frametitle{凑微分法}
	\everymath{\displaystyle}
	\begin{exampleblock}{$\int x \sqrt{1-x^2} \mathrm{d} x$}
		$$
			\begin{aligned}
				\int x \sqrt{1-x^2} \mathrm{d} x & \pause= -\frac{1}{2} \int \sqrt{1-x^2} \mathrm{d}\left(1-x^2\right)       \\
				                                 & \pause= -\frac{1}{2} \cdot \frac{2}{3}\left(1-x^2\right)^{\frac{3}{2}}+C.
			\end{aligned}
		$$
	\end{exampleblock}
	\pause
	\begin{exampleblock}{$\int (3x+2)^{100} \mathrm{d}x$}
		$$
			\begin{aligned}
				\int (3x+2)^{100} \mathrm{d}x & \pause= \frac{1}{3} \int (3x+2)^{100} \mathrm{d}(3x+2)         \\
				\pause\underline{\text{let }u=3x+2} & \pause=  \frac{1}{3} \int u^{100} \mathrm{d}u                  \\
				                              & \pause= \frac{1}{303} u^{101}+C \pause = \frac{1}{303}(3x+2)^{101}+C.
			\end{aligned}
		$$
	\end{exampleblock}
\end{frame}

\begin{frame}
	\frametitle{凑微分法}
	\everymath{\displaystyle}
	{\small
		\begin{exampleblock}{$\int x e^{x^2} \mathrm{d}x$}
			$$
				\begin{aligned}
					\int x e^{x^2} \mathrm{d}x   & \pause= \frac{1}{2} \int e^{x^2} \mathrm{d}(x^2)   \\
					\pause\underline{\text{let }u=x^2} & \pause=  \frac{1}{2} \int e^u \mathrm{d}u          \\
					                             & \pause= \frac{1}{2} e^u+C \pause= \frac{1}{2} e^{x^2}+C.
				\end{aligned}
			$$
		\end{exampleblock}
		\pause
		\begin{exampleblock}{$\int x^4 \cos x^5 \mathrm{d}x$}
			$$
				\begin{aligned}
					\int x^4 \cos x^5 \mathrm{d}x &\pause = \frac{1}{5} \int \cos x^5 \mathrm{d}(x^5)      \\
					\pause\underline{\text{let }u=x^5}  &\pause =  \frac{1}{5} \int \cos u \mathrm{d}u           \\
					                              &\pause = \frac{1}{5} \sin u+C =\pause \frac{1}{5} \sin x^5+C.
				\end{aligned}
			$$
		\end{exampleblock}
	}
\end{frame}


\begin{frame}
	\frametitle{凑微分法}
	\everymath{\displaystyle}
	{\small
		\begin{exampleblock}{$\int \frac{x}{\sqrt{1-x^4}} \mathrm{d}x$}
			$$
				\begin{aligned}
					\int \frac{x}{\sqrt{1-x^4}} \mathrm{d}x & \pause= \frac{1}{2} \int \frac{\mathrm{d}(x^2)}{\sqrt{1-x^4}} \\
					\pause\underline{\text{let }u=x^2}            & \pause=  \frac{1}{2} \int \frac{\mathrm{d}u}{\sqrt{1-u^2}}    \\
					                                        & \pause= \frac{1}{2} \arcsin u+C \pause= \frac{1}{2} \arcsin x^2+C.
				\end{aligned}
			$$
		\end{exampleblock}
		\pause
		\begin{exampleblock}{$\int \frac{\mathrm{d}x}{x(1+2 \ln x)}$}
			$$
				\begin{aligned}
					\int \frac{\mathrm{d}x}{x(1+2 \ln x)} & \pause= \int \frac{1}{1+2 \ln x} \mathrm{d}(\ln x)                  \\
					\pause\underline{\text{let }u=1+2 \ln x}    & \pause=  \frac{1}{2} \int \frac{1}{1+2 \ln x} \mathrm{d}(1+2 \ln x) \\
					                                      & \pause= \frac{1}{2} \ln |1+2 \ln x|+C.
				\end{aligned}
			$$
		\end{exampleblock}
	}
\end{frame}


\begin{frame}
	\frametitle{凑微分法}
	\everymath{\displaystyle}
	{\small
	\begin{exampleblock}{$\int \frac{1}{a^2+x^2} \mathrm{d}x\,\,(a>0)$}
		$$
		\begin{aligned}
			\int \frac{1}{a^2+x^2} \mathrm{d}x   & \pause = \frac{1}{a^2} \int \frac{1}{1+\frac{x^2}{a^2}} \mathrm{d}x   \\
			\pause\underline{\text{let }u=\frac{x}{a}} & \pause =  \frac{1}{a} \int \frac{1}{1+u^2} \mathrm{d}u          \\
												 & \pause = \frac{1}{a} \arctan u+C \pause = \frac{1}{a} \arctan\left(\frac{x}{a}\right)+C.
		\end{aligned}
		$$
	\end{exampleblock}
	\pause
	\begin{exampleblock}{$\int \frac{1}{\sqrt{a^2-x^2}} \mathrm{d}x$}
		$$
		\begin{aligned}
			\int \frac{1}{\sqrt{a^2-x^2}} \mathrm{d}x   & \pause = \frac{1}{a} \int \frac{a}{\sqrt{1-\left(\frac{x}{a}\right)^2}} \mathrm{d}\left(\frac{x}{a}\right)   \\
			\pause\underline{\text{let }u=\frac{x}{a}} & \pause =  \frac{1}{a} \int \frac{1}{\sqrt{1-u^2}} \mathrm{d}u          \\
												 & \pause = \arcsin(u)+C \pause = \arcsin\left(\frac{x}{a}\right)+C.
		\end{aligned}
		$$
	\end{exampleblock}
	}
\end{frame}


\begin{frame}
	\frametitle{凑微分法}
	\everymath{\displaystyle}
	\begin{exampleblock}{$\int \frac{1}{x^2-a^2} \mathrm{d}x$}
		$$
		\begin{aligned}
			\int \frac{1}{x^2-a^2} \mathrm{d}x   & \pause  = \frac{1}{2a} \int \left(\frac{1}{x-a}-\frac{1}{x+a}\right) \mathrm{d}x   \\
			& \pause = \frac{1}{2a}(\ln |x-a|-\ln |x+a|)+C \\
			& \pause  = \frac{1}{2a} \ln \left|\frac{x-a}{x+a}\right|+C.
		\end{aligned}
		$$
	\end{exampleblock}
	\pause 
	\begin{exampleblock}{$\int \sec x \mathrm{d}x$}
		$$
		\begin{aligned}
			\int \sec x \mathrm{d}x   & \pause  = \int \frac{1}{\cos x} \mathrm{d}x \pause  = \int \frac{\cos x}{\cos^2 x} \mathrm{d}x   \\
			& \pause = \int \frac{\mathrm{d}\sin x}{1-\sin^2 x} \pause  = \frac{1}{2} \ln \left|\frac{1+\sin x}{1-\sin x}\right|+C.
		\end{aligned}
		$$
	\end{exampleblock}
\end{frame}

\subsection{变量代换方法二: 变量代换法}

\begin{frame}
	\frametitle{变量代换法}
	\everymath{\displaystyle}
	如果凑不出 $g'(x)$ 使得积分好计算. 
	面对问题
	$$
		\int f(x)\mathrm{d}x
	$$
	可以试着引入变量替换:
	$$
		\begin{aligned}
			x=\phi(x);\,\,\,\, \,\,\,\,\mathrm{d}x=\phi'(t)\mathrm{d}t.
		\end{aligned}
	$$
	将问题演变成
	$$
		\int f(x)\mathrm{d}x=\int f(\phi(t))\phi'(t)\mathrm{d}t 
	$$
	如果演变后的问题方便计算, 假如最后计算出的结果为 $F(t)+C$, 那么再经过逆变换
	$$
		t=\phi^{-1}(x),
	$$
	后可以得到结果 $F(\phi^{-1}x)+C$.
\end{frame}

\subsection{一些基本思路: 根式代换}

\subsection{一些基本思路: 三角代换}

\section{有理函数积分}

% Thank you page
\beamertemplateshadingbackground{structure.fg!90}{structure.fg}
\begin{frame}[plain]
	\vfill
	\centering
	{
		\centering \Huge \color{white} Thank you for your attention!\\[10pt]Questions?
	}
	\vfill
\end{frame}


\end{document}


