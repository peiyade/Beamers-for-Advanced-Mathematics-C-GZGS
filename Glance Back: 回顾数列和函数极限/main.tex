% !TEX encoding = UTF-8 Unicode.

% Based on https://github.com/Miracle0565/BUCT-Beamer-Theme

\documentclass[
10pt,
aspectratio=43,
]{beamer}
\setbeamercovered{transparent=10}
\usetheme[
%  showheader,
%  red,
  purple,
%  gray,
%  graytitle,
  colorblocks,
%  noframetitlerule,
]{Verona}

\usepackage[T1]{fontenc}
\usepackage{tikz}
\usepackage[utf8]{inputenc}
\usepackage{lipsum}
%%%%%%%%%%%%%%%%%%%%%%%%%%%%%%%
% Mac上使用如下命令声明隶书字体, windows也有相关方式, 大家可自行修改
\providecommand{\lishu}{\CJKfamily{zhli}}
%%%%%%%%%%%%%%%%%%%%%%%%%%%%%%%
\usepackage{tikz}
\usetikzlibrary{fadings}
%
%\setbeamertemplate{sections/subsections in toc}[ball]
\usepackage{xeCJK}
\usepackage{listings}
\usepackage{caption}
\usepackage{subfigure}
\usefonttheme{professionalfonts}
\def\mathfamilydefault{\rmdefault}
\usepackage{amsmath}
\usepackage{multirow}
\usepackage{booktabs}
\usepackage{bm}
\setbeamertemplate{section in toc}{\hspace*{1em}\inserttocsectionnumber.~\inserttocsection\par}
\setbeamertemplate{subsection in toc}{\hspace*{2em}\inserttocsectionnumber.\inserttocsubsectionnumber.~\inserttocsubsection\par}
\setbeamerfont{subsection in toc}{size=\small}
\AtBeginSection[]{%
	\begin{frame}%
		\frametitle{Outline}%
		\textbf{\tableofcontents[currentsection]} %
	\end{frame}%
}

\AtBeginSubsection[]{%
	\begin{frame}%
		\frametitle{Outline}%
		\textbf{\tableofcontents[currentsection, currentsubsection]} %
	\end{frame}%
}

\title{高等数学C}
%\subtitle{A Simple while elegant template}
\author[P.Yu]{余沛}
\mail{peiy\_gzgs@qq.com}
\institute[Guangzhou College of Technology and Business]{Guangzhou College of Technology and Business \\
  广州工商学院}
\date{\today}
\titlegraphic[width=4cm]{logo.png}{}




%%%%%%%%%%%%%%%%%%%%%%%%%%%%%%%%
% ----------- 标题页 ------------
%%%%%%%%%%%%%%%%%%%%%%%%%%%%%%%%



\begin{document}

\maketitle

%%% define code
\defverbatim[colored]\lstI{
	\begin{lstlisting}[language=C++,basicstyle=\ttfamily,keywordstyle=\color{red}]
	int main() {
	// Define variables at the beginning
	// of the block, as in C:
	CStash intStash, stringStash;
	int i;
	char* cp;
	ifstream in;
	string line;
	[...]
	\end{lstlisting}
}
%%%%%%%%%%%%%%%%%%%%%%%%%%%%%%%%
% ----------- FRAME ------------
%%%%%%%%%%%%%%%%%%%%%%%%%%%%%%%%
\section{回顾: 简要回顾微分学}
\begin{frame}{牛顿: 函数值变化的分析}
	\begin{block}{}
		\begin{enumerate}
			\item   考虑函数 $y=f(x)$;
			\item 	对于自变量 $x_0$, 计算对应的因变量 $f(x_0)$;
			\item 	对于任意增量 $\Delta x$, 可以考察自变量 $x_0+\Delta x$, 并计算对应的因变量 $f(x_0+\Delta x)$;
			\item  	考察自变量增量和因变量增量之间的比值关系:
			      \begin{itemize}
				      \item 自变量的增量为 $\Delta x = x+\Delta x -x$;
				      \item 因变量的增量为 $\Delta y = f(x+\Delta x) -f(x)$;
				      \item 因变量的增量和自变量的增量之比为
				            $$\frac{\Delta y}{\Delta x} = \frac{f(x+\Delta x) -f(x)}{\Delta x};$$
			      \end{itemize}
			\item  考察该比值在 $\Delta x$ 趋于零的情况.
		\end{enumerate}
	\end{block}
	{\small	《自然哲学的数学原理》(Sir Issac Newton 著, 赵振江译) 这么解释 :
	\begin{quote}
		"那些最终比, 随着它们量的消失, 实际上不是最终量的比, 而是无线减小的量的比持续靠近的极限, 他们能比任意给定的差更接近. 但在量被减小以至无穷之前, 既不能超过, 也不能到达此极限."
	\end{quote}
	}
\end{frame}


\begin{frame}{牛顿: 为什么我们要讨论"极限"过程}
	\begin{block}{}
		\begin{enumerate}{}
			\item  	考察自变量增量和因变量增量之间的比值关系:
			      \begin{itemize}
				      \item 因变量的增量和自变量的增量之比为
				            $$\frac{\Delta y}{\Delta x} = \frac{f(x+\Delta x) -f(x)}{\Delta x};$$
			      \end{itemize}
			\item  考察该比值在 $\Delta x$ 趋于零的情况.
		\end{enumerate}
	\end{block}
	\begin{quote}
	\end{quote}
	{\small	《自然哲学的数学原理》(Sir Issac Newton (1642 – 1726) 著, 赵振江译) 原文解释到 :
	\begin{quote}
		"那些最终比, 随着它们量的消失, 实际上不是最终量的比, 而是无限减小的量的比持续靠近的极限, 他们能比任意给定的差更接近. 但在量被减小以至无穷之前, 既不能超过, 也不能到达此极限."
	\end{quote}
	}
	\begin{center}
		{\color{red}{这里本身蕴含一组"动态-静态"关系.}}
	\end{center}

	\pause

	这个来处理动静之间的关系工作由 Karl Weierstrass (1815 – 1897) 给出.
\end{frame}

\begin{frame}{魏尔斯特拉斯: 在讨论"极限"时, 我们在讨论什么}
	\begin{block}{}
		\begin{enumerate}{}
			\item  	考察自变量增量和因变量增量之间的比值关系:
			      \begin{itemize}
				      \item 因变量的增量和自变量的增量之比为
				            $$\frac{\Delta y}{\Delta x} = \frac{f(x+\Delta x) -f(x)}{\Delta x};$$
			      \end{itemize}
			\item  考察该比值在 $\Delta x$ 趋于零的情况.
		\end{enumerate}
	\end{block}
	魏尔斯特拉斯的讨论角度:
	\begin{enumerate}
		\item 更一般的情况: 自变量可以不趋于零的情况.
		\item 最简单的情况: 自变量的变化平稳有序, 仅仅考虑因变量的变化趋势.
		\item 先讨论数列极限的问题.
	\end{enumerate}
\end{frame}

\section{回顾: 数列极限}
\begin{frame}{魏尔斯特拉斯: 在讨论"极限"时, 我们在讨论什么}
	数列的极限定义及其来源:
	\begin{block}{定义: 数列极限与收敛数列}
		对于数列 $\{x_n\}$ 和常数 $a$, 如果有: 对于任意的 $\epsilon>0$, 都存在正整数 $N$, 使得对于任意满足 $n>N$ 的 $n$, 不等式
		$$|x_n-a|<\epsilon$$
		都成立, 
		那么称常数 $a$ 是数列 $\{x_n\}$ 的极限, 或者数列 $\{x_n\}$ 收敛, 并且收敛于 $a$, 记为
		\begin{equation*}
			\lim_{n\to\infty}x_n=a,  \quad \text{或} \quad x_n\to a (n\to\infty).
		\end{equation*}
	\end{block}
\end{frame}
\section{回顾: 函数极限}


% Thank you page
\beamertemplateshadingbackground{structure.fg!90}{structure.fg}
\begin{frame}[plain]
	\vfill
	\centering
	{
	\centering \Huge \color{white} Thank you for your attention!\\[10pt]Questions?\\ [10pt]
	}
	\vfill
\end{frame}


\end{document}


