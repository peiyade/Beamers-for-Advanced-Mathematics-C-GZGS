% !TEX encoding = UTF-8 Unicode.

% Based on https://github.com/Miracle0565/BUCT-Beamer-Theme

\documentclass[
10pt,
aspectratio=43,
]{beamer}
\setbeamercovered{transparent=10}
\usetheme[
%  showheader,
%  red,
  purple,
%  gray,
%  graytitle,
  colorblocks,
%  noframetitlerule,
]{Verona}

\usepackage[T1]{fontenc}
\usepackage{tikz}
\usepackage[utf8]{inputenc}
\usepackage{lipsum}
%%%%%%%%%%%%%%%%%%%%%%%%%%%%%%%
% Mac上使用如下命令声明隶书字体, windows也有相关方式, 大家可自行修改
\providecommand{\lishu}{\CJKfamily{zhli}}
%%%%%%%%%%%%%%%%%%%%%%%%%%%%%%%
\usepackage{tikz}
\usetikzlibrary{fadings}
%
%\setbeamertemplate{sections/subsections in toc}[ball]
\usepackage{xeCJK}
\usepackage{listings}
\usepackage{caption}
\usepackage{subfigure}
\usefonttheme{professionalfonts}
\def\mathfamilydefault{\rmdefault}
\usepackage{amsmath}
\usepackage{multirow}
\usepackage{booktabs}
\usepackage{bm}
\setbeamertemplate{section in toc}{\hspace*{1em}\inserttocsectionnumber.~\inserttocsection\par}
\setbeamertemplate{subsection in toc}{\hspace*{2em}\inserttocsectionnumber.\inserttocsubsectionnumber.~\inserttocsubsection\par}
\setbeamerfont{subsection in toc}{size=\small}
\AtBeginSection[]{%
	\begin{frame}%
		\frametitle{Outline}%
		\textbf{\tableofcontents[currentsection]} %
	\end{frame}%
}

\AtBeginSubsection[]{%
	\begin{frame}%
		\frametitle{Outline}%
		\textbf{\tableofcontents[currentsection, currentsubsection]} %
	\end{frame}%
}

\title{高等数学C}
%\subtitle{A Simple while elegant template}
\author[P.Yu]{余沛}
\mail{peiy\_gzgs@qq.com}
\institute[Guangzhou College of Technology and Business]{Guangzhou College of Technology and Business \\
  广州工商学院}
\date{\today}
\titlegraphic[width=4cm]{logo.png}{}




%%%%%%%%%%%%%%%%%%%%%%%%%%%%%%%%
% ----------- 标题页 ------------
%%%%%%%%%%%%%%%%%%%%%%%%%%%%%%%%



\begin{document}

\maketitle

%%% define code
\defverbatim[colored]\lstI{
	\begin{lstlisting}[language=C++,basicstyle=\ttfamily,keywordstyle=\color{red}]
	int main() {
	// Define variables at the beginning
	// of the block, as in C:
	CStash intStash, stringStash;
	int i;
	char* cp;
	ifstream in;
	string line;
	[...]
	\end{lstlisting}
}
%%%%%%%%%%%%%%%%%%%%%%%%%%%%%%%%
% ----------- FRAME ------------
%%%%%%%%%%%%%%%%%%%%%%%%%%%%%%%%
\section{导数的概念与求导法}

\subsection{导数的定义与微分定义}
\begin{frame}{导数的定义}
	\begin{block}{定义:函数导数}
		设 $f$, $\delta>0$ $(x_0-\delta,x_0+\delta)$, (或 $(x_0-\delta,x_0], [x_0,x_0+\delta)$) 上有定义, 对于 $0<\Delta x<\delta$, 如果极限
		\begin{equation*}
			\begin{array}[c]{c}
				\displaystyle\lim_{\Delta x\to 0 } \frac{f(x_0+\Delta x)-f(x_0)}{\Delta x},\bigskip \\
				\displaystyle\left(\text{或}\,\,\lim_{\Delta x\to0}\frac{f(x)-f(x-\Delta x)}{\Delta x}, \,\, \lim_{\Delta x\to0}\frac{f(x+\Delta x)-f(x)}{\Delta x} \right)
			\end{array}
		\end{equation*}
		存在, 则称 $f$ 在 $x_0$ 处可导 (或左导数存在, 右导数存在), 则称该极限值为 $f$ 在 $x_0$ 处的导数或微商 (或 左导数, 右导数), 记为 $f'(x_0)$ (或 $f'_-(x_0)$, $f'_+(x_0)$).
	\end{block}
	若左右导数都存在且相等, 那么导数存在, 并与前两者的值相等.
\end{frame}

\begin{frame}{求导流程与微分概念}
	\begin{block}{求导流程: 差分 $\Delta y$, $\Delta x$ 的比较}
		\begin{enumerate}
			\item 	给出 $\Delta x$; 算出 $\Delta y$; 求增量,
			\item  	求增量比 $\frac{\Delta y}{\Delta x}$; 求差商,
			\item  	求极限.
		\end{enumerate}
	\end{block}

	\begin{block}{微分概念: 差分运算的形式化表示}
		设函数 $y=f(x)$, 在 $x_0$ 的某个邻域内有定义, 自变量在 $x_0$ 处取得增量 $\Delta x$, 因变量在 $f(x_0)$ 处取得增量 $\Delta y=f\left(x_0+\Delta x\right)-f\left(x_0\right)$,
		$$
			\Delta y=A \Delta x+o(\Delta x),
		$$
		其中 $\mathrm{A}$ 是与 $x_0$ 有关而与 $\Delta x$ 无关的常数, $o(\Delta x)$ 是比 $\Delta x$ 高阶的无穷小量. 那么称函数 $y=f(x)$ 在点 $x_0$ {\bf 可微}, $A \Delta x$ 称为微分, 即 $\mathrm{d} y=A \Delta x$, 由于 $\Delta x = \Delta x$, 记为
		$$
			\mathrm{d}y = A \mathrm{d}x.
		$$
	\end{block}
\end{frame}

\subsection{求导计算法则与微分计算法则}

\begin{frame}
	\frametitle{导数四则运算法则和微分计算法则}

	\begin{columns}[t]
		\column{0.5\textwidth}
		\begin{exampleblock}{微分四则运算法则}
			\begin{itemize}
				\item $\mathrm{d}(u \pm v) = \mathrm{d}u \pm \mathrm{d}v$,
				\item $\mathrm{d}(Cu) = C \mathrm{d}u$,
				\item $\mathrm{d}(uv) = v \mathrm{d}u + u \mathrm{d}v$,
				\item $\displaystyle \mathrm{d}\left(\frac{u}{v}\right) = \frac{v \mathrm{d}u - u \mathrm{d}v}{v^2}$ ($v \neq 0$).
			\end{itemize}
		\end{exampleblock}
		\begin{exampleblock}{反函数求导法则}
			设 $y = f(x)$ 是一个可导函数, 且 $f'(x) \neq 0$, 则有
			\[
				\left(f^{-1}(y)\right)' = \frac{1}{f'(f^{-1}(y))}=\frac{1}{f'(x)}.
			\]
		\end{exampleblock}
		\column{0.5\textwidth}
		\begin{exampleblock}{导数计算法则}
			\begin{itemize}
				\item $(f \pm g)' = f' \pm g'$,
				\item $(Cf)' = Cf'$,
				\item $(fg)' = f'g + fg'$,
				\item $\displaystyle \left(\frac{f}{g}\right)' = \frac{f'g - fg'}{g^2}$ ($g \neq 0$).
			\end{itemize}
		\end{exampleblock}
		\begin{exampleblock}{链式法则}
			设 $y = f(g(x))$, 其中 $f(u)$ 和 $g(x)$ 都是可导函数, 则有
			\[
				\frac{\mathrm{d}y}{\mathrm{d}x} = \frac{\mathrm{d}y}{\mathrm{d}u} \cdot \frac{du}{dx} = f'(g(x)) \cdot g'(x).
			\]
		\end{exampleblock}

	\end{columns}

\end{frame}
\begin{frame}{算例}
	提供一些常见导数的算例:
	\begin{block}{}
		{\small 基本结果:
			\begin{enumerate}
				\item $f(x)=C$,  $f'(x) =0$;
				\item $f(x)=x^n, (n\in\mathbb{N}^+)$,  $f'(x) = n x^{n-1}$;
				\item $f(x)=\sin(x)$,  $f'(x)=\cos(x)$; \,\, $g(x)=\cos(x)$,  $g'(x)=-\sin(x)$;
				\item $f(x)=\log_a  x$,\, $f'(x)= \displaystyle\frac{1}{\log a \cdot x}$;
			\end{enumerate}
			利用反函数和复合函数法则后:
			\begin{enumerate}
				\item $f(x) = e^x$, $f'(x) = e^x$; $f(x)=a^x(a>0)$, $f'(x)=\log a \cdot a^x$
				\item $f(x) = \arcsin(x)$, $f'(x) = \displaystyle\frac{1}{\sqrt{1-x^2}}$;
				\item $f(x) = \arccos(x)$, $f'(x) = \displaystyle-\frac{1}{\sqrt{1-x^2}}$;
				\item $f(x) = x^a,\,\,(a\neq0)$, $\frac{f'(x)}{f(x)}=\left(\log(f(x))\right)'=a\cdot \frac 1x$, $f'(x)=a\cdot\frac1x\cdot f(x)=ax^{a-1}$.
			\end{enumerate}
		}
	\end{block}
\end{frame}


\begin{frame}
	\frametitle{计算三角函数的导数}
	\begin{columns}
		\column{0.5\textwidth}
		\centering
		计算以下函数的导数:
		\begin{align*}
			f(x) & = \tan x, \\
			g(x) & = \cot x, \\
			h(x) & = \sec x, \\
			k(x) & = \csc x.
		\end{align*}
		\pause
		使用导数的定义,我们有:
		\begin{align*}
			f'(x) & = \sec^2 x,       \\
			g'(x) & = -\csc^2 x,      \\
			h'(x) & = \sec x \tan x,  \\
			k'(x) & = -\csc x \cot x.
		\end{align*}

		\column{0.5\textwidth}
		\centering
		\pause
		计算以下函数的导数:
		\begin{align*}
			p(x) & = \arcsin x,       \\
			q(x) & = \arccos x,       \\
			r(x) & = \arctan x,       \\
			s(x) & = \text{arccot} x.
		\end{align*}
		\pause
		使用导数的定义,我们有:
		\begin{align*}
			p'(x) & = \frac{1}{\sqrt{1-x^2}},  \\
			q'(x) & = -\frac{1}{\sqrt{1-x^2}}, \\
			r'(x) & = \frac{1}{1+x^2},         \\
			s'(x) & = -\frac{1}{1+x^2}.
		\end{align*}
	\end{columns}
\end{frame}

\begin{frame}
	\frametitle{计算函数的导数}
	\begin{columns}
		\column{0.5\textwidth}
		\centering
		计算以下函数的导数:
		\begin{align*}
			f(x) & = \sin x + x^2,                  \\
			g(x) & = x^3 - \cos x + \ln x + \sin 5, \\
			h(x) & = \sin 2x.
		\end{align*}

		\pause
		使用导数的链式法则,我们有:
		\begin{align*}
			f'(x) & = \cos x + 2x,                 \\
			g'(x) & = 3x^2 + \sin x + \frac{1}{x}, \\
			h'(x) & = 2\cos 2x.
		\end{align*}

		\column{0.5\textwidth}
		\centering
		计算以下函数的导数:
		\begin{align*}
			p(x) & = (x^2 + 1)(\sin x - 1),             \\
			q(x) & = \frac{x+1}{x-1},                   \\
			r(x) & = 3x^5 + x\tan x + \frac{\cos x}{x}.
		\end{align*}

		\pause
		使用导数的乘积法则和商法则,我们有:
		\begin{align*}
			p'(x) & = (2x)(\sin x - 1) + (x^2 + 1)(\cos x),           \\
			q'(x) & = -\frac{2}{(x-1)^2},                             \\
			r'(x) & = 15x^4 + \tan x + x\sec^2 x - \frac{\sin x}{x^2} \\
			      & \,\,\,\,\,\,\,-\frac{\cos (x)}{x^2}.
		\end{align*}
	\end{columns}
\end{frame}

\begin{frame}
	\frametitle{计算复合函数的导数}
	\begin{block}{题目}
		\begin{itemize}
			\item $f(x)=(1-2 x)^7,$
			\item $g(x)=\sin ^2 x,$
			\item $h(x)=\sqrt{\cot \frac{x}{2}},$
			\item $i(x)=\ln \sin \mathrm{e}^x.$
		\end{itemize}
	\end{block}

	\pause

	\begin{exampleblock}{答案}
		\begin{align*}
			f'(x) & = -14(1-2x)^6,                                             \\
			g'(x) & = 2\sin x \cos x,                                          \\
			h'(x) & = -\frac{1}{4}\left(\cot \frac{x}{2}\right)^{-\frac12} \csc^2 \frac{x}{2},  \\
			i'(x) & = \mathrm{e}^x\frac{\cos \mathrm{e}^x}{\sin \mathrm{e}^x}.
		\end{align*}
	\end{exampleblock}

\end{frame}

\begin{frame}
	\frametitle{计算变量的导数}

	\begin{block}{题目}
		\begin{itemize}
  			\item $y = f(\sin x)$,
			\item $y = \log \left(x+\sqrt{x^2+1}\right)$,
			\item $y = \log |x|$.
		\end{itemize}
	\end{block}

	\pause

	\begin{exampleblock}{答案}
		{\small
			\begin{itemize}
				\item   $\displaystyle\frac{dy}{dx} = f'(\sin x) \cdot \cos x,$
                \item   $\displaystyle\frac{dy}{dx} = \frac{1}{\sqrt{x^2+1}},$
				\item   $\displaystyle\frac{dy}{dx} = \frac{1}{x}.$
			\end{itemize}
		}
	\end{exampleblock}
\end{frame}

\begin{frame}
	\frametitle{计算函数的导数}

	\begin{block}{题目}
		\begin{itemize}
			\item $\displaystyle\frac{x}{2} \sqrt{a^2-x^2}+\frac{a^2}{2} \arcsin \frac{x}{a}$,
			\item $\displaystyle\frac{\sqrt{x+1}-\sqrt{x-1}}{\sqrt{x+1}+\sqrt{x-1}}$,
			\item $x^{a^a}+a^{x^a}+a^{a^x}$.
		\end{itemize}
	\end{block}

	\pause

	\begin{exampleblock}{答案}
		\begin{itemize}
			 \item $\displaystyle \sqrt{a^2-x^2},$
			 \item $\displaystyle1-\frac{x}{\sqrt{x^2-1}},$
			 \item $\displaystyle a^a x^{a^a-1}+\ln a \cdot aa^{x^a}  x^{a-1}+a^{a^x+x}\ln^2 a.$
		\end{itemize}
	\end{exampleblock}

\end{frame}

\begin{frame}
	\frametitle{计算函数的导数}

	\begin{block}{题目}
		\begin{itemize}
			\item $y=x^{\sin x},$
			\item $\displaystyle y=\sqrt{\frac{(x-1)(x-2)}{(x-3)(x-4)}}$.
		\end{itemize}
	\end{block}

	\pause

	\begin{exampleblock}{答案}
		\begin{itemize}
			\item $\displaystyle x^{\sin x}\left(\cos x \log x + \frac{\sin x}{x}\right),$
			\item $\displaystyle  \frac{1}{2} \sqrt{\frac{(x-1)(x-2)}{(x-3)(x-4)}\left[\frac{1}{x-1}+\frac{1}{x-2}-\frac{1}{x-3}-\frac{1}{x-4}\right]}.$
		\end{itemize}
	\end{exampleblock}

\end{frame}

\begin{frame}
	\frametitle{计算抽象函数的导数}

	\begin{block}{题目}
		设 $f(x)=\varphi(a+b x)-\varphi(a-b x)$, 其中的 $\varphi(x)$ 在 $x=a$ 处可导, 求 $f^{\prime}(0)$.
	\end{block}

	\pause

	\begin{exampleblock}{答案}
		根据链式法则, 我们有
		\[
			f^{\prime}(x) = \varphi^{\prime}(a+b x) \cdot (b) - \varphi^{\prime}(a-b x) \cdot (-b) = b(\varphi^{\prime}(a+b x) + \varphi^{\prime}(a-b x)).
		\]
		因此, $f^{\prime}(0) = b(\varphi^{\prime}(a) + \varphi^{\prime}(a)) = 2b\varphi^{\prime}(a)$.
	\end{exampleblock}

\end{frame}

\begin{frame}
	\frametitle{计算抽象函数的导数}

	\begin{block}{题目}
		设 $f(x)$ 在 $(0,+\infty)$ 的上有定义,且 $f^{\prime}(1)=a(\neq 0)$,又 $\forall x, y \in(0,+\infty)$,有 $f(x y)=f(x)+f(y)$,求 $f(x)$。
	\end{block}

	\pause

	\begin{exampleblock}{}
		首先, 有 \[
			f(x)=f(x)+f(1),\,\,\,\, f(1)=0.
		\]
		然后 按照导数的定义, $f(x)=f'(1)\log x$, 因为
		\begin{align*}
			f'(x) & =\lim_{\Delta x\to0}\frac{f(x+\Delta x)-f(x)}{\Delta x}                                    \\
			      & =\lim_{\Delta x\to0}\frac{f\left[x\left(1+\frac{\Delta x}{x}\right)\right]-f(x)}{\Delta x} \\
			      & =\lim_{\Delta x\to0}\frac{f(x)+f\left[1+\frac{\Delta x}{x}\right]-f(x)}{\Delta x}          \\
			      & =f'(1)\cdot \frac{1}{x}.
		\end{align*}
	\end{exampleblock}
\end{frame}

% Thank you page
\beamertemplateshadingbackground{structure.fg!90}{structure.fg}
\begin{frame}[plain]
	\vfill
	\centering
	{
	\centering \Huge \color{white} Thank you for your attention!\\[10pt]Questions?\\ [10pt] Homework: Page123: 46, 47, 57.
	}
	\vfill
\end{frame}


\end{document}


