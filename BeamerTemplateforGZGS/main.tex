% !TEX encoding = UTF-8 Unicode

% Based on https: //github.com/Miracle0565/BUCT-Beamer-Theme

\documentclass[
10pt, 
aspectratio=43, 
]{beamer}
\setbeamercovered{transparent=10}
\usetheme[
%  showheader, 
%  red, 
  purple, 
%  gray, 
%  graytitle, 
  colorblocks, 
%  noframetitlerule, 
]{Verona}

\usepackage[T1]{fontenc}
\usepackage{tikz}
\usepackage[utf8]{inputenc}
\usepackage{lipsum}
%%%%%%%%%%%%%%%%%%%%%%%%%%%%%%%
% Mac上使用如下命令声明隶书字体, windows也有相关方式, 大家可自行修改
\providecommand{\lishu}{\CJKfamily{zhli}}
%%%%%%%%%%%%%%%%%%%%%%%%%%%%%%%
\usepackage{tikz}
\usetikzlibrary{fadings}
%
%\setbeamertemplate{sections/subsections in toc}[ball]
\usepackage{xeCJK}
\usepackage{listings}
\usepackage{caption}
\usepackage{subfigure}
\usefonttheme{professionalfonts}
\def\mathfamilydefault{\rmdefault}
\usepackage{amsmath}
\usepackage{multirow}
\usepackage{booktabs}
\usepackage{bm}
\setbeamertemplate{section in toc}{\hspace*{1em}\inserttocsectionnumber.~\inserttocsection\par}
\setbeamertemplate{subsection in toc}{\hspace*{2em}\inserttocsectionnumber.\inserttocsubsectionnumber.~\inserttocsubsection\par}
\setbeamerfont{subsection in toc}{size=\small}
\AtBeginSection[]{%
	\begin{frame}%
		\frametitle{Outline}%
		\textbf{\tableofcontents[currentsection]} %
	\end{frame}%
}

\AtBeginSubsection[]{%
	\begin{frame}%
		\frametitle{Outline}%
		\textbf{\tableofcontents[currentsection,  currentsubsection]} %
	\end{frame}%
}

\title{高等数学C}
%\subtitle{A Simple while elegant template}
\author[P.Yu]{余沛}
\mail{peiy\_gzgs@qq.com}
\institute[Guangzhou College of Technology and Business]{Guangzhou College of Technology and Business \\
  广州工商学院}
\date{\today}
\titlegraphic[width=4cm]{logo.png}{}




%%%%%%%%%%%%%%%%%%%%%%%%%%%%%%%%
% ----------- 标题页 ------------
%%%%%%%%%%%%%%%%%%%%%%%%%%%%%%%%



\begin{document}

\maketitle

%%% define code
\defverbatim[colored]\lstI{
	\begin{lstlisting}[language=C++, basicstyle=\ttfamily, keywordstyle=\color{red}]
	int main() {
	// Define variables at the beginning
	// of the block,  as in C: 
	CStash intStash,  stringStash;
	int i;
	char* cp;
	ifstream in;
	string line;
	[...]
	\end{lstlisting}
}
%%%%%%%%%%%%%%%%%%%%%%%%%%%%%%%%
% ----------- FRAME ------------
%%%%%%%%%%%%%%%%%%%%%%%%%%%%%%%%

\section{数列极限}
\subsection{回顾: 等价无穷小}
\begin{frame}
\begin{block}{无穷小量定义}设函数 $f(x)$ 在 $x=a$ 处有定义, 如果对于任意给定的正数 $\varepsilon$, 存在正数 $\delta$, 使得当 $0 < |x-a| < \delta$ 时, 有 $|f(x)| < \varepsilon$, 则称函数 $f(x)$ 在 $x=a$ 处为无穷小量. 
\end{block}
\begin{itemize}
\item<2-> \textbf{幂函数}: 当 $x$ 趋近于 $0$ 时, 函数 $f(x) = x^n$ 是无穷小量, 其中 $n$ 是正整数. \\
\item<3-> \textbf{指数函数}: 当 $x$ 趋近于 $0$ 时, 函数 $f(x) = e^x - 1$ 是无穷小量. \\
\item<4-> \textbf{三角函数}: 当 $x$ 趋近于 $0$ 时, 函数 $f(x) = \sin(x)$ 和 $f(x) = \tan(x)$ 是无穷小量. \\
\item<5-> \textbf{对数函数}: 当 $x$ 趋近于 $1$ 时, 函数 $f(x) = \log(x)$ 是无穷小量. 
\end{itemize}

\end{frame}
\begin{frame}
\frametitle{函数的无穷小量的比较}
当 $x\to0$ 时,  $x,  x^2,  2 x$ 都是无穷小量,  比较它们趋向于 0 的速度, 
\begin{itemize}
\item $\lim _{x \rightarrow 0} \frac{x^2}{x}=0,  x^2$ 比 $x$ 要快得多; 称 $x^2$ 是比 $x$ 高阶无穷小;
\item $\lim _{x \rightarrow 0} \frac{2 x}{x}=2, 2 x$ 与 $x$ 大致相同; 称 $2 x$ 与 $x$ 是同阶无穷小;
\item $\lim _{x \rightarrow 0} \frac{x}{x^2}=\infty,  x$ 比 $x^2$ 要慢得多. 称 $x$ 是比 $x^2$ 较低阶无穷小. 
\end{itemize}
\pause
\begin{block}{等价无穷小}
特别地, 如果有
$$
\lim \frac{\alpha}{\beta} = 1, 
$$
则称 $\beta$ 与 $\alpha$ 是等价无穷小量, 记作 $\alpha ~ \beta$.
\end{block}
\end{frame}

\subsection{常用的等价无穷小}

\begin{frame}
\frametitle{等价无穷小替换定理与常用的等价无穷小}

\begin{block}{定理(无穷小等价替换定理)}
如果 $\alpha_{1 \hookleftarrow} \beta_1,  \alpha_{2 \hookleftarrow} \beta_2$ 且极限 $\lim \frac{\beta_1}{\beta_2}$ 存在,  则
$$
\lim \frac{\alpha_1}{\alpha_2}=\lim \frac{\beta_1}{\beta_2}
$$
\end{block}
\pause
\begin{proof}
$\quad \lim \frac{\alpha_1}{\alpha_2}=\lim \frac{\alpha_1}{\beta_1} \cdot \frac{\beta_2}{\alpha_2} \cdot \frac{\beta_1}{\beta_2}=\lim \frac{\beta_1}{\beta_2}$.

\end{proof}
\pause
常用的等价无穷小: 
当 $x \rightarrow 0$时
$$
 x \sim \sin x \sim \tan x \sim \arcsin x \sim \arctan x \sim \ln (x+1) \sim e^x-1, 
$$
$$
1-\cos x \sim \frac{1}{2} x^2.
$$

\end{frame}

\begin{frame}
\frametitle{等价无穷小替换定理与常用的等价无穷小}
\begin{exampleblock}{例1}
 $ \quad \lim _{x \rightarrow 0} \frac{\tan m x}{\sin n x}=\lim _{x \rightarrow 0} \frac{m x}{n x}=\frac{m}{n} \quad(\tan m x \sim m x,  \sin n x\sim n x)$
\end{exampleblock}
\pause
\begin{exampleblock}{例2}
 $ \lim _{x \rightarrow 0} \frac{\sin ^2 x}{x^3+2 x^2}=\lim _{x \rightarrow 0} \frac{x^2}{x^3+2 x^2}=\lim _{x \rightarrow 0} \frac{1}{x+2}=\frac{1}{2}\left(\sin ^2 x \sim x^2\right)$
\end{exampleblock}
\end{frame}
\begin{frame}
\frametitle{等价无穷小替换定理与常用的等价无穷小}
\begin{exampleblock}{例3}
\begin{equation*}
\begin{aligned}
&\lim _{x \rightarrow 1} \frac{\ln x}{x^2-1}=\lim _{x \rightarrow 1} \frac{\ln [1+(x-1)]}{(x+1)(x-1)}=\lim _{x \rightarrow 1} \frac{x-1}{(x+1)(x-1)}=\frac{1}{2}, \\
&(\ln [(x-1)+1] \sim x-1)
\end{aligned}
\end{equation*}
\end{exampleblock}
\pause
\begin{exampleblock}{例4} \begin{equation*}
\begin{aligned}
& \lim _{x \rightarrow 0} \frac{\left(1+x^2\right)^{\frac{1}{3}}-1}{\cos x-1}=\lim _{x \rightarrow 0} \frac{\frac{1}{3} x^2}{-\frac{1}{2} x^2}= -\frac{2}{3}\\
& \left(\left(1+x^2\right)^{\frac{1}{3}}-1 \sim \frac{1}{3} x^2,  \cos x-1 \sim-\frac{1}{2} x^2\right)
\end{aligned}
\end{equation*}
\end{exampleblock}
\end{frame}

\section{函数的连续性}
\subsection{增量与差分}

\begin{frame}
\frametitle{增量,  差分与函数连续性}
\begin{block}{增量}
\begin{itemize}
    \item $\Delta x:  \Delta x=x_1-x_0$,  即自变量的增量 $=$ 终值 - 初值,  或终值 $x_1=x_0+\Delta x$;
    \item $\Delta y:  \Delta y=y_1-y_0$,  即因变量的增量 $=$ 终值 - 初值,  或终值 $y_1=y_0+\Delta y$;
\end{itemize}
$\Delta x$,  $\Delta y$ 也被称为关于自变量, 因变量的差分.
\end{block}
\end{frame}

\begin{frame}
\frametitle{函数的连续性概念}
\begin{block}{函数的连续性}
 设函数 $y=f(x)$ 在点 $x_0$ 的某个邻域内有定义,  如果当自变量的增量 $\Delta x \rightarrow 0$ 时,  相应因变量的增量 $\Delta y \rightarrow 0$,  则称 $f(x)$ 在点 $x_0$ 处\textbf{连续}, \\
 记作 $$\lim _{\Delta x \rightarrow 0} \Delta y=0$$. 
\end{block}
\pause
初等函数在定义域范围内都是连续的. 
\pause
\begin{exampleblock}{不连续}
否则,  称 $f(x)$ 在点 $x_0$ 处不连续. 
\end{exampleblock}
\pause
初等函数在不在定义域的区间上都是不连续的. 
\end{frame}



\subsection{分段函数的连续性}
\subsection{间断点概念}
\subsection{连续函数的逐点性质}
\subsection{闭区间上连续函数的性质}
\subsubsection{有界性}
\subsubsection{介值性与零点存在定理}




% Thank you page
\beamertemplateshadingbackground{structure.fg!90}{structure.fg}
\begin{frame}[plain]
	\vfill
	\centering
	{
		\centering \Huge \color{white} Thank you for your attention!\\[10pt]Questions?
	}
	\vfill
\end{frame}


\end{document}


