% !TEX encoding = UTF-8 Unicode.

% Based on https://github.com/Miracle0565/BUCT-Beamer-Theme

\documentclass[
10pt,
aspectratio=43,
]{beamer}
\setbeamercovered{transparent=10}
\usetheme[
%  showheader,
%  red,
  purple,
%  gray,
%  graytitle,
  colorblocks,
%  noframetitlerule,
]{Verona}

\usepackage[T1]{fontenc}
\usepackage{tikz}
\usepackage[utf8]{inputenc}
\usepackage{lipsum}
\usepackage{pgfplots}
%%%%%%%%%%%%%%%%%%%%%%%%%%%%%%%
% Mac上使用如下命令声明隶书字体, windows也有相关方式, 大家可自行修改
\providecommand{\lishu}{\CJKfamily{zhli}}
%%%%%%%%%%%%%%%%%%%%%%%%%%%%%%%
\usepackage{tikz}
\usetikzlibrary{fadings}
%
%\setbeamertemplate{sections/subsections in toc}[ball]
\usepackage{xeCJK}
\usepackage{listings}
\usepackage{caption}
\usepackage{subfigure}
\usefonttheme{professionalfonts}
\def\mathfamilydefault{\rmdefault}
\usepackage{amsmath}
\usepackage{multirow}
\usepackage{booktabs}
\usepackage{bm}
\usepackage{mathtools}
\usepackage[T1]{fontenc}
\setbeamertemplate{section in toc}{\hspace*{1em}\inserttocsectionnumber.~\inserttocsection\par}
\setbeamertemplate{subsection in toc}{\hspace*{2em}\inserttocsectionnumber.\inserttocsubsectionnumber.~\inserttocsubsection\par}
\setbeamerfont{subsection in toc}{size=\small}
\AtBeginSection[]{%
	\begin{frame}%
		\frametitle{Outline}%
		\textbf{\tableofcontents[currentsection]} %
	\end{frame}%
}

\AtBeginSubsection[]{%
	\begin{frame}%
		\frametitle{Outline}%
		\textbf{\tableofcontents[currentsection, currentsubsection]} %
	\end{frame}%
}

\pgfplotsset{
    integral segments/.code={\pgfmathsetmacro\integralsegments{#1}},
    integral segments=3,
    integral/.style args={#1:#2}{
        ybar interval,
        domain=#1+((#2-#1)/\integralsegments)/2:#2+((#2-#1)/\integralsegments)/2,
        samples=\integralsegments+1,
        x filter/.code=\pgfmathparse{\pgfmathresult-((#2-#1)/\integralsegments)/2}
    }
}


\title{高等数学C}
%\subtitle{A Simple while elegant template}
\author[P.Yu]{余沛}
\mail{peiy\_gzgs@qq.com}
\institute[Guangzhou College of Technology and Business]{Guangzhou College of Technology and Business \\
  广州工商学院}
\date{\today}
\titlegraphic[width=4cm]{logo.png}{}




%%%%%%%%%%%%%%%%%%%%%%%%%%%%%%%%
% ----------- 标题页 ------------
%%%%%%%%%%%%%%%%%%%%%%%%%%%%%%%%



\begin{document}

\maketitle

%%% define code
\defverbatim[colored]\lstI{
	\begin{lstlisting}[language=C++,basicstyle=\ttfamily,keywordstyle=\color{red}]
	int main() {
	// Define variables at the beginning
	// of the block, as in C:
	CStash intStash, stringStash;
	int i;
	char* cp;
	ifstream in;
	string line;
	[...]
	\end{lstlisting}
}
%%%%%%%%%%%%%%%%%%%%%%%%%%%%%%%%
% ----------- FRAME ------------
%%%%%%%%%%%%%%%%%%%%%%%%%%%%%%%%
\section{积分法}
\begin{frame}
	\frametitle{回顾:定积分}
	\begin{block}{Riemann 积分的定义}
		设$f(x)$在区间$[a,b]$上有定义, $\mathrm{P}:a=x_0<x_1<x_2<\ldots<x_n=b$.
		在每段$[x_{i-1},x_i]$上 {\bf 任意}选取点$\xi_i$, 得到近似和
		$$
			\sum_{i=1}^n f(\xi_i)(x_i-x_{i-1}).
		$$
		如果对于$\mathrm{P}$的精细化
		我们就称$f(x)$在区间$[a,b]$上是 {\bf Riemann 可积}的, 并将这个极限值称为$f(x)$在$[a,b]$上的 {\bf Riemann 积分}, 记为
		\[
			\lim_{\lambda\to 0} \sum_{i=1}^n f(\xi_i)(x_i-x_{i-1}) := \int_a^b f(x)\mathrm{~d} x.
		\]
	\end{block}

\end{frame}
\begin{frame}
	\frametitle{归纳:定积分的性质}
	\everymath{\displaystyle}
	\begin{enumerate}
		\item 线性性: $\int_a^b[k_1f_1(x) + k_2f_2(x)] \mathrm{d} x=\int_a^b k_1f_1(x) \mathrm{~d} x+\int_a^b k_2f_2(x) \mathrm{~d}x$;
		\item 区间可加性: $\int_a^b f(x) \mathrm{~d} x=\int_a^c f(x) \mathrm{~d} x+\int_c^b f(x) \mathrm{~d} x$;
		\item 保号性: $f(x) \geq 0, x\in[a,b]$, 则 $\int_a^b f(x) \mathrm{~d} x\geq 0$;
		\item 上下界估计: $(b-a)\inf_{x\in[a,b]}f(x)=\int_a^{b} f(x) \mathrm{~d} x=(b-a)\sup_{x\in[a,b]}f(x)$;
		\item 中值定理: 设 $f(x)$ 在 $[a, b]$ 上连续, 则存在 $\xi\in[a, b]$, 使
		      $$
			      \int_a^b f(x) \mathrm{~d} x=f(\xi)(b-a).
		      $$
	\end{enumerate}
	\vspace{0.5cm}
\end{frame}

\begin{frame}
	\frametitle{回顾: Newton-Leibniz公式}

	\begin{theorem}[Newton-Leibniz公式]
		如果函数 $F(x)$ 是连续函数 $f(x)$ 在区间 $[a, b]$ 上的一个原函数, 则
		$$
			\int_a^b f(x) \mathrm{~d} x=F(b)-F(a) .
		$$
	\end{theorem}
	\vspace{1cm}
	\pause
	\begin{quote}
		原函数 $F(x)$ 满足
		$$
			F (x)+C=\int f(x)\mathrm{~d}x.
		$$
	\end{quote}
\end{frame}

\begin{frame}
	\frametitle{练习}
	\everymath{\displaystyle}
	\begin{block}{计算: $\int_{-1}^{\sqrt{3}} \frac{\mathrm{d} x}{1+x^2}$}
		\pause
		由于 $\int\frac{\mathrm{~d}x}{1+x^2}=\arctan x +C$, $\arctan x$ 为 $\arctan x$ 的一个原函数, 所以,
		\pause
		$$
			\int_{-1}^{\sqrt{3}} \frac{\mathrm{~d} x}{1+x^2}=\left.\arctan x\right|_{-1}^{\sqrt{3}}=\arctan \sqrt{3}-\arctan (-1)=\frac{\pi}{3}-\left(-\frac{\pi}{4}\right)=\frac{7}{12} \pi.
		$$
	\end{block}
	\pause
	\begin{block}{计算正弦曲线 $y=\sin x$ 在 $[0, 2\pi]$ 上与 $x$ 轴所围成的平面图形的面积}
		\pause
		这图形是曲边梯形的一个特例, 它的面积\\\vspace{0.2cm}
		$
			A=\int_0^{2\pi} |\sin x| \mathrm{~d} x=\int_0^{\pi} \sin x \mathrm{~d} x +\int_{\pi}^{2\pi} -\sin x \mathrm{~d} x.
		$
		\pause

		\vspace{0.2cm}
		$
			A= \left.-\cos x\right|_0^\pi+\left.\cos x\right|_\pi^{2\pi}=4.
		$
	\end{block}

\end{frame}

\subsection{换元积分}
\begin{frame}
	\frametitle{换元积分法}
	\everymath{\displaystyle}
	\begin{block}{第一类换元积分法}
		设 $f(u)$ 具有原函数 $F(u)$, 即 $F^{\prime}(u)=f(u)$, 如果 $u$ 是中间变量: $u=\varphi(x)$, 且设 $\varphi(x)$ 可微, 那么根据复合函数微分法, 有
		$$
			\mathrm{~d} F(\varphi(x))=f(\varphi(x)) \varphi^{\prime}(x) \mathrm{~d} x,
		$$ 从而根据不定积分的定义就得到
		$$
			\int f(\varphi(x)) \varphi^{\prime}(x) \mathrm{~d} x=F(\varphi(x))+C=\left.\left(\int f(u) \mathrm{~d} u\right)\right|_{u=\varphi(x)}.
		$$
	\end{block}
	\pause
	\begin{block}{第二类换元积分法}
		设 $x=\psi(t)$ 是单调的可导函数, 并且 $\psi^{\prime}(t) \neq 0$, 又设 $f(\psi(t)) \psi^{\prime}(t)$ 具有原函数, 则有换元公式 $\int f(x) \mathrm{~d} x=\left.\left(\int f(\psi(t)) \psi^{\prime}(t) \mathrm{~d} t\right)\right|_{t=\psi^{-1}(x)}$, 其中 $\psi^{-1}(t)$ 是 $x=\psi(t)$ 的反函数。
	\end{block}
\end{frame}

\begin{frame}
	\frametitle{定积分的换元积分法}
	\everymath{\displaystyle}
	\begin{theorem}[定积分的换元积分法]
		假设函数 $f(x)$ 在区间 $[a,b]$ 上连续, 函数 $x=\varphi(t)$ 满足条件:
		\begin{enumerate}
			\item $\varphi(\alpha)=a, \varphi(\beta)=b$;
			\item $\varphi(t)$ 在 $[\alpha, \beta]$(或 $[\beta, \alpha]$) 上具有连续导数, 且其值域不越出 $[a, b]$;
		\end{enumerate}
		则有
		$$
			\int_a^b f(x) \mathrm{~d} x=\int_\alpha^\beta f\left[\varphi(t)\right] \varphi^{\prime}(t) \mathrm{~d} t .
		$$
	\end{theorem}
\end{frame}

\begin{frame}
	\frametitle{定积分的换元积分法}
	\everymath{\displaystyle}
	\begin{block}{计算 $\int_0^a \sqrt{a^2-x^2} \mathrm{~d} x\quad(a>0)$}
		提示: $\sqrt{a^2-x^2}=\sqrt{a^2-a^2 \sin ^2 t}=a \cos t, \mathrm{~d}=a \cos t$. 当 $x=0$ 时 $t=0$, 当 $x=a$ 时 $t=\frac{\pi}{2}$.
	\end{block}
	\pause
	\begin{exampleblock}{}
		$$
			\begin{aligned}
				 & \int_0^a \sqrt{a^2-x^2} \mathrm{~d}x \stackrel{\text { 令 } x=a \sin t}{=} \int_0^{\frac{\pi}{2}} a \cos t \cdot a \cos t \mathrm{~d}t \\
				 & =a^2 \int_0^{\frac{\pi}{2}} \cos ^2 t \mathrm{~d}t=\frac{a^2}{2} \int_0^{\frac{\pi}{2}}(1+\cos 2 t) \mathrm{~d}t                      \\
				 & =\frac{a^2}{2}\left.\left(t+\frac{1}{2} \sin 2 t\right)\right|_0^{\frac{\pi}{2}}=\frac{1}{4} \pi a^2 .
			\end{aligned}
		$$
	\end{exampleblock}
	\pause
	这也是半径为 $a$ 的四分之一圆的面积的证明.
\end{frame}

\begin{frame}
	\frametitle{定积分的换元积分法}
	\everymath{\displaystyle}
	\begin{block}{计算 $\int_0^{\frac{\pi}{2}} \cos ^5 x \sin x \mathrm{~d} x$}
		提示: 当 $x=0$ 时 $t=1$, 当 $x=\frac{\pi}{2}$ 时 $t=0$. 或 $\int_0^{\frac{\pi}{2}} \cos ^5 x \sin x \mathrm{~d} x=-\int_0^{\frac{\pi}{2}} \cos ^5 x \mathrm{~d} \cos x$
		$$
			=-\left.\left(\frac{1}{6} \cos ^6 x\right)\right|_0 ^{\frac{\pi}{2}}=-\frac{1}{6} \cos ^6 \frac{\pi}{2}+\frac{1}{6} \cos ^6 0=\frac{1}{6} .
		$$
	\end{block}
	\pause
	\begin{exampleblock}{}
		令 $t=\cos x$, 则
		$$
			\begin{aligned}
				 & \int_0^{\frac{\pi}{2}} \cos ^5 x \sin x \mathrm{~d} x=-\int_0^{\frac{\pi}{2}} \cos ^5 x \mathrm{~d}\cos x                               \\
				 & \stackrel{\text { 令 } \cos x=t}{=}-\int_1^0 t^5 \mathrm{~d} t=\int_0^1 t^5 \mathrm{~d} t=\left(\frac{1}{6} t^6\right)_0^1=\frac{1}{6} .
			\end{aligned}
		$$
	\end{exampleblock}
\end{frame}

\begin{frame}
	\frametitle{定积分的换元积分法}
	\everymath{\displaystyle}
	\begin{block}{计算 $\int_0^\pi \sqrt{\sin ^3 x-\sin ^5 x} \mathrm{~d} x$}
		提示: $\sqrt{\sin ^3 x-\sin ^5 x}=\sqrt{\sin ^3 x\left(1-\sin ^2 x\right)}=\sin ^{\frac{3}{2}} x|\cos x|$.在 $\left[0, \frac{\pi}{2}\right]$ 上 $|\cos x|=\cos x$, 在 $\left[\frac{\pi}{2}, \pi\right]$ 上 $|\cos x|=-\cos x$.
	\end{block}
	\pause
	\begin{exampleblock}{}
		$$
			\begin{aligned}
				 & \int_0^\pi \sqrt{\sin ^3 x-\sin ^5 x} \mathrm{~d} x=\int_0^\pi \sin ^{\frac{3}{2}} x|\cos x| \mathrm{~d} x                                                                               \\
				 & =\int_0^{\frac{\pi}{2}} \sin ^{\frac{3}{2}} x \cos x \mathrm{~d} x-\int_{\frac{\pi}{2}}^\pi \sin ^{\frac{3}{2}} x \cos x \mathrm{~d} x                                                   \\
				 & =\int_0^{\frac{\pi}{2}} \sin ^{\frac{3}{2}} x d \sin x-\int_{\frac{\pi}{2}}^\pi \sin ^{\frac{3}{2}} x d \sin x                                                                           \\
				 & =\left.\frac{2}{5} \sin ^{\frac{5}{2}} x\right|_0^{\frac{\pi}{2}}-\left.\frac{2}{5} \sin ^{\frac{5}{2}} x\right|_{\frac{\pi}{2}}^\pi=\frac{2}{5}-\left(-\frac{2}{5}\right)=\frac{4}{5} .
			\end{aligned}
		$$
	\end{exampleblock}
\end{frame}

\begin{frame}
	\frametitle{定积分的换元积分法}
	\everymath{\displaystyle}
	\begin{block}{计算 $\int_0^4 \frac{x+2}{\sqrt{2 x+1}} \mathrm{~d} x$}
		提示: $x=\frac{t^2-1}{2}, \mathrm{~d}x=t\mathrm{~d}t$; 当 $x=0$ 时 $t=1$, 当 $x=4$ 时 $t=3$.
	\end{block}
	\pause
	\begin{exampleblock}{}
		$$
			\begin{aligned}
				 & \int_0^4 \frac{x+2}{\sqrt{2 x+1}} \mathrm{~d} x \stackrel{\text { 令 } \sqrt{2 x+1}=t}{=} \int_1^3 \frac{\frac{t^2-1}{2}+2}{t} \cdot t \mathrm{~d} t \\
				 & =\frac{1}{2} \int_1^3\left(t^2+3\right) \mathrm{~d} t                                                                                               \\
				 & =\left.\frac{1}{2}\left[\frac{1}{3} t^3+3 t\right]\right|_1 ^3                                                                                      \\
				 & =\frac{1}{2}\left[\left(\frac{27}{3}+9\right)-\left(\frac{1}{3}+3\right)\right]=\frac{22}{3} .
			\end{aligned}
		$$
	\end{exampleblock}
\end{frame}

\begin{frame}
	\frametitle{定积分的换元积分法}
	\everymath{\displaystyle}
	\begin{block}{}
		证明: 若 $f(x)$ 在 $[-a, a]$ 上连续且为偶函数, 则
		$$
			\int_{-a}^a f(x) \mathrm{~d} x=2 \int_0^a f(x) \mathrm{~d} x .
		$$
	\end{block}
	\pause
	\begin{exampleblock}{}
		因为 $\int_{-a}^a f(x) \mathrm{~d} x=\int_{-a}^0 f(x) \mathrm{~d} x+\int_0^a f(x) \mathrm{~d} x$,\\\vspace{0.2cm}
		由于 $\int_{-a}^0 f(x) \mathrm{~d} x \stackrel{\text { 令 } x=-t}{=}-\int_a^0 f(-t) \mathrm{~d} t=\int_0^a f(-t) \mathrm{~d} t=\int_0^a f(-x) \mathrm{~d} x$,\\\vspace{0.2cm}
		因此,
		$$
			\begin{aligned}
				\int_{-a}^a f(x) \mathrm{~d} x & =\int_0^a f(-x) \mathrm{~d} x+\int_0^a f(x) \mathrm{~d} x                                            \\
				                               & =\int_0^a[f(-x)+f(x)] \mathrm{~d} x=\int_{-a}^a 2 f(x) \mathrm{~d} x=2 \int_0^a f(x) \mathrm{~d} x .
			\end{aligned}
		$$
	\end{exampleblock}
\end{frame}

\begin{frame}
	\frametitle{定积分的换元积分法}
	\everymath{\displaystyle}
	\begin{block}{}
		若 $f(x)$ 在 $[0,1]$ 上连续, 证明\\\vspace{0.1cm}
		$\int_0^{\frac{\pi}{2}} f(\sin x) \mathrm{~d} x=\int_0^{\frac{\pi}{2}} f(\cos x) \mathrm{~d} x$; $\int_0^\pi x f(\sin x) \mathrm{~d} x=\frac{\pi}{2} \int_0^\pi f(\sin x) \mathrm{~d} x$.
	\end{block}
	\pause
	\begin{exampleblock}{}
		$$
			\begin{aligned}
				 & \int_0^{\frac{\pi}{2}} f(\sin x) \mathrm{~d} x=-\int_{\frac{\pi}{2}}^0 f\left[\sin \left(\frac{\pi}{2}-t\right)\right] \mathrm{~d} t        \\
				 & \quad=\int_0^{\frac{\pi}{2}} f\left[\sin \left(\frac{\pi}{2}-t\right)\right] \mathrm{~d} t=\int_0^{\frac{\pi}{2}} f(\cos x) \mathrm{~d} x .
			\end{aligned}
		$$
	\end{exampleblock}
\end{frame}

\begin{frame}
	\frametitle{定积分的换元积分法}
	\everymath{\displaystyle}
	\begin{exampleblock}{}
		令 $x=-t$, 则
		$$
			\begin{aligned}
				\int_0^\pi x f(\sin x) \mathrm{~d} x & =-\int_\pi^0(\pi-t) f[\sin (\pi-t)] \mathrm{~d} t                                          \\
				                                     & =\int_0^\pi(\pi-t) f[\sin (\pi-t)] \mathrm{~d} t=\int_0^\pi(\pi-t) f(\sin t) \mathrm{~d} t \\
				                                     & =\pi \int_0^\pi f(\sin t) \mathrm{~d} t-\int_0^\pi t f(\sin t) \mathrm{~d} t               \\
				                                     & =\pi \int_0^\pi f(\sin x) \mathrm{~d} x-\int_0^\pi x f(\sin x) \mathrm{~d} x,
			\end{aligned}
		$$
		所以
		$$
			\int_0^\pi x f(\sin x) \mathrm{~d} x=\frac{\pi}{2} \int_0^\pi f(\sin x) \mathrm{~d} x.
		$$
	\end{exampleblock}
\end{frame}

\begin{frame}
	\frametitle{定积分的换元积分法}
	\everymath{\displaystyle}
	\begin{block}{}
		设函数
		$$
			f(x)=\left\{\begin{array}{cc}x e^{-x^2} & x \geq 0 \\ \frac{1}{1+\cos x} & -1<x<0\end{array}\right.,
		$$
		计算 $\int_1^4 f(x-2) \mathrm{~d} x$.\\\vspace{0.2cm}
		提示: 设 $x-2=t$, 则 $\mathrm{~d} x=\mathrm{~d} t$; 当 $x=1$ 时 $t=-1$, 当 $x=4$ 时 $t=2$.
	\end{block}
	\pause
	\begin{exampleblock}{}
		设 $x-2=t$, 则
		$$
			\begin{aligned}
				\int_1^4 f(x-2) \mathrm{~d} x & =\int_{-1}^2 f(t) \mathrm{~d} t=\int_{-1}^0 \frac{1}{1+\cos t} \mathrm{~d} t+\int_0^2 t e^{-t^2} \mathrm{~d} t                \\
				                              & =\left[\tan \frac{t}{2}\right]_{-1}^0-\left[\frac{1}{2} e^{-t^2}\right]_0^2=\tan \frac{1}{2}-\frac{1}{2} e^{-4}+\frac{1}{2} .
			\end{aligned}
		$$
	\end{exampleblock}
\end{frame}

\begin{frame}
	\frametitle{定积分的换元积分法}
	\everymath{\displaystyle}
	\begin{block}{练习}
		\begin{enumerate}
			\item
			      $$
				      \int_0^1 x\left(x^2+1\right)^3 \mathrm{~d} x;
			      $$
			\item
			      $$
				      \int_0^1 \frac{e^x \mathrm{~d} x}{e^x+1}.
			      $$
		\end{enumerate}
	\end{block}
	\pause
	\begin{exampleblock}{答案}
		\begin{enumerate}[<+->]
			\item $\frac{15}{8}$;\vspace{0.5cm}
			\item $\ln \left(\frac{1+e}{2}\right)$.
		\end{enumerate}
	\end{exampleblock}
\end{frame}

\begin{frame}
	\frametitle{定积分的换元积分法}
	\everymath{\displaystyle}
	\begin{block}{练习}
		\begin{enumerate}
			\item
			      $$
				      \int_0^{1 / 2} \frac{x^2}{\sqrt{1-x^2}} \mathrm{~d} x,\quad (x=\cos t);
			      $$
			\item
			      $$
				      \int_1^2 \frac{\sqrt{x^2-1}}{x} \mathrm{~d} x,\quad (x=\csc t).
			      $$
		\end{enumerate}
	\end{block}
	\pause
	\begin{exampleblock}{答案}
		\begin{enumerate}[<+->]
			\item $\frac{1}{24} \left(2 \pi -3 \sqrt{3}\right)$;\vspace{0.5cm}
			\item $\sqrt{3}-\frac{\pi }{3}$.
		\end{enumerate}
	\end{exampleblock}
\end{frame}

\subsection{分部积分}

\begin{frame}
	\frametitle{定积分的分部积分法}
	\everymath{\displaystyle}
	\begin{block}{定积分的分部积分公式}
		设函数 $u(x) 、 v(x)$ 在区间 $[a, b]$ 上具有连续导数 $u^{\prime}(x) 、 v^{\prime}(x)$, 由 $(u v)^{\prime}=u^{\prime} v+u v^{\prime}$ 得 $u v^{\prime}=u v-u^{\prime} v$, 式两端在区间 $[a, b]$ 上积分得
		$$
			\int_a^b u v^{\prime} \mathrm{~d} x=[u v]_a^b-\int_a^b u^{\prime} v \mathrm{~d} x
		$$
		或
		$$
			\int_a^b u \mathrm{~d} v=[u v]_a^b-\int_a^b v \mathrm{~d} u.
		$$
	\end{block}
\end{frame}

\begin{frame}
	\frametitle{定积分的分部积分法}
	\everymath{\displaystyle}
	\begin{block}{计算 $\int_0^{\frac{1}{2}} \arcsin x \mathrm{~d} x$}
		$$
			\begin{aligned}
				 & \int_0^{\frac{1}{2}} \arcsin x \mathrm{~d} x=[x \arcsin x]_0^{\frac{1}{2}}-\int_0^{\frac{1}{2}} x \mathrm{~d} \arcsin x \\
				 & =\frac{1}{2} \cdot \frac{\pi}{6}-\int_0^{\frac{1}{2}} \frac{x}{\sqrt{1-x^2}} \mathrm{~d} x                              \\
				 & =\frac{\pi}{12}+\frac{1}{2} \int_0^{\frac{1}{2}} \frac{1}{\sqrt{1-x^2}} \mathrm{~d}\left(1-x^2\right)                   \\
				 & =\frac{\pi}{12}+\left[\sqrt{1-x^2}\right]_0^{\frac{1}{2}}=\frac{\pi}{12}+\frac{\sqrt{3}}{2}-1 .
			\end{aligned}
		$$
	\end{block}
\end{frame}

\begin{frame}
	\frametitle{定积分的分部积分法}
	\everymath{\displaystyle}
	\begin{block}{计算 $\int_0^1 e^{\sqrt{x}} \mathrm{~d} x$}
		令 $\sqrt{x}=t$, 则
		$$
			\begin{aligned}
				\int_0^1 e^{\sqrt{x}} \mathrm{~d} x & =2 \int_0^1 e^t t \mathrm{~d} t =2 \int_0^1 t \mathrm{~d} e^t                       \\
				                                    & =2\left[t e^t\right]_0^1-2 \int_0^1 e^t \mathrm{~d} t =2 e-2\left[e^t\right]_0^1=2.
			\end{aligned}
		$$
	\end{block}
\end{frame}

\begin{frame}
	\frametitle{定积分的分部积分法}
	\everymath{\displaystyle}
	{\small
		\begin{block}{计算 $I_n=\int_0^{\frac{\pi}{2}} \sin ^n x \mathrm{~d} x\quad$ ($n$为正整数) }
			$$
				\begin{aligned}
					I_n & =\int_0^{\frac{\pi}{2}} \sin ^n x \mathrm{~d} x=-\int_0^{\frac{\pi}{2}} \sin ^{n-1} x \mathrm{~d} \cos x                                           \\
					    & =-\left[\cos x \sin ^{n-1} x\right]_0^{\frac{\pi}{2}}+\int_0^{\frac{\pi}{2}} \cos x \mathrm{~d} \sin ^{n-1} x                                      \\
					    & =(n-1) \int_0^{\frac{\pi}{2}} \cos ^2 x \sin ^{n-2} x \mathrm{~d} x=(n-1) \int_0^{\frac{\pi}{2}}\left(\sin ^{n-2} x-\sin ^n x\right) \mathrm{~d} x \\
					    & =(n-1) \int_0^{\frac{\pi}{2}} \sin ^{n-2} x \mathrm{~d} x-(n-1) \int_0^{\frac{\pi}{2}} \sin ^n x \mathrm{~d} x                                     \\
					    & =(n-1) I_n-{ }_2-(n-1) I_n.
				\end{aligned}
			$$
			所以,
			$$
				I_n=\frac{n-1}{n} I_{n-2},\quad I_0=\int_0^{\frac{\pi}{2}} \mathrm{~d} x=\frac{\pi}{2}, \quad I_1=\int_0^{\frac{\pi}{2}} \sin x \mathrm{~d} x=1.
			$$
		\end{block}
	}
\end{frame}

\begin{frame}
	\frametitle{定积分的分部积分法}
	\everymath{\displaystyle}
	\begin{block}{练习}
		\begin{enumerate}
			\item
			      $$
				      \int_1^5 \ln x  \mathrm{~d} x;
			      $$
			\item
			      $$
				      \int_0^{\frac{\pi}{2}} x \sin x \mathrm{~d} x.
			      $$
		\end{enumerate}
	\end{block}
	\pause 
	\begin{exampleblock}{答案}
		\begin{enumerate}[<+->]
			\item $5\ln 5-4$;\vspace{0.5cm}
			\item $1$.
		\end{enumerate}
	\end{exampleblock}
\end{frame}

\begin{frame}
	\frametitle{定积分的分部积分法}
	\everymath{\displaystyle}
	\begin{block}{练习}
		\begin{enumerate}
			\item
			      $$
				      \int_0^1 x e^{-x}  \mathrm{~d} x;
			      $$
			\item
			      $$
				      \int_1^2 x \ln x \mathrm{~d} x.
			      $$
		\end{enumerate}
	\end{block}
	\pause 
	\begin{exampleblock}{答案}
		\begin{enumerate}[<+->]
			\item $\frac{e-2}{e}$;\vspace{0.5cm}
			\item $\ln 4-\frac{3}{4}$.
		\end{enumerate}
	\end{exampleblock}
\end{frame}

% Thank you page
\beamertemplateshadingbackground{structure.fg!90}{structure.fg}
\begin{frame}[plain]
	\vfill
	\centering
	{
		\centering \Huge \color{white} Thank you for your attention!\\[10pt]Questions?
	}
	\vfill
\end{frame}


\end{document}


