% !TEX encoding = UTF-8 Unicode.

% Based on https://github.com/Miracle0565/BUCT-Beamer-Theme

\documentclass[
10pt,
aspectratio=43,
]{beamer}
\setbeamercovered{transparent=10}
\usetheme[
%  showheader,
%  red,
  purple,
%  gray,
%  graytitle,
  colorblocks,
%  noframetitlerule,
]{Verona}

\usepackage[T1]{fontenc}
\usepackage{tikz}
\usepackage[utf8]{inputenc}
\usepackage{lipsum}
%%%%%%%%%%%%%%%%%%%%%%%%%%%%%%%
% Mac上使用如下命令声明隶书字体, windows也有相关方式, 大家可自行修改
\providecommand{\lishu}{\CJKfamily{zhli}}
%%%%%%%%%%%%%%%%%%%%%%%%%%%%%%%
\usepackage{tikz}
\usetikzlibrary{fadings}
%
%\setbeamertemplate{sections/subsections in toc}[ball]
\usepackage{xeCJK}
\usepackage{listings}
\usepackage{caption}
\usepackage{subfigure}
\usefonttheme{professionalfonts}
\def\mathfamilydefault{\rmdefault}
\usepackage{amsmath}
\usepackage{multirow}
\usepackage{booktabs}
\usepackage{bm}
\setbeamertemplate{section in toc}{\hspace*{1em}\inserttocsectionnumber.~\inserttocsection\par}
\setbeamertemplate{subsection in toc}{\hspace*{2em}\inserttocsectionnumber.\inserttocsubsectionnumber.~\inserttocsubsection\par}
\setbeamerfont{subsection in toc}{size=\small}
\AtBeginSection[]{%
	\begin{frame}%
		\frametitle{Outline}%
		\textbf{\tableofcontents[currentsection]} %
	\end{frame}%
}

\AtBeginSubsection[]{%
	\begin{frame}%
		\frametitle{Outline}%
		\textbf{\tableofcontents[currentsection, currentsubsection]} %
	\end{frame}%
}

\title{高等数学C}
%\subtitle{A Simple while elegant template}
\author[P.Yu]{余沛}
\mail{peiy\_gzgs@qq.com}
\institute[Guangzhou College of Technology and Business]{Guangzhou College of Technology and Business \\
  广州工商学院}
\date{\today}
\titlegraphic[width=4cm]{logo.png}{}




%%%%%%%%%%%%%%%%%%%%%%%%%%%%%%%%
% ----------- 标题页 ------------
%%%%%%%%%%%%%%%%%%%%%%%%%%%%%%%%



\begin{document}

\maketitle

%%% define code
\defverbatim[colored]\lstI{
	\begin{lstlisting}[language=C++,basicstyle=\ttfamily,keywordstyle=\color{red}]
	int main() {
	// Define variables at the beginning
	// of the block, as in C:
	CStash intStash, stringStash;
	int i;
	char* cp;
	ifstream in;
	string line;
	[...]
	\end{lstlisting}
}
%%%%%%%%%%%%%%%%%%%%%%%%%%%%%%%%
% ----------- FRAME ------------
%%%%%%%%%%%%%%%%%%%%%%%%%%%%%%%%
\section{微分中值定理}

\subsection{Rolle定理}
\begin{frame}{Rolle定理}
	\begin{block}{定理陈述}
	  设函数$f(x)$满足以下条件: 
	  \begin{enumerate}
		\item $f(x)$在闭区间$[a, b]$上连续. 
		\item $f(x)$在开区间$(a, b)$上可导. 
		\item $f(a) = f(b)$. 
	  \end{enumerate}
	  那么在开区间$(a, b)$内至少存在一个数$c$, 使得$f'(c) = 0$. 
	\end{block}
  
	\pause
  
	\begin{example}
	  考虑函数$f(x) = x^2 - 4x + 3$在区间$[1, 3]$上的情况. 我们可以看到$f(x)$满足Rolle定理的所有条件. 因此, 在区间$(1, 3)$内至少存在一个数$c$, 使得$f'(c) = 0$. 
	\end{example}
  \end{frame}

  
\begin{frame}{Rolle定理的证明}
	\begin{proof}
	  假设函数$f(x)$满足Rolle定理的所有条件, 即在闭区间$[a, b]$上连续, 在开区间$(a, b)$上可导, 并且$f(a) = f(b)$. 我们需要证明存在至少一个数$c$, 使得$f'(c) = 0$. \pause
  
	  由于$f(x)$在闭区间$[a, b]$上连续, 根据闭区间上的最大值和最小值定理, 存在两个数$x_1$和$x_2$, 满足$a \leq x_1 < x_2 \leq b$, 使得$f(x_1)$是$f(x)$在闭区间$[a, b]$上的最大值, $f(x_2)$是$f(x)$在闭区间$[a, b]$上的最小值. \pause
  
	  如果$f(x)$在开区间$(a, b)$上恒为常数, 即$f(x)$在$(a, b)$上的导数恒为零, 那么我们可以取$c$为$(a, b)$中的任意一个数, 使得$f'(c) = 0$. \pause
  
	  否则, 根据导数的定义, 我们可以得到$f(x_1) = f(x_2)$. 由于$f(a) = f(b)$, 根据介值定理, 存在至少一个数$c$, 满足$a < c < b$, 使得$f(c) = f(x_1) = f(x_2)$. 根据Lagrange中值定理, 存在至少一个数$d$, 满足$a < d < c$, 使得
	  \[
		f'(d) = \frac{f(c) - f(a)}{c - a} = 0.
		\]
  
	  综上所述, 我们证明了Rolle定理. 
	\end{proof}
  \end{frame}

\subsection{Lagrange定理}
\begin{frame}{Lagrange定理}
	\begin{block}{定理陈述}
	  设函数$f(x)$满足以下条件: 
	  \begin{enumerate}
		\item $f(x)$在闭区间$[a, b]$上连续. 
		\item $f(x)$在开区间$(a, b)$上可导. 
	  \end{enumerate}
	  那么在开区间$(a, b)$内至少存在一个数$c$, 使得$f(b) - f(a) = f'(c)(b - a)$. 
	\end{block}
  
	\pause
  
	\begin{example}
	  考虑函数$f(x) = x^2$在区间$[1, 3]$上的情况. 我们可以看到$f(x)$满足Lagrange定理的所有条件. 因此, 在区间$(1, 3)$内至少存在一个数$c$, 使得$f(3) - f(1) = f'(c)(3 - 1)$. 
	\end{example}
  \end{frame}

  \begin{frame}{Lagrange中值定理的证明}
	\begin{proof}
	  假设函数$f(x)$和$g(x)$满足Lagrange中值定理的所有条件, 即在闭区间$[a, b]$上连续, 在开区间$(a, b)$上可导, 并且$g'(x) \neq 0$. 我们需要证明存在至少一个数$c$, 使得
	  $$\frac{f(b) - f(a)}{g(b) - g(a)} = \frac{f'(c)}{g'(c)}$$. 
  \pause
	  定义辅助函数
	  $$h(x) = f(x) - \frac{f(b) - f(a)}{g(b) - g(a)}(g(x) - g(a))$$. 根据$h(x)$的定义, 我们有$h(a) = f(a)$和$h(b) = f(b)$. 
	\end{proof}
  \end{frame}

  \begin{frame}{Lagrange中值定理的证明}
	\begin{proof}

  由于$f(x)$和$g(x)$在闭区间$[a, b]$上连续, 并且$g'(x) \neq 0$, 根据复合函数的连续性和导数的除法法则, 我们可以得到$h(x)$在闭区间$[a, b]$上连续, 并且在开区间$(a, b)$上可导. 

  \pause
	  根据Rolle定理, 存在至少一个数$c$, 使得$h'(c) = 0$. 根据$h(x)$的定义和导数的链式法则, 我们可以得到$$h'(c) = f'(c) - \frac{f(b) - f(a)}{g(b) - g(a)}g'(c)$$. 
  
	  综上所述, 我们证明了Lagrange中值定理. 
	\end{proof}
  \end{frame}

  \subsection{Cauchy中值定理}
  \begin{frame}{Cauchy中值定理}
	\begin{block}{定理陈述}
	  设函数$f(x)$和$g(x)$满足以下条件: 
	  \begin{enumerate}
		\item $f(x)$和$g(x)$在闭区间$[a, b]$上连续. 
		\item $f(x)$和$g(x)$在开区间$(a, b)$上可导. 
		\item $g'(x) \neq 0$, 即$g(x)$在$(a, b)$上不恒为零. 
	  \end{enumerate}
	  那么在开区间$(a, b)$内至少存在一个数$c$, 使得$\frac{f(b) - f(a)}{g(b) - g(a)} = \frac{f'(c)}{g'(c)}$. 
	\end{block}
  
	\pause
  
	\begin{example}
	  考虑函数$f(x) = \sin(x)$和$g(x) = \cos(x)$在区间$[0, \frac{\pi}{2}]$上的情况. 我们可以看到$f(x)$和$g(x)$满足Cauchy中值定理的所有条件. 因此, 在区间$(0, \frac{\pi}{2})$内至少存在一个数$c$, 使得$\frac{\sin(\frac{\pi}{2}) - \sin(0)}{\cos(\frac{\pi}{2}) - \cos(0)} = \frac{\cos(c)}{-\sin(c)}$. 
	\end{example}
  \end{frame}

  \begin{frame}{Cauchy中值定理的证明}
	\begin{proof}
	  假设函数$f(x)$和$g(x)$满足Cauchy中值定理的所有条件, 即在闭区间$[a, b]$上连续, 在开区间$(a, b)$上可导, 并且$g'(x) \neq 0$. 我们需要证明存在至少一个数$c$, 使得
	  $$\frac{f(b) - f(a)}{g(b) - g(a)} = \frac{f'(c)}{g'(c)}$$. 
  
	  定义辅助函数
	  $$h(x) = (f(b) - f(a))g(x) - (g(b) - g(a))f(x)$$.
	  根据$h(x)$的定义, 我们有$h(a) = h(b) = 0$. 
	\end{proof}
\end{frame}  

	  \begin{frame}{Cauchy中值定理的证明}
		\begin{proof}
	  由于$f(x)$和$g(x)$在闭区间$[a, b]$上连续, 并且$g'(x) \neq 0$, 根据复合函数的连续性和导数的乘法法则, 我们可以得到$h(x)$在闭区间$[a, b]$上连续, 并且在开区间$(a, b)$上可导. 
  
	  根据Rolle定理, 存在至少一个数$c$, 使得$h'(c) = 0$. 根据$h(x)$的定义和导数的链式法则, 我们可以得到
	  $$h'(c) = (f(b) - f(a))g'(c) - (g(b) - g(a))f'(c)$$. 
  
	  综上所述, 我们证明了Cauchy中值定理. 
	\end{proof}
  \end{frame}


% Thank you page
\beamertemplateshadingbackground{structure.fg!90}{structure.fg}
\begin{frame}[plain]
	\vfill
	\centering
	{
	\centering \Huge \color{white} Thank you for your attention!\\[10pt]Questions?\\ [10pt] Homework: Page167: 1, 2, 3.
	}
	\vfill
\end{frame}


\end{document}


