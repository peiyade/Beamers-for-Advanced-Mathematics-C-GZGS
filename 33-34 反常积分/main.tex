% !TEX encoding = UTF-8 Unicode.

% Based on https://github.com/Miracle0565/BUCT-Beamer-Theme

\documentclass[
10pt,
aspectratio=43,
]{beamer}
\setbeamercovered{transparent=10}
\usetheme[
%  showheader,
%  red,
  purple,
%  gray,
%  graytitle,
  colorblocks,
%  noframetitlerule,
]{Verona}

\usepackage[T1]{fontenc}
\usepackage{tikz}
\usepackage[utf8]{inputenc}
\usepackage{lipsum}
\usepackage{pgfplots}
%%%%%%%%%%%%%%%%%%%%%%%%%%%%%%%
% Mac上使用如下命令声明隶书字体, windows也有相关方式, 大家可自行修改
\providecommand{\lishu}{\CJKfamily{zhli}}
%%%%%%%%%%%%%%%%%%%%%%%%%%%%%%%
\usepackage{tikz}
\usetikzlibrary{fadings}
%
%\setbeamertemplate{sections/subsections in toc}[ball]
\usepackage{xeCJK}
\usepackage{listings}
\usepackage{caption}
\usepackage{subfigure}
\usefonttheme{professionalfonts}
\def\mathfamilydefault{\rmdefault}
\usepackage{amsmath}
\usepackage{multirow}
\usepackage{booktabs}
\usepackage{bm}
\usepackage{mathtools}
\usepackage[T1]{fontenc}
\setbeamertemplate{section in toc}{\hspace*{1em}\inserttocsectionnumber.~\inserttocsection\par}
\setbeamertemplate{subsection in toc}{\hspace*{2em}\inserttocsectionnumber.\inserttocsubsectionnumber.~\inserttocsubsection\par}
\setbeamerfont{subsection in toc}{size=\small}
\AtBeginSection[]{%
	\begin{frame}%
		\frametitle{Outline}%
		\textbf{\tableofcontents[currentsection]} %
	\end{frame}%
}

\AtBeginSubsection[]{%
	\begin{frame}%
		\frametitle{Outline}%
		\textbf{\tableofcontents[currentsection, currentsubsection]} %
	\end{frame}%
}

\pgfplotsset{
    integral segments/.code={\pgfmathsetmacro\integralsegments{#1}},
    integral segments=3,
    integral/.style args={#1:#2}{
        ybar interval,
        domain=#1+((#2-#1)/\integralsegments)/2:#2+((#2-#1)/\integralsegments)/2,
        samples=\integralsegments+1,
        x filter/.code=\pgfmathparse{\pgfmathresult-((#2-#1)/\integralsegments)/2}
    }
}


\title{高等数学C}
%\subtitle{A Simple while elegant template}
\author[P.Yu]{余沛}
\mail{peiy\_gzgs@qq.com}
\institute[Guangzhou College of Technology and Business]{Guangzhou College of Technology and Business \\
  广州工商学院}
\date{\today}
\titlegraphic[width=4cm]{logo.png}{}




%%%%%%%%%%%%%%%%%%%%%%%%%%%%%%%%
% ----------- 标题页 ------------
%%%%%%%%%%%%%%%%%%%%%%%%%%%%%%%%



\begin{document}

\maketitle

%%% define code
\defverbatim[colored]\lstI{
	\begin{lstlisting}[language=C++,basicstyle=\ttfamily,keywordstyle=\color{red}]
	int main() {
	// Define variables at the beginning
	// of the block, as in C:
	CStash intStash, stringStash;
	int i;
	char* cp;
	ifstream in;
	string line;
	[...]
	\end{lstlisting}
}
%%%%%%%%%%%%%%%%%%%%%%%%%%%%%%%%
% ----------- FRAME ------------
%%%%%%%%%%%%%%%%%%%%%%%%%%%%%%%%

\section{反常积分}
\begin{frame}
	\frametitle{回顾:定积分}
	\begin{block}{Riemann 积分的定义}
		设$f(x)$在区间$[a,b]$上有定义, $\mathrm{P}:a=x_0<x_1<x_2<\ldots<x_n=b$.
		在每段$[x_{i-1},x_i]$上 {\bf 任意}选取点$\xi_i$, 得到近似和
		$$
			\sum_{i=1}^n f(\xi_i)(x_i-x_{i-1}).
		$$
		如果对于$\mathrm{P}$的精细化
		我们就称$f(x)$在区间$[a,b]$上是 {\bf Riemann 可积}的, 并将这个极限值称为$f(x)$在$[a,b]$上的 {\bf Riemann 积分}, 记为
		\[
			\lim_{\lambda\to 0} \sum_{i=1}^n f(\xi_i)(x_i-x_{i-1}) := \int_a^b f(x)\mathrm{~d} x.
		\]
	\end{block}

\end{frame}
\begin{frame}
	\frametitle{归纳:定积分的性质}
	\everymath{\displaystyle}
	\begin{enumerate}
		\item 线性性: $\int_a^b[k_1f_1(x) + k_2f_2(x)] \mathrm{d} x=\int_a^b k_1f_1(x) \mathrm{~d} x+\int_a^b k_2f_2(x) \mathrm{~d}x$;
		\item 区间可加性: $\int_a^b f(x) \mathrm{~d} x=\int_a^c f(x) \mathrm{~d} x+\int_c^b f(x) \mathrm{~d} x$;
		\item 保号性: $f(x) \geq 0, x\in[a,b]$, 则 $\int_a^b f(x) \mathrm{~d} x\geq 0$;
		\item 上下界估计: $(b-a)\inf_{x\in[a,b]}f(x)=\int_a^{b} f(x) \mathrm{~d} x=(b-a)\sup_{x\in[a,b]}f(x)$;
		\item 中值定理: 设 $f(x)$ 在 $[a, b]$ 上连续, 则存在 $\xi\in[a, b]$, 使
		      $$
			      \int_a^b f(x) \mathrm{~d} x=f(\xi)(b-a).
		      $$
	\end{enumerate}
	\vspace{0.5cm}
\end{frame}

\begin{frame}
	\frametitle{回顾: Newton-Leibniz公式}

	\begin{theorem}[Newton-Leibniz公式]
		如果函数 $F(x)$ 是连续函数 $f(x)$ 在区间 $[a, b]$ 上的一个原函数, 则
		$$
			\int_a^b f(x) \mathrm{~d} x=F(b)-F(a) .
		$$
	\end{theorem}
	\vspace{1cm}
	\pause
	\begin{quote}
		原函数 $F(x)$ 满足
		$$
			F (x)+C=\int f(x)\mathrm{~d}x.
		$$
	\end{quote}
\end{frame}


\begin{frame}
	\frametitle{再回顾:定积分}
	\begin{block}{Riemann 积分的定义}
		设$f(x)$在区间$[a,b]$上有定义, $\mathrm{P}:a=x_0<x_1<x_2<\ldots<x_n=b$.
		在每段$[x_{i-1},x_i]$上 {\bf 任意}选取点$\xi_i$, 得到近似和
		$$
			\sum_{i=1}^n f(\xi_i)(x_i-x_{i-1}).
		$$
		如果对于$\mathrm{P}$的精细化
		我们就称$f(x)$在区间$[a,b]$上是 {\bf Riemann 可积}的, 并将这个极限值称为$f(x)$在$[a,b]$上的 {\bf Riemann 积分}, 记为
		\[
			\lim_{\lambda\to 0} \sum_{i=1}^n f(\xi_i)(x_i-x_{i-1}) := \int_a^b f(x)\mathrm{~d} x.
		\]
	\end{block}
	\vspace{0.2cm}
	{\color{blue}那么, $f(x)$ 在 $[a,b]$ 上无定义的情况, 和 $a, b$其中之一为无穷大的情形, 应当怎么去定义求面积的过程?}
\end{frame}

\begin{frame}
	\frametitle{无穷上下限反常积分的定义}
	\everymath{\displaystyle}
	\begin{block}{定义}
		设函数 $f(x)$ 在区间 $[a,+\infty)$ 上连续, 取 $b>a$. 如果极限 $\lim _{b \rightarrow+\infty} \int_a^b f(x) \mathrm{~d}x$存在, 则称此极限为函数 $f(x)$ 在无穷区间 $[a,+\infty)$ 上的反常积分, 记作 $\int_a^{+\infty} f(x) \mathrm{~d}x$, 即 $\int_a^{+\infty} f(x) \mathrm{~d}x=\lim _{b \rightarrow+\infty} \int_a^b f(x) \mathrm{~d}x$
								这时也称反常积分 $\int_a^{+\infty} f(x) \mathrm{~d}x$ 收敛.\\\vspace{0.2cm}
								同理, 可以定义处在无穷区间 $(-\infty,b]$ 上的反常积分 $\int_{-\infty}^b f(x) \mathrm{~d}x$ 和 无穷区间 $(-\infty,+\infty]$ 上的反常积分 $\int_{-\infty}^{+\infty} f(x) \mathrm{~d}x$ 及其相关的敛散性.
	\end{block}
\end{frame}

\begin{frame}
	\frametitle{无穷上下限反常积分的计算}
	\everymath{\displaystyle}
	利用 Newton-Leibniz 公式, 如果 $F(x)$ 是 $f(x)$ 的原函数
	$$
		\begin{aligned}
			\int_a^{+\infty} f(x)  \mathrm{~d}x & =\lim _{b \rightarrow+\infty} \int_a^b f(x) \mathrm{~d}x=\lim _{b \rightarrow+\infty}[F(x)]_a^b \\\vspace{0.4cm}
			                                    & =  \lim _{b \rightarrow+\infty} F(b)-F(a)=\lim _{x \rightarrow+\infty} F(x)-F(a) .
		\end{aligned}
	$$
	类似地
	$$
		\begin{aligned}
			\int_{-\infty}^b f(x)  \mathrm{~d}x & =\lim _{a \rightarrow-\infty} \int_a^b f(x) \mathrm{~d}x=\lim _{a \rightarrow-\infty}[F(x)]_a^b \\\vspace{0.4cm}
			                                    & =  \lim _{a \rightarrow-\infty} F(b)-F(a)=F(b)-\lim _{x \rightarrow-\infty} F(x) .
		\end{aligned}
	$$
\end{frame}


\begin{frame}
	\frametitle{无穷上下限反常积分的计算}
	\everymath{\displaystyle}
	\begin{block}{计算反常积分 $\int_{-\infty}^{+\infty} \frac{1}{1+x^2} \mathrm{~d}x$}
		$$
			\begin{aligned}
				\int_{-\infty}^{+\infty} \frac{1}{1+x^2} \mathrm{~d} x & =[\arctan x]_{-\infty}^{+\infty} =\lim _{x \rightarrow+\infty} \arctan x-\lim _{x \rightarrow-\infty} \arctan x \\
				                                                       & =\frac{\pi}{2}-\left(-\frac{\pi}{2}\right)=\pi .
			\end{aligned}
		$$
	\end{block}
\end{frame}

\begin{frame}
	\frametitle{无穷上下限反常积分的计算}
	\everymath{\displaystyle}
	\begin{block}{计算反常积分 $\int_0^{+\infty} t e^{-p t} \mathrm{~d} t \quad(p$ 是常数, 且 $p>0)$}
		$$
			\begin{aligned}
				\int_0^{+\infty} t e^{-p t} \mathrm{~d}t & =\left[\int t e^{-p t} \mathrm{~d}t\right]_0^{+\infty}=\left[-\frac{1}{p} \int t \mathrm{~d} e^{-p t}\right]_0^{+\infty} \\
				                                         & =\left[-\frac{1}{p} t e^{-p t}+\frac{1}{p} \int e^{-p t} \mathrm{~d}t\right]_0^{+\infty}                                 \\
				                                         & =\left[-\frac{1}{p} t e^{-p t}-\frac{1}{p^2} e^{-p t}\right]_0^{+\infty}                                                 \\
				                                         & =\lim _{t \rightarrow+\infty}\left[-\frac{1}{p} t e^{-p t}-\frac{1}{p^2} e^{-p t}\right]+\frac{1}{p^2}=\frac{1}{p^2} .
			\end{aligned}
		$$
	\end{block}
\end{frame}

\begin{frame}
	\frametitle{无穷上下限反常积分的计算}
	\everymath{\displaystyle}
	\begin{block}{讨论反常积分 $\int_a^{+\infty} \frac{1}{x^p} \mathrm{~d}x(a>0)$ 的敛散性}
		\begin{enumerate}
			\item 当 $p=1$ 时, $\int_a^{+\infty} \frac{1}{x^p} \mathrm{~d}x=\int_a^{+\infty} \frac{1}{x} \mathrm{~d}x=[\ln x]_a^{+\infty}=+\infty$.
			\item 当 $p=1$ 时, $\int_a^{+\infty} \frac{1}{x^p} \mathrm{~d}x=\int_a^{+\infty} \frac{1}{x} \mathrm{~d}x=[\ln x]_a^{+\infty}=+\infty$.
			\item 当 $p<1$ 时, $\int_a^{+\infty} \frac{1}{x^p} \mathrm{~d}x=\left[\frac{1}{1-p} x^{1-p}\right]_a^{+\infty}=+\infty$.
			\item 当 $p>1$ 时, $\int_a^{+\infty} \frac{1}{x^p} \mathrm{~d}x=\left[\frac{1}{1-p} x^{1-p}\right]_a^{+\infty}=\frac{a^{1-p}}{p-1}$.
		\end{enumerate}
		因此, 当 $p>1$ 时, 此反常积分收玫, 其值为 $\frac{a^{1-p}}{p-1}$; 当 $p \leq 1$ 时, 此反常积分发散.
	\end{block}
\end{frame}

\begin{frame}
	\frametitle{瑕积分的定义}
	\everymath{\displaystyle}
	\begin{block}{定义}
		设函数 $f(x)$ 在区间 $(a, b]$ 上连续, 而在点 $a$ 的右邻域内无界. 取 $\varepsilon>0$, 如果极限
		$$
			\lim _{t \rightarrow a^{+}} \int_t^b f(x) \mathrm{~d}x
		$$

		存在, 则称此极限为函数 $f(x)$ 在 $(a, b]$ 上的反常积分, 仍然记作 $\int_a^b f(x) \mathrm{~d}x$, 即
		$$
			\int_a^b f(x) \mathrm{~d}x=\lim _{t \rightarrow a^{+}} \int_t^b f(x) \mathrm{~d}x .
		$$

		这时也称反常积分 $\int_a^b f(x) \mathrm{~d}x$ 收敛.
		如果上述极限不存在, 就称反常积分 $\int_a^b f(x) \mathrm{~d}x$ 发散.
	\end{block}
\end{frame}

\begin{frame}
	\frametitle{无穷上下限反常积分的计算}
	\everymath{\displaystyle}
	利用 Newton-Leibniz 公式, 如果 $F(x)$ 是 $f(x)$ 的原函数
	$$
		\begin{aligned}
			\int_a^b f(x) \mathrm{~d}x & =\lim _{t \rightarrow a^{+}} \int_t^b f(x) \mathrm{~d}x=\lim _{t \rightarrow a^{+}}[F(x)]_t^b \\
			                           & =F(b)-\lim _{t \rightarrow a^{+}} F(t)=F(b)-\lim _{x \rightarrow a^{+}} F(x) .
		\end{aligned}
	$$
	\begin{enumerate}
		\item 当 $a$ 为瑕点时, $\int_a^b f(x) \mathrm{~d}x=[F(x)]_a^b=F(b)-\lim _{x \rightarrow a^{+}} F(x)$ ;
		\item 当 $b$ 为瑕点时, $\int_a^b f(x) \mathrm{~d}x=[F(x)]_a^b=\lim _{x \rightarrow b^{-}} F(x)-F(a)$;
		\item
		      当 $c(a<c<b)$ 为瑕点时,
		      $$
			      \begin{aligned}
				      \int_a^b f(x) \mathrm{~d}x & =\int_a^c f(x) \mathrm{~d}x+\int_c^b f(x) \mathrm{~d}x                                                   \\
				                                 & =\left[\lim _{x \rightarrow c^{-}} F(x)-F(a)\right]+\left[F(b)-\lim _{x \rightarrow c^{+}} F(x)\right] .
			      \end{aligned}
		      $$
	\end{enumerate}
\end{frame}

\begin{frame}
	\frametitle{瑕积分的计算}
	\everymath{\displaystyle}
	\begin{block}{计算反常积分 $\int_0^a \frac{1}{\sqrt{a^2-x^2}} d x$}
		因为 $\lim _{x \rightarrow a^{-}} \frac{1}{\sqrt{a^2-x^2}}=+\infty$, 所以点 $a$ 为被积函数的瑕点.
		$$
			\int_0^a \frac{1}{\sqrt{a^2-x^2}} d x=\left[\arcsin \frac{x}{a}\right]_0^a=\lim _{x \rightarrow a^{-}} \arcsin \frac{x}{a}-0=\frac{\pi}{2}
		$$
	\end{block}
	\begin{block}{讨论反常积分 $\int_{-1}^1 \frac{1}{x^2} d x$ 的收敛性}
		函数 $\frac{1}{x^2}$ 在区间 $[-1,1]$ 上除 $x=0$ 外连续, 且 $\lim _{x \rightarrow 0} \frac{1}{x^2}=\infty$.
		由于 $\int_{-1}^0 \frac{1}{x^2} d x=\left[-\frac{1}{x}\right]_{-1}^0=\lim _{x \rightarrow 0^{-}}\left(-\frac{1}{x}\right)-1=+\infty$,即反常积分 $\int_{-1}^0 \frac{1}{x^2} d x$ 发散, 所以反常积分 $\int_{-1}^1 \frac{1}{x^2} d x$ 发散.
	\end{block}
\end{frame}

\begin{frame}
	\frametitle{瑕积分的计算}
	\everymath{\displaystyle}
	\begin{block}{讨论反常积分 $\int_a^b \frac{d x}{(x-a)^q}$ 的收敛性}
		\begin{enumerate}
			\item 当 $q=1$ 时, $\int_a^b \frac{d x}{(x-a)^q}=\int_a^b \frac{d x}{x-a}=[\ln (x-a)]_a^b=+\infty$;
			\item 当 $q>1$ 时, $\int_a^b \frac{d x}{(x-a)^q}=\left[\frac{1}{1-q}(x-a)^{1-q}\right]_a^b=+\infty$;
			\item 当 $q<1$ 时, $\int_a^b \frac{d x}{(x-a)^q}=\left[\frac{1}{1-q}(x-a)^{1-q}\right]_a^b=\frac{1}{1-q}(b-a)^{1-q}$.
		\end{enumerate}
		因此, 当 $q<1$ 时, 此反常积分收敛, 其值为 $\frac{1}{1-q}(b-a)^{1-q}$; 当 $q \geq 1$ 时, 此反常积分发散.
	\end{block}
\end{frame}

% Thank you page
\beamertemplateshadingbackground{structure.fg!90}{structure.fg}
\begin{frame}[plain]
	\vfill
	\centering
	{
		\centering \Huge \color{white} Thank you for your attention!\\[10pt]Questions?\\[10pt]Homework: Page 224: 4, 5:(1)(3)(5)(7)(9)(11)(13)(15)\\
		6: (1)(3)(5)(7)(9)\\
		12:(1)(3)(5)(7)(9)
	}
	\vfill
\end{frame}


\end{document}


