% !TEX encoding = UTF-8 Unicode.

% Based on https://github.com/Miracle0565/BUCT-Beamer-Theme

\documentclass[
10pt,
aspectratio=43,
]{beamer}
\setbeamercovered{transparent=10}
\usetheme[
%  showheader,
%  red,
  purple,
%  gray,
%  graytitle,
  colorblocks,
%  noframetitlerule,
]{Verona}

\usepackage[T1]{fontenc}
\usepackage{tikz}
\usepackage[utf8]{inputenc}
\usepackage{lipsum}
\usepackage{pgfplots}
%%%%%%%%%%%%%%%%%%%%%%%%%%%%%%%
% Mac上使用如下命令声明隶书字体, windows也有相关方式, 大家可自行修改
\providecommand{\lishu}{\CJKfamily{zhli}}
%%%%%%%%%%%%%%%%%%%%%%%%%%%%%%%
\usepackage{tikz}
\usetikzlibrary{fadings}
%
%\setbeamertemplate{sections/subsections in toc}[ball]
\usepackage{xeCJK}
\usepackage{listings}
\usepackage{caption}
\usepackage{subfigure}
\usefonttheme{professionalfonts}
\def\mathfamilydefault{\rmdefault}
\usepackage{amsmath}
\usepackage{multirow}
\usepackage{booktabs}
\usepackage{bm}
\usepackage{mathtools}
\usepackage[T1]{fontenc}
\setbeamertemplate{section in toc}{\hspace*{1em}\inserttocsectionnumber.~\inserttocsection\par}
\setbeamertemplate{subsection in toc}{\hspace*{2em}\inserttocsectionnumber.\inserttocsubsectionnumber.~\inserttocsubsection\par}
\setbeamerfont{subsection in toc}{size=\small}
\AtBeginSection[]{%
	\begin{frame}%
		\frametitle{Outline}%
		\textbf{\tableofcontents[currentsection]} %
	\end{frame}%
}

\AtBeginSubsection[]{%
	\begin{frame}%
		\frametitle{Outline}%
		\textbf{\tableofcontents[currentsection, currentsubsection]} %
	\end{frame}%
}

\pgfplotsset{
    integral segments/.code={\pgfmathsetmacro\integralsegments{#1}},
    integral segments=3,
    integral/.style args={#1:#2}{
        ybar interval,
        domain=#1+((#2-#1)/\integralsegments)/2:#2+((#2-#1)/\integralsegments)/2,
        samples=\integralsegments+1,
        x filter/.code=\pgfmathparse{\pgfmathresult-((#2-#1)/\integralsegments)/2}
    }
}


\title{高等数学C}
%\subtitle{A Simple while elegant template}
\author[P.Yu]{余沛}
\mail{peiy\_gzgs@qq.com}
\institute[Guangzhou College of Technology and Business]{Guangzhou College of Technology and Business \\
  广州工商学院}
\date{\today}
\titlegraphic[width=4cm]{logo.png}{}




%%%%%%%%%%%%%%%%%%%%%%%%%%%%%%%%
% ----------- 标题页 ------------
%%%%%%%%%%%%%%%%%%%%%%%%%%%%%%%%



\begin{document}

\maketitle

%%% define code
\defverbatim[colored]\lstI{
	\begin{lstlisting}[language=C++,basicstyle=\ttfamily,keywordstyle=\color{red}]
	int main() {
	// Define variables at the beginning
	// of the block, as in C:
	CStash intStash, stringStash;
	int i;
	char* cp;
	ifstream in;
	string line;
	[...]
	\end{lstlisting}
}
%%%%%%%%%%%%%%%%%%%%%%%%%%%%%%%%
% ----------- FRAME ------------
%%%%%%%%%%%%%%%%%%%%%%%%%%%%%%%%
\section{知识补充}
\subsection{微分关系}
\begin{frame}
	\frametitle{微分的定义}
	\begin{block}{定义}
		设函数 $y=f(x)$ 在 $x_0$ 的某个邻域内有定义, 当自变量在 $x_0$ 处取得增量 $\Delta x$ 时,如果函数的增量 $\Delta y=f\left(x_0+\Delta x\right)-f\left(x_0\right)$ 可以表示为
		$$
			\Delta y=A \Delta x+o(\Delta x)
		$$

		其中 $\mathrm{A}$ 是与 $x_0$ 有关而与 $\Delta x$ 无关的常数, $o(\Delta x)$ 是比 $\Delta x$ 高阶的无穷小量, 则函数 $y=f(x)$在点 $x_0$ 可微, $A \Delta x$ 称为微分, 即 $\mathrm{~d} y=A \Delta x$.

		\vspace{0.2cm}
		由于 $\Delta x=\Delta x$ 是恒等等式, 于是有
		$$
			\mathrm{~d} y=A \Delta x=A\mathrm{~d}x.
		$$
	\end{block}
	\vspace{0.5cm}
	{\huge
		\begin{quote}
			$$
				\mathrm{d} y=f^{\prime}(x) \mathrm{~d} x
			$$
		\end{quote}
	}
\end{frame}


\begin{frame}
	\frametitle{参数函数求微分}
	\everymath{\displaystyle}
	\begin{block}{隐函数求微分方式}
		\begin{equation*}
			\left\{
			\begin{array}{ll}
				y=f_1(t);\vspace{0.2cm} \\
				x=f_2(t).
			\end{array}
			\right.
		\end{equation*}
		求 $\frac{\mathrm{d}y}{\mathrm{d}x}$, $\frac{\mathrm{d}^2y}{\mathrm{d}x^2}$.
	\end{block}
	\begin{exampleblock}{隐函数求微分}
		对两个方程进行微分, 得到:
		\begin{equation*}
			\left\{
			\begin{array}{ll}
				\mathrm{d}y=\mathrm{d}(f_1(t))=f_1'(t)\mathrm{d}t;\vspace{0.2cm} \\
				\mathrm{d}x=\mathrm{d}(f_2(t))=f_2'(t)\mathrm{d}t.
			\end{array}
			\right.
		\end{equation*}
		于是有
		$$
			\frac{\mathrm{d}y}{\mathrm{d}x}=\frac{f_1'(t)\mathrm{d}t}{f_2'(t)\mathrm{d}t}=\frac{f_1'(t)}{f_2'(t)}.
		$$
	\end{exampleblock}
\end{frame}

\begin{frame}
	\frametitle{参数函数求微分}
	\everymath{\displaystyle}
	\begin{exampleblock}{隐函数求二阶微分}
		根据以上计算, 可以得到:
		\begin{equation*}
			\left\{
			\begin{array}{ll}
				\frac{\mathrm{d}y}{\mathrm{d}x}=\frac{f_1'(t)}{f_2'(t)},\vspace{0.1cm} \\
				\mathrm{d}x=f_2'(t).
			\end{array}
			\right.
		\end{equation*}
		分别进行对 $t$ 变量的微分, 得到
		$$
			\left\{
			\begin{array}{ll}
				\mathrm{d}\left(\frac{\mathrm{d}y}{\mathrm{d}x}\right)=\left(\frac{f_1'(t)}{f_2'(t)}\right)'\mathrm{d}t\vspace{0.1cm} \\
				\mathrm{d}x=f_2'(t)\mathrm{d}t.
			\end{array}
			\right.
		$$
		因此:
		$$
			\frac{\mathrm{d}^2y}{\mathrm{d}x^2}=\frac{\mathrm{d}\left(\frac{\mathrm{d}y}{\mathrm{d}x}\right)}{\mathrm{d}x}=\frac{\left(\frac{f_1'(t)}{f_2'(t)}\right)'\mathrm{d}t}{f_2'(t)\mathrm{d}t}=\frac{\left(\frac{f_1'(t)}{f_2'(t)}\right)'}{f_2'(t)}=\frac{f_1^{''}(t)f_2^{'}(t)-f_1^{'}(t)f_2^{''}(t)}{\left(f_2'(t)\right)^3}.
		$$
	\end{exampleblock}
\end{frame}

\begin{frame}
	\frametitle{隐函数求微分}
	\everymath{\displaystyle}
	\begin{block}{隐函数}
		由方程 $F(x, y)=0$ 所确定的函数, 称之为隐函数.
	\end{block}
	\begin{exampleblock}{举个栗子}
		$$
			e^{x+y}=\sin x\cos y,\qquad \ln \sqrt{x^2+y^2}=\arctan \frac{y}{x}.
		$$
	\end{exampleblock}
	\begin{block}{计算$\frac{\mathrm{d}y}{\mathrm{d}x}$}
		\begin{enumerate}
			\item 移项: 化归成 $F(x, y)=0$ 的形式;
			\item 计算微分: $\mathrm{d}F(x,y)=0\,\,\Rightarrow\,\,F_x(x,y)\mathrm{d}x+F_y(x,y)\mathrm{d}y=0$;
			\item 得到结果: $F_y(x,y)\mathrm{d}y=-F_x(x,y)\mathrm{d}x\,\,\Rightarrow\,\,\frac{\mathrm{d}y}{\mathrm{d}x}=-\frac{F_x(x,y)}{F_y(x,y)}$.
		\end{enumerate}
	\end{block}
\end{frame}

\begin{frame}
	\frametitle{隐函数求微分}
	\everymath{\displaystyle}
	\begin{exampleblock}{计算栗子里的$\frac{\mathrm{d}y}{\mathrm{d}x}$和$\frac{\mathrm{d}^2y}{\mathrm{d}x^2}$}
		$$
			e^{x+y}=\sin x\cos y.
		$$
	\end{exampleblock}
	\begin{block}{Ans}
		由于:
		$$
			\mathrm{d}\left(e^{x+y}-\sin x\cos y\right)=\left(e^{x+y}-\cos x\cos y\right)\mathrm{d}x+\left(e^{x+y}+\sin x\sin y\right)\mathrm{d}y=0.
		$$
		得到
		$$
			\frac{\mathrm{d}y}{\mathrm{d}x}=-\frac{e^{x+y}-\cos x\cos y}{e^{x+y}+\sin x\sin y}.
		$$
		那么怎么得到 $\frac{\mathrm{d}^2y}{\mathrm{d}x^2}$呢?
	\end{block}
\end{frame}

\begin{frame}
	\frametitle{隐函数求微分}
	\everymath{\displaystyle}
	\begin{exampleblock}{计算栗子里的$\frac{\mathrm{d}y}{\mathrm{d}x}$和$\frac{\mathrm{d}^2y}{\mathrm{d}x^2}$}
		$$
			e^{x+y}=\sin x\cos y,\,\,\, \text{已知}\,\,\, \frac{\mathrm{d}y}{\mathrm{d}x}=-\frac{e^{x+y}-\cos x\cos y}{e^{x+y}+\sin x\sin y}.
		$$
	\end{exampleblock}
	\begin{block}{Ans}
		由于:
		$$
			\begin{aligned}
				\mathrm{d}\left(\frac{\mathrm{d}y}{\mathrm{d}x}\right) & =-\mathrm{d}\left(\frac{e^{x+y}-\cos x\cos y}{e^{x+y}+\sin x\sin y}\right)                                                     \\
				                                                       & =-\frac{\mathrm{d}\left(e^{x+y}-\cos x\cos y\right)\left(e^{x+y}+\sin x\sin y\right)}{\left(e^{x+y}+\sin x\sin y\right)^2}     \\
				                                                       & \quad+\frac{\left(e^{x+y}-\cos x\cos y\right)\mathrm{d}\left(e^{x+y}+\sin x\sin y\right)}{\left(e^{x+y}+\sin x\sin y\right)^2}
			\end{aligned}
		$$
		$\mathrm{d}\left(e^{x+y}-\cos x\cos y\right)=\left(e^{x+y}+\sin x\cos y\right)\mathrm{d}x+\left(e^{x+y}+\cos x\sin y\right)\mathrm{d}y,$\\
		$\mathrm{d}\left(e^{x+y}+\sin x\sin y\right)=\left(e^{x+y}+\cos x\sin y\right)\mathrm{d}x+\left(e^{x+y}+\sin x\cos y\right)\mathrm{d}y.$
	\end{block}
\end{frame}

\begin{frame}
	\frametitle{隐函数求微分}
	\everymath{\displaystyle}
	\begin{exampleblock}{计算栗子里的$\frac{\mathrm{d}y}{\mathrm{d}x}$和$\frac{\mathrm{d}^2y}{\mathrm{d}x^2}$}
		$$
			e^{x+y}=\sin x\cos y,\,\,\, \text{已知}\,\,\, \frac{\mathrm{d}y}{\mathrm{d}x}=-\frac{e^{x+y}-\cos x\cos y}{e^{x+y}+\sin x\sin y}.
		$$
	\end{exampleblock}
	\begin{block}{Ans}
		由于:
		$$
			\begin{aligned}
				 & \mathrm{d}\left(\frac{\mathrm{d}y}{\mathrm{d}x}\right)                                                                                                                                    \\
				 & =-\frac{\left(\left(e^{x+y}+\sin x\cos y\right)\mathrm{d}x+\left(e^{x+y}+\cos x\sin y\right)\mathrm{d}y\right)\left(e^{x+y}+\sin x\sin y\right)}{\left(e^{x+y}+\sin x\sin y\right)^2}     \\
				 & \quad+\frac{\left(e^{x+y}-\cos x\cos y\right)\left(\left(e^{x+y}+\cos x\sin y\right)\mathrm{d}x+\left(e^{x+y}+\sin x\cos y\right)\mathrm{d}y\right)}{\left(e^{x+y}+\sin x\sin y\right)^2}
			\end{aligned}
		$$
	\end{block}
\end{frame}

\begin{frame}
	\frametitle{隐函数求微分}
	\everymath{\displaystyle}
	\begin{block}{继续}
		由于:
		$$
			\begin{aligned}
				 & \frac{\mathrm{d}^2y}{\mathrm{d}x^2}=\frac{\mathrm{d}}{\mathrm{d}x}\left(\frac{\mathrm{d}y}{\mathrm{d}x}\right)                                                                                     \\
				 & =-\frac{\left(\left(e^{x+y}+\sin x\cos y\right)+\left(e^{x+y}+\cos x\sin y\right)\frac{\mathrm{d}y}{\mathrm{d}x}\right)\left(e^{x+y}+\sin x\sin y\right)}{\left(e^{x+y}+\sin x\sin y\right)^2}     \\
				 & \quad+\frac{\left(e^{x+y}-\cos x\cos y\right)\left(\left(e^{x+y}+\cos x\sin y\right)+\left(e^{x+y}+\sin x\cos y\right)\frac{\mathrm{d}y}{\mathrm{d}x}\right)}{\left(e^{x+y}+\sin x\sin y\right)^2}
			\end{aligned}
		$$
		其中, $\frac{\mathrm{d}y}{\mathrm{d}x}=-\frac{e^{x+y}-\cos x\cos y}{e^{x+y}+\sin x\sin y}$.
	\end{block}
\end{frame}

\begin{frame}
	\frametitle{隐函数求微分}
	\everymath{\displaystyle}
	\begin{block}{练习题}
		$$
			\mathrm{e}^y+x y-\mathrm{e}=0.
		$$
	\end{block}

	\begin{exampleblock}{}
		$$
		\begin{aligned}
			0&=\mathrm{d}\left(\mathrm{e}^y+x y-\mathrm{e}\right)\\
			&=y\mathrm{d}x+(e^y+x)\mathrm{d}y.
		\end{aligned}
		$$
	\end{exampleblock}
	\begin{exampleblock}{}
		$$
		0=y\mathrm{d}x+(e^y+x)\mathrm{d}y\Rightarrow\frac{\mathrm{d}y}{\mathrm{d}x}=-\frac{y}{x+e^y}.
		$$
	\end{exampleblock}
\end{frame}

\begin{frame}
	\frametitle{隐函数求微分}
	\everymath{\displaystyle}
	\begin{block}{练习题}
		$$
			\ln \sqrt{x^2+y^2}=\arctan \frac{y}{x}.
		$$
	\end{block}

	\begin{exampleblock}{}
		$$
		\begin{aligned}
			0&=\mathrm{d}\left(\ln \sqrt{x^2+y^2}-\arctan \frac{y}{x}\right)\\
			&=\frac{1}{{\sqrt{x^2+y^2}}}\cdot\frac{x\mathrm{d}x+y\mathrm{d}y}{\sqrt{x^2+y^2}}-\frac{1}{1+\left(\frac{y}{x}\right)^2}\cdot\frac{x\mathrm{d}y-y\mathrm{d}x}{x^2}\\
			&=\frac{(y-x)\mathrm{d}y+(x+y)\mathrm{d}x}{x^2+y^2}.
		\end{aligned}
		$$
	\end{exampleblock}
	\begin{exampleblock}{}
		$$
		0=(y-x)\mathrm{d}y+(x+y)\mathrm{d}x\Rightarrow\frac{\mathrm{d}y}{\mathrm{d}x}=-\frac{x+y}{x-y}.
		$$
	\end{exampleblock}
\end{frame}



\subsection{反常积分}
\begin{frame}
	\frametitle{回顾:定积分}
	\begin{block}{Riemann 积分的定义}
		设$f(x)$在区间$[a,b]$上有定义, $\mathrm{P}:a=x_0<x_1<x_2<\ldots<x_n=b$.
		在每段$[x_{i-1},x_i]$上 {\bf 任意}选取点$\xi_i$, 得到近似和
		$$
			\sum_{i=1}^n f(\xi_i)(x_i-x_{i-1}).
		$$
		如果对于$\mathrm{P}$的精细化
		我们就称$f(x)$在区间$[a,b]$上是 {\bf Riemann 可积}的, 并将这个极限值称为$f(x)$在$[a,b]$上的 {\bf Riemann 积分}, 记为
		\[
			\lim_{\lambda\to 0} \sum_{i=1}^n f(\xi_i)(x_i-x_{i-1}) := \int_a^b f(x)\mathrm{~d} x.
		\]
	\end{block}

\end{frame}
\begin{frame}
	\frametitle{归纳:定积分的性质}
	\everymath{\displaystyle}
	\begin{enumerate}
		\item 线性性: $\int_a^b[k_1f_1(x) + k_2f_2(x)] \mathrm{d} x=\int_a^b k_1f_1(x) \mathrm{~d} x+\int_a^b k_2f_2(x) \mathrm{~d}x$;
		\item 区间可加性: $\int_a^b f(x) \mathrm{~d} x=\int_a^c f(x) \mathrm{~d} x+\int_c^b f(x) \mathrm{~d} x$;
		\item 保号性: $f(x) \geq 0, x\in[a,b]$, 则 $\int_a^b f(x) \mathrm{~d} x\geq 0$;
		\item 上下界估计: $(b-a)\inf_{x\in[a,b]}f(x)=\int_a^{b} f(x) \mathrm{~d} x=(b-a)\sup_{x\in[a,b]}f(x)$;
		\item 中值定理: 设 $f(x)$ 在 $[a, b]$ 上连续, 则存在 $\xi\in[a, b]$, 使
		      $$
			      \int_a^b f(x) \mathrm{~d} x=f(\xi)(b-a).
		      $$
	\end{enumerate}
	\vspace{0.5cm}
\end{frame}

\begin{frame}
	\frametitle{回顾: Newton-Leibniz公式}

	\begin{theorem}[Newton-Leibniz公式]
		如果函数 $F(x)$ 是连续函数 $f(x)$ 在区间 $[a, b]$ 上的一个原函数, 则
		$$
			\int_a^b f(x) \mathrm{~d} x=F(b)-F(a) .
		$$
	\end{theorem}
	\vspace{1cm}
	\pause
	\begin{quote}
		原函数 $F(x)$ 满足
		$$
			F (x)+C=\int f(x)\mathrm{~d}x.
		$$
	\end{quote}
\end{frame}




% Thank you page
\beamertemplateshadingbackground{structure.fg!90}{structure.fg}
\begin{frame}[plain]
	\vfill
	\centering
	{
		\centering \Huge \color{white} Thank you for your attention!\\[10pt]Questions?
	}
	\vfill
\end{frame}


\end{document}


